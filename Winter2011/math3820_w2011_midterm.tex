\documentclass[12pt]{article}

\usepackage[active]{srcltx}

\usepackage{amsmath,amsfonts,amssymb,amsthm}
\usepackage{easybmat}
\usepackage{graphicx}
%\usepackage[hypertex]{hyperref}
%\usepackage[pdftex]{hyperref}
\usepackage{fancyhdr}
\usepackage{subfigure}

\usepackage{makeidx}
\makeindex

\theoremstyle{plain}
\newtheorem{theorem}{Theorem}
\newtheorem{property}[theorem]{Property}
\newtheorem{condition}[theorem]{Condition}
\newtheorem{proposition}[theorem]{Proposition}
\newtheorem{axiom}[theorem]{Axiom}
\newtheorem{lemma}[theorem]{Lemma}
\newtheorem{corollary}[theorem]{Corollary}
%%
% % Make numbering specific to the Appendix
% \newtheorem{Atheorem}{Theorem}[chapter]
% \newtheorem{Aproperty}[Atheorem]{Property}
% \newtheorem{Acondition}[Atheorem]{Condition}
% \newtheorem{Aproposition}[Atheorem]{Proposition}
% \newtheorem{Aaxiom}[Atheorem]{Axiom}
% \newtheorem{Alemma}[Atheorem]{Lemma}
% \newtheorem{Acorollary}[theorem]{Corollary}

% Make definitions non italicized
%\theoremstyle{definition}
\newtheorem{definition}[theorem]{Definition}
% Make numbering specific to the Appendix
%\newtheorem{Adefinition}[Atheorem]{Definition}
\newenvironment{defi}{\vskip0.2cm\addtocounter{theorem}{1}\par\noindent\bf Definition~\arabic{section}.\arabic{subsection}.\arabic{theorem}\rm.}{\hfill{$\circ$}\par\vskip0.25cm}


%% Example environment and counter.
\newcounter{cmpt_exercise}
\newenvironment{exercise}{\addtocounter{cmpt_exercise}{1}\vskip0.2cm\par\noindent\begin{small}\bf Exercise~\arabic{cmpt_exercise}\,\,\rm --}{\hfill{$\circ$}\end{small}\par\vskip0.25cm}
\newenvironment{problem}{\addtocounter{cmpt_exercise}{1}\vskip0.2cm\par\noindent{\Large\bf Problem~\arabic{cmpt_exercise}\,\,\rm --}}{\par\vskip0.25cm}


\newenvironment{example}{\vskip0.2cm\par\noindent\begin{small}\bf Example\,\,\rm --}{\hfill{$\diamond$}\end{small}\par\vskip0.25cm}
\newenvironment{remark}{\vskip0.2cm\par\noindent\begin{small}\bf Remark\,\,\rm --}{\hfill{$\circ$}\end{small}\par\vskip0.25cm}
%\newenvironment{aparte}[1]{\vskip0.2cm\par\noindent\begin{quote}\begin{small}\bf Apart\'e : #1\,\,\rm --}{\hfill{$\circ$}\end{small}\end{quote}\par\vskip0.25cm}
\newenvironment{aparte}[1]{\vskip0.3cm\par\begin{center}\begin{tabular}{|p{0.9\textwidth}|}\hline{\bf Apart\'e : #1}}{\\ \hline\end{tabular}\end{center}\par\vskip0.25cm}

\renewcommand{\labelenumi}{\roman{enumi})}
\renewcommand{\labelenumii}{\alph{enumii})}
\newcommand{\espv}{\vspace{.5\baselineskip}}
\def\IR{\mathbb{R}}
\def\IC{\mathbb{C}}
\def\IN{\mathbb{N}}
\def\IQ{\mathbb{Q}}
\def\IZ{\mathbb{Z}}
\def\rank{\textrm{rank }}
\def\Sp{\textrm{Sp }}
\def\Span{\textrm{Span }}
\def\Tr{\textrm{Tr }}
\def\D{\mathcal{D}}
\def\I{\mathcal{I}}
\def\U{\mathcal{U}}
\def\R{\mathcal{R}}
\def\Q{\mathcal{Q}}
\def\O{\mathcal{O}}
\def\Mn{\mathcal{M}_n}
\def\NN#1{\|#1\|}
\def\N3#1{|\!|\!|#1|\!|\!|}
\def\diag{\textrm{diag}}
\def\tr{\textrm{tr}}
\def\ker{\textrm{Ker }}

\def\M{\mathcal{M}}

\setlength{\textwidth}{17cm} 
\addtolength{\oddsidemargin}{-1.5cm}
\setlength{\textheight}{22cm}
\addtolength{\topmargin}{-2cm} 
\setlength{\headheight}{25.3pt}

%% Fancyhdr related stuff
\pagestyle{fancy}
\lhead{MATH 3820 -- Dynamical Systems -- Winter 2011 -- Midterm}
\rhead{\thepage}
\cfoot{}

\usepackage[hang,small,bf]{caption}
\setlength{\captionmargin}{20pt}

\makeatletter
\def\cleardoublepage{\clearpage\if@twoside \ifodd\c@page\else
\hbox{}
% \vspace*{\fill}
% \begin{center}
% This page intentionally contains only this sentence.
% \end{center}% Make numbering specific to the Appendix
\newtheorem{Atheorem}{Theorem}[chapter]
\newtheorem{Aproperty}[Atheorem]{Property}
\newtheorem{Acondition}[Atheorem]{Condition}
\newtheorem{Aproposition}[Atheorem]{Proposition}
\newtheorem{Aaxiom}[Atheorem]{Axiom}
\newtheorem{Alemma}[Atheorem]{Lemma}
\newtheorem{Acorollary}[theorem]{Corollary}

% \vspace{\fill}
\thispagestyle{empty}
\newpage
\if@twocolumn\hbox{}\newpage\fi\fi\fi}
\makeatother

%\author{Julien Arino}
%\address{University of Manitoba}
%\title{MATH 8430\\ Lecture Notes}
\title{University of Manitoba\\ Math 3820 -- Winter 2011}
\author{Midterm}
\date{Thursday, March 10, 2011}

\renewcommand{\abstractname}{Instructions}
%%%%%%%%%%%%%%%%
%%%%%%%%%%%%%%%%
%%%%%%%%%%%%%%%%
%%%%%%%%%%%%%%%%
%%%%%%%%%%%%%%%%
%%%%%%%%%%%%%%%%
\begin{document}

\maketitle
\begin{abstract}
This test is 2 hours; it has 2 questions on 2 pages. Notes are allowed.
In marking, attention will be paid to the overall legibility of solutions; so detail and structure your answers and make good use of scrap paper.
\end{abstract}

\thispagestyle{empty}


\noindent{\bf 1.}
The following equation
\begin{equation}\label{eq:Ricker}
N_{t+1}=N_t e^{r(1-N_t/K)}
\end{equation}
is called Ricker's equation and was introduced in the context of a model of fisheries. The parameter $r\geq 0$ is the intrinsic growth rate, $K>0$ is the carrying capacity. We consider equation \eqref{eq:Ricker} together with a nonnegative initial condition $N_0$.

\noindent{\bf 1.a.}
Give an interpretation of the form of the equation.

\noindent{\bf 1.b.}
Discuss the existence of solutions to \eqref{eq:Ricker}. Do solutions always remain nonnegative? bounded?

\noindent{\bf 1.c.}
Study the existence of fixed points for \eqref{eq:Ricker} and its/their stability as a function of parameter values.




\vskip1cm
\noindent{\bf 2.}
We consider the following system of ordinary differential equations
\begin{subequations}\label{sys:chemostat}
\begin{align}
\frac{d}{dt} S &= D(S^0-S)-\frac{\mu_1}{Y_1}SP \label{sys:chemostat_S} \\
\frac{d}{dt} P &= \mu_1 SP -DP-\frac{mu_2}{Y_2}PZ \label{sys:chemostat_P} \\
\frac{d}{dt} Z &= \mu_2 PZ-DZ \label{sys:chemostat_Z} 
\end{align}
\end{subequations}
together with the nonnegative initial condition $(S(0),P(0),Z(0))=(N_0,P_0,Z_0)$. With this system, we aim to describe the interactions between a substrate $S$, phytoplankton $P$ and zooplankton $Z$ in a chemostat. All parameters are assumed to be positive.

\noindent{\bf 2.a.}
Give an interpretation of the form of \eqref{sys:chemostat}. The parameters $Y_1,Y_2$ are called \emph{yield coefficients}. Can you see why? Also, beside the addition of a zooplankton equation, what is the difference with the chemostat model we studied? Discuss this difference and its implication in terms of the assumptions made on the nutrient uptake/growth mechanism.

\noindent{\bf 2.b.}
Do solutions to \eqref{sys:chemostat} exist? are they unique? do they remain nonnegative? are they bounded?

\noindent{\bf 2.c.}
We propose to \emph{nondimensionalize} the system to transform it into a system easier to study yet possessing the same properties. For this, we introduce the following change of variables
\[
x=\frac{S}{S^0},\quad y=\frac{P}{Y_1S^0},\quad z=\frac{Z}{Y_1Y_2S^0}
\]
and of time
\[
t=DT.
\]
(Note that $d/dT=D\; d/dt$.)
Show that, using these, \eqref{sys:chemostat} can be transformed into the following system
\begin{subequations}\label{sys:chemostat_nondim}
\begin{align}
\frac{d}{dt} x &= 1-x-Axy \label{sys:chemostat_nondim_x} \\
\frac{d}{dt} y &= Axy-y-Byz \label{sys:chemostat_nondim_y} \\
\frac{d}{dt} z &= Byz-z, \label{sys:chemostat_nondim_z}
\end{align}
\end{subequations}
where
\[
A=\frac{\mu_1S^0}{D},\quad \beta=\frac{\mu_2S^0Y_1}{D}.
\]
\noindent{\bf 2.d.}
Verify that the properties established in {\bf 2.b} are still true for \eqref{sys:chemostat_nondim}.

\noindent{\bf 2.e.}
Show that an (asymptotic) mass conservation property holds for \eqref{sys:chemostat_nondim}, so that, for all nonnegative initial conditions,
\[
\lim_{t\to\infty} x(t)+y(t)+z(t)=1.
\]

\noindent{\bf 2.f.}
Use the result of {\bf 2.e} to write \eqref{sys:chemostat_nondim} as a 2-dimensional system involving the variables $y$ and $z$. This new system has 2 equilibria. Find them and study their local stability properties as a function of the parameters they involve. Summarize the situation as a function of the parameters.

\noindent{\bf 2.g.}
Express the results of {\bf 2.f} for system \eqref{sys:chemostat_nondim}. Can you interpret these results? Can you relate these results to the original system \eqref{sys:chemostat}? Discuss the pros and cons of the original system versus the nondimensionalized system.


\end{document}