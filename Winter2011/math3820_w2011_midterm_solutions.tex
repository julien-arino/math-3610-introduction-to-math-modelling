\documentclass[12pt]{article}

\usepackage[active]{srcltx}

\usepackage{amsmath,amsfonts,amssymb,amsthm}
\usepackage{easybmat}
\usepackage{graphicx}
%\usepackage[hypertex]{hyperref}
%\usepackage[pdftex]{hyperref}
\usepackage{fancyhdr}
\usepackage{subfigure}

\usepackage{makeidx}
\makeindex

\theoremstyle{plain}
\newtheorem{theorem}{Theorem}
\newtheorem{property}[theorem]{Property}
\newtheorem{condition}[theorem]{Condition}
\newtheorem{proposition}[theorem]{Proposition}
\newtheorem{axiom}[theorem]{Axiom}
\newtheorem{lemma}[theorem]{Lemma}
\newtheorem{corollary}[theorem]{Corollary}
%%
% % Make numbering specific to the Appendix
% \newtheorem{Atheorem}{Theorem}[chapter]
% \newtheorem{Aproperty}[Atheorem]{Property}
% \newtheorem{Acondition}[Atheorem]{Condition}
% \newtheorem{Aproposition}[Atheorem]{Proposition}
% \newtheorem{Aaxiom}[Atheorem]{Axiom}
% \newtheorem{Alemma}[Atheorem]{Lemma}
% \newtheorem{Acorollary}[theorem]{Corollary}

% Make definitions non italicized
%\theoremstyle{definition}
\newtheorem{definition}[theorem]{Definition}
% Make numbering specific to the Appendix
%\newtheorem{Adefinition}[Atheorem]{Definition}
\newenvironment{defi}{\vskip0.2cm\addtocounter{theorem}{1}\par\noindent\bf Definition~\arabic{section}.\arabic{subsection}.\arabic{theorem}\rm.}{\hfill{$\circ$}\par\vskip0.25cm}


%% Example environment and counter.
\newcounter{cmpt_exercise}
\newenvironment{exercise}{\addtocounter{cmpt_exercise}{1}\vskip0.2cm\par\noindent\begin{small}\bf Exercise~\arabic{cmpt_exercise}\,\,\rm --}{\hfill{$\circ$}\end{small}\par\vskip0.25cm}
\newenvironment{problem}{\addtocounter{cmpt_exercise}{1}\vskip0.2cm\par\noindent{\Large\bf Problem~\arabic{cmpt_exercise}\,\,\rm --}}{\par\vskip0.25cm}


\newenvironment{example}{\vskip0.2cm\par\noindent\begin{small}\bf Example\,\,\rm --}{\hfill{$\diamond$}\end{small}\par\vskip0.25cm}
\newenvironment{remark}{\vskip0.2cm\par\noindent\begin{small}\bf Remark\,\,\rm --}{\hfill{$\circ$}\end{small}\par\vskip0.25cm}
%\newenvironment{aparte}[1]{\vskip0.2cm\par\noindent\begin{quote}\begin{small}\bf Apart\'e : #1\,\,\rm --}{\hfill{$\circ$}\end{small}\end{quote}\par\vskip0.25cm}
\newenvironment{aparte}[1]{\vskip0.3cm\par\begin{center}\begin{tabular}{|p{0.9\textwidth}|}\hline{\bf Apart\'e : #1}}{\\ \hline\end{tabular}\end{center}\par\vskip0.25cm}

\renewcommand{\labelenumi}{\roman{enumi})}
\renewcommand{\labelenumii}{\alph{enumii})}
\newcommand{\espv}{\vspace{.5\baselineskip}}
\def\IR{\mathbb{R}}
\def\IC{\mathbb{C}}
\def\IN{\mathbb{N}}
\def\IQ{\mathbb{Q}}
\def\IZ{\mathbb{Z}}
\def\rank{\textrm{rank }}
\def\Sp{\textrm{Sp }}
\def\Span{\textrm{Span }}
\def\Tr{\textrm{Tr }}
\def\D{\mathcal{D}}
\def\I{\mathcal{I}}
\def\U{\mathcal{U}}
\def\R{\mathcal{R}}
\def\Q{\mathcal{Q}}
\def\O{\mathcal{O}}
\def\Mn{\mathcal{M}_n}
\def\NN#1{\|#1\|}
\def\N3#1{|\!|\!|#1|\!|\!|}
\def\diag{\textrm{diag}}
\def\tr{\textrm{tr}}
\def\ker{\textrm{Ker }}

\def\M{\mathcal{M}}

\setlength{\textwidth}{17cm} 
\addtolength{\oddsidemargin}{-1.5cm}
\setlength{\textheight}{22cm}
\addtolength{\topmargin}{-2cm} 
\setlength{\headheight}{25.3pt}

%% Fancyhdr related stuff
\pagestyle{fancy}
\lhead{MATH 3820 -- Dynamical Systems -- Winter 2011 -- Midterm}
\rhead{\thepage}
\cfoot{}

\usepackage[hang,small,bf]{caption}
\setlength{\captionmargin}{20pt}

\makeatletter
\def\cleardoublepage{\clearpage\if@twoside \ifodd\c@page\else
\hbox{}
% \vspace*{\fill}
% \begin{center}
% This page intentionally contains only this sentence.
% \end{center}% Make numbering specific to the Appendix
\newtheorem{Atheorem}{Theorem}[chapter]
\newtheorem{Aproperty}[Atheorem]{Property}
\newtheorem{Acondition}[Atheorem]{Condition}
\newtheorem{Aproposition}[Atheorem]{Proposition}
\newtheorem{Aaxiom}[Atheorem]{Axiom}
\newtheorem{Alemma}[Atheorem]{Lemma}
\newtheorem{Acorollary}[theorem]{Corollary}

% \vspace{\fill}
\thispagestyle{empty}
\newpage
\if@twocolumn\hbox{}\newpage\fi\fi\fi}
\makeatother

%\author{Julien Arino}
%\address{University of Manitoba}
%\title{MATH 8430\\ Lecture Notes}
\title{University of Manitoba\\ Math 3820 -- Winter 2011}
\author{Midterm -- Solutions}
\date{Thursday, March 10, 2011}

\renewcommand{\abstractname}{Comments}
%%%%%%%%%%%%%%%%
%%%%%%%%%%%%%%%%
%%%%%%%%%%%%%%%%
%%%%%%%%%%%%%%%%
%%%%%%%%%%%%%%%%
%%%%%%%%%%%%%%%%
\begin{document}

\maketitle
\begin{abstract}
Question 1 was very short and should have posed no major problem, except maybe the boundedness of solutions. The maximum mark was 40.
\end{abstract}

\thispagestyle{empty}


\noindent{\bf 1. [10 pts total]}
The following equation
\begin{equation}\label{eq:Ricker}
N_{t+1}=N_t e^{r(1-N_t/K)}
\end{equation}
is called Ricker's equation and was introduced in the context of a model of fisheries. The parameter $r\geq 0$ is the intrinsic growth rate, $K>0$ is the carrying capacity. We consider equation \eqref{eq:Ricker} together with a nonnegative initial condition $N_0$.

\noindent{\bf 1.a. [2 pts]}
Give an interpretation of the form of the equation.

\noindent{\bf Solution.}
The term $1-N_t/K$ is positive if $N_t<K$, negative if $N_t>K$ and zero if $N_t=K$; so these situations correspond to growth, decrease and stationarity of the population, respectively. The magnitude of the change, when there is some change, is $r$.


\noindent{\bf 1.b. [3 pts]}
Discuss the existence of solutions to \eqref{eq:Ricker}. Do solutions always remain nonnegative? bounded?

\noindent{\bf Solution.}
Solutions exist for all initial conditions, as the function
\[
f(N)=Ne^{r(1-N/K)}
\]
is continuous from $\IR$ to $\IR$. Also, $f:\IR_+\to\IR_+$, so solutions remain nonnegative for nonnegative initial conditions. Also, we have
\[
f'(N)=e^{r(1-N/K)}-\frac{rN}{K}e^{r(1-N/K)}=\frac{K-rN}{K}e^{r(1-N/K)}.
\]
As the exponential in this expression is always positive, the sign of $f'$ depends on that of $K-rN$. We have $K-rN<0$ if $N>K/r$, so any value of $N_t$ larger than $K/r$ leads to a solution that decreases until it becomes smaller than $K/r$. Therefore, the largest possible value of $N_t$ is $N_0$ (if the initial condition is for example much larger than $K$ or $K/r$), or 
\[
N_{max}=\max_{N\in[0,K/r]}f(N),
\]
which we know to exist since $f$ continuous. Then, for all $t$, 
\[
N_t\leq\max\{N_{max},N_0\}.
\]
In other words, solutions are bounded.


\noindent{\bf 1.c. [5 pts]}
Study the existence of fixed points for \eqref{eq:Ricker} and its/their stability as a function of parameter values.

\noindent{\bf Solution.}
First, note that if $r=0$, $N_{t+1}=N_t$, and therefore, given an initial condition $N_0$, $N_{t}=N_0$ for all $t$, implying that $N_0$ is a stable fixed point. Note that here, $f'(N)=1$ for all $N$, so to conclude about stability, we use the definition.

Other (non degenerate) fixed point(s) of \eqref{eq:Ricker} in the case $r>0$ satisfy the fixed point equation
\[
p=pe^{r(1-p/K)}.
\]
Clearly, $p=0$ is a solution. Also, assuming $p\neq 0$, the fixed point equation is equivalent to
\[
1=e^{r(1-p/K)}.
\]
Taking the logarithm,
\[
r(1-p/K)=0,
\]
or, in other words, $p=K$. So we have two fixed points, $N=0$ and $N=K$. 

At $N=0$,
\[
f'(0)=e^r,
\]
so $N=0$ is attractive if $e^r<1$, i.e., $r<0$. Since we have assumed $r\geq 0$, this situation is impossible and $N=0$ is always repelling.

At $N=K$,
\[
f'(K)=1-r,
\]
so $N=K$ is attractive if $|1-r|<1$, i.e., $0<r<2$, and repelling otherwise.





\newpage
\noindent{\bf 2. [30 pts total]}
We consider the following system of ordinary differential equations
\begin{subequations}\label{sys:chemostat}
\begin{align}
\frac{d}{dt} S &= D(S^0-S)-\frac{\mu_1}{Y_1}SP \label{sys:chemostat_S} \\
\frac{d}{dt} P &= \mu_1 SP -DP-\frac{\mu_2}{Y_2}PZ \label{sys:chemostat_P} \\
\frac{d}{dt} Z &= \mu_2 PZ-DZ \label{sys:chemostat_Z} 
\end{align}
\end{subequations}
together with the nonnegative initial condition $(S(0),P(0),Z(0))=(S_0,P_0,Z_0)$. With this system, we aim to describe the interactions between a substrate $S$, phytoplankton $P$ and zooplankton $Z$ in a chemostat. All parameters are assumed to be positive.

\noindent{\bf 2.a. [3 pts]}
Give an interpretation of the form of \eqref{sys:chemostat}. The parameters $Y_1,Y_2$ are called \emph{yield coefficients}. Can you see why? Also, beside the addition of a zooplankton equation, what is the difference with the chemostat model we studied? Discuss this difference and its implication in terms of the assumptions made on the nutrient uptake/growth mechanism.

\noindent{\bf Solution.}
This is a regular chemostat model but with an extra species tacked on top of the resident species. This new species consumes the species that consumes the substrate. $Y_1$ and $Y_2$ describe the conversion between absorption of substrate and growth and absorption of phytoplankton and growth, respectively, as ingesting 1 unit of food does not necessarily translate into growth by 1 unit.

The absorption/growth function is here of mass-action type, meaning that contrary to a Michaelis-Menten type kinetic, it does not saturate with high values of $S$.

\noindent{\bf 2.b. [5 pts]}
Do solutions to \eqref{sys:chemostat} exist? are they unique? do they remain nonnegative? are they bounded?

\noindent{\bf Solution.}
Clearly, the vector field is $C^1$ (it is actually $C^\infty$, as a collection of multivariate polynomials), so solutions exist and are unique. 

Nonnegativity is established as usual. Assume nonnegative initial conditions, and assume $S(t_1)=0$ for some $t_1\geq 0$ in \eqref{sys:chemostat_S}. Maybe $t_1=0$, i.e., $S_0=0$. In that case, $S'=DS^0$ and $S(t)>0$ for $t>0$. Otherwise, $t_1>0$, but this is impossible since if $S(t_1)=0$, then $S'(t_1)\leq 0$ because $S$ must be decreasing there, and yet $S'(t_1)=DS^0>0$, a contradiction. Therefore, the only possibility for $S(t)=0$ is when $S(0)=S_0=0$.

Inspecting \eqref{sys:chemostat_P}, we see that $P=0$ is positively invariant under the flow of \eqref{sys:chemostat}. Because of uniqueness of solutions, any solution that has $P=0$ therefore needs to have $P(0)=P_0=0$ and any solution that has $P_0>0$ can never have $P(t)=0$. The situation is similar for \eqref{sys:chemostat_Z} for $Z$.
Therefore, solutions remain nonnegative for nonnegative initial conditions.

To establish boundedness, consider the variable $M=S+P+Z$ representing the total biomass (organic and inorganic) in the chemostat. We have
\begin{align*}
M' &= (S+P+Z)' \\
&= DS^0-DS-DP-DZ-\frac{1-Y_1}{Y_1}\mu_1 SP-\frac{1-Y_2}{Y_2}\mu_2 PZ \\
&\leq D(S^0-M),
\end{align*}
where we have assumed that $Y_i\leq 1$. Note that this is a valid hypothesis: conversion always occurs at a loss (or at least, does not create more mass than is present), so absorption of $N$ units of nutrient always results in growth by an amount $YN$, with $Y\leq 1$.

As a consequence, $\lim_{t\to\infty}M(t)=S^0$. Since solutions are nonnegative, this implies that they are bounded.


\noindent{\bf 2.c. [6 pts]}
We propose to \emph{nondimensionalize} the system to transform it into a system easier to study yet possessing the same properties. For this, we introduce the following change of variables
\[
x=\frac{S}{S^0},\quad y=\frac{P}{Y_1S^0},\quad z=\frac{Z}{Y_1Y_2S^0}
\]
and of time
\[
t=DT.
\]
(Note that $d/dT=D\; d/dt$.)
Show that, using these, \eqref{sys:chemostat} can be transformed into the following system
\begin{subequations}\label{sys:chemostat_nondim}
\begin{align}
\frac{d}{dT} x &= 1-x-Axy \label{sys:chemostat_nondim_x} \\
\frac{d}{dT} y &= Axy-y-Byz \label{sys:chemostat_nondim_y} \\
\frac{d}{dT} z &= Byz-z, \label{sys:chemostat_nondim_z}
\end{align}
\end{subequations}
where
\[
A=\frac{\mu_1S^0}{D},\quad \beta=\frac{\mu_2S^0Y_1}{D}.
\]

\noindent{\bf Solution.}
We have
\[
\frac{d}{dt} x=\frac{1}{S^0}\left(\frac{d}{dt}S\right),\quad
\frac{d}{dt} y=\frac{1}{Y_1S^0}\left(\frac{d}{dt}P\right)
\]
and
\[
\frac{d}{dt} z=\frac{1}{Y_1Y_2S^0}\left(\frac{d}{dt}Z\right).
\]
Therefore, \eqref{sys:chemostat} is equivalent to
\begin{align*}
\frac{d}{dt} x &= D(S^0-S)-\frac{\mu_1}{Y_1}SP \label{sys:chemostat_S} \\
\frac{d}{dt} y &= \mu_1 SP -DP-\frac{\mu_2}{Y_2}PZ \label{sys:chemostat_P} \\
\frac{d}{dt} z &= \mu_2 PZ-DZ \label{sys:chemostat_Z} 
\end{align*}




\noindent{\bf 2.d. [4 pts]}
Verify that the properties established in {\bf 2.b} are still true for \eqref{sys:chemostat_nondim}.

\noindent{\bf Solution.}

\noindent{\bf 2.e. [2 pts]}
Show that an (asymptotic) mass conservation property holds for \eqref{sys:chemostat_nondim}, so that, for all nonnegative initial conditions,
\[
\lim_{t\to\infty} x(t)+y(t)+z(t)=1.
\]

\noindent{\bf Solution.}
Let $M=x+y+z$. Then
\[
M'=1-(x+y+z)=1-M,
\]
and as previously, $\lim_{t\to\infty}M(t)=\lim_{t\to\infty} x(t)+y(t)+z(t)=1$.


\noindent{\bf 2.f. [8 pts]}
Use the result of {\bf 2.e} to write \eqref{sys:chemostat_nondim} as a 2-dimensional system involving the variables $y$ and $z$. This new system has 3 equilibria. Find them and study their local stability properties as a function of the parameters they involve. Summarize the situation as a function of the parameters.

\noindent{\bf Solution.}
Since $\lim_{t\to\infty} x(t)+y(t)+z(t)=1$, we can assume that $x=1-y-z$. Neglecting the equation for $x'$, \eqref{sys:chemostat_nondim} thus transforms into
\begin{align*}
\frac{d}{dT} y &= A(1-y-z)y-y-Byz \\
\frac{d}{dT} z &= Byz-z,
\end{align*}
which we can write as
\begin{subequations}\label{sys:chemostat_nondim_reduced}
\begin{align}
\frac{d}{dT} y &= -\{A(y+z-1)+Bz+1\}y \label{sys:chemostat_nondim_reduced_y} \\
\frac{d}{dT} z &= (By-1)z. \label{sys:chemostat_nondim_reduced_z}
\end{align}
\end{subequations}
At equilibrium, from \eqref{sys:chemostat_nondim_reduced_z}, $z=0$ or $y=1/B$.

Substitute $z=0$ into \eqref{sys:chemostat_nondim_reduced_y} at equilibrium:
\[
-\{A(y-1)+1\}y=0,
\]
so $y=0$ or $y=1+1/A$. So we find 2 equilibria, $(y,z)=(0,0)$ and $(y,z)=(1+1/A,0)$. Now substitute $y=1/B$ into \eqref{sys:chemostat_nondim_reduced_y} at equilibrium:
\[
-\left\{A\left(\frac 1B+z-1\right)+Bz+1\right\}\frac 1B=0,
\]
that is
\[
A\left(\frac 1B+z-1\right)+Bz+1=0,
\]
or, in other words,
\[
z=\frac{A-\dfrac AB-1}{A+B}.
\]
So we find a third equilibrium, 
\[
(y,z)=\left(\frac 1B,\frac{AB-A-B}{(A+B)B}\right).
\]
[Note: the text had a typo -I gave full marks to anybody who had studied 2 EPs, bonus marks to those who saw the typo and studied 3]





\noindent{\bf 2.g. [2 pts]}
Express the results of {\bf 2.f} for system \eqref{sys:chemostat_nondim}. Can you interpret these results? Can you relate these results to the original system \eqref{sys:chemostat}? Discuss the pros and cons of the original system versus the nondimensionalized system.

\noindent{\bf Solution.}
We have to add the third variable, $x=1-y-z$.

The expressions are nondimensional, so it is hard to interpret them biologically. The equilibria were, however, easier to compute using the nondimensionalized form. In general: nondimensionalization is useful for computations as well as numerical simulations (because the numbers are all of the same order). They are, however, difficult to interpret in terms of the original problem. You will often see this: a paper that is more mathematical will use nondimensionalization; a paper that is more geared towards modelling does not.


\end{document}