\documentclass[12pt]{article}

\usepackage[active]{srcltx}

\usepackage{amsmath,amsfonts,amssymb,amsthm}
\usepackage{easybmat}
\usepackage{graphicx}
%\usepackage[hypertex]{hyperref}
%\usepackage[pdftex]{hyperref}
\usepackage{fancyhdr}
\usepackage{subfigure}

%\usepackage{makeidx}
%\makeindex

\theoremstyle{plain}
\newtheorem{theorem}{Theorem}
\newtheorem{property}[theorem]{Property}
\newtheorem{condition}[theorem]{Condition}
\newtheorem{proposition}[theorem]{Proposition}
\newtheorem{axiom}[theorem]{Axiom}
\newtheorem{lemma}[theorem]{Lemma}
\newtheorem{corollary}[theorem]{Corollary}
%%
% % Make numbering specific to the Appendix
% \newtheorem{Atheorem}{Theorem}[chapter]
% \newtheorem{Aproperty}[Atheorem]{Property}
% \newtheorem{Acondition}[Atheorem]{Condition}
% \newtheorem{Aproposition}[Atheorem]{Proposition}
% \newtheorem{Aaxiom}[Atheorem]{Axiom}
% \newtheorem{Alemma}[Atheorem]{Lemma}
% \newtheorem{Acorollary}[theorem]{Corollary}

% Make definitions non italicized
%\theoremstyle{definition}
\newtheorem{definition}[theorem]{Definition}
% Make numbering specific to the Appendix
%\newtheorem{Adefinition}[Atheorem]{Definition}
\newenvironment{defi}{\vskip0.2cm\addtocounter{theorem}{1}\par\noindent\bf
Definition~\arabic{section}.\arabic{subsection}.\arabic{theorem}\rm.}{\hfill{
$\circ$}\par\vskip0.25cm}


%% Example environment and counter.
\newcounter{cmpt_exercise}
\newenvironment{exercise}{\addtocounter{cmpt_exercise}{1}
\vskip0.2cm\par\noindent\begin{small}\bf Exercise~\arabic{cmpt_exercise}\,\,\rm
--}{\hfill{$\circ$}\end{small}\par\vskip0.25cm}
\newenvironment{problem}{\addtocounter{cmpt_exercise}{1}\vskip0.2cm\par\noindent
{\Large\bf Problem~\arabic{cmpt_exercise}\,\,\rm --}}{\par\vskip0.25cm}


\newenvironment{example}{\vskip0.2cm\par\noindent\begin{small}\bf Example\,\,\rm
--}{\hfill{$\diamond$}\end{small}\par\vskip0.25cm}
\newenvironment{remark}{\vskip0.2cm\par\noindent\begin{small}\bf Remark\,\,\rm
--}{\hfill{$\circ$}\end{small}\par\vskip0.25cm}
%\newenvironment{aparte}[1]{\vskip0.2cm\par\noindent\begin{quote}\begin{small}
%\bf Apart\'e : #1\,\,\rm
%--}{\hfill{$\circ$}\end{small}\end{quote}\par\vskip0.25cm}
\newenvironment{aparte}[1]{\vskip0.3cm\par\begin{center}\begin{tabular}{|p{
0.9\textwidth}|}\hline{\bf Apart\'e : #1}}{\\
\hline\end{tabular}\end{center}\par\vskip0.25cm}

\renewcommand{\labelenumi}{\roman{enumi})}
\renewcommand{\labelenumii}{\alph{enumii})}
\newcommand{\espv}{\vspace{.5\baselineskip}}
\def\IR{\mathbb{R}}
\def\IC{\mathbb{C}}
\def\IN{\mathbb{N}}
\def\IQ{\mathbb{Q}}
\def\IZ{\mathbb{Z}}
\def\rank{\textrm{rank }}
\def\Sp{\textrm{Sp }}
\def\Span{\textrm{Span }}
\def\Tr{\textrm{Tr }}
\def\D{\mathcal{D}}
\def\I{\mathcal{I}}
\def\U{\mathcal{U}}
\def\R{\mathcal{R}}
\def\Q{\mathcal{Q}}
\def\O{\mathcal{O}}
\def\Mn{\mathcal{M}_n}
\def\NN#1{\|#1\|}
\def\N3#1{|\!|\!|#1|\!|\!|}
\def\diag{\textrm{diag}}
\def\tr{\textrm{tr}}
\def\ker{\textrm{Ker }}

\def\M{\mathcal{M}}

\setlength{\textwidth}{17cm} 
\addtolength{\oddsidemargin}{-1.5cm}
\setlength{\textheight}{24cm}
\addtolength{\topmargin}{-2.5cm} 
\setlength{\headheight}{25.3pt}

%% Fancyhdr related stuff
\pagestyle{fancy}
\lhead{MATH 3820 -- Intro Math Modelling -- Assignment 3}
\rhead{\thepage}
\cfoot{}

\usepackage[hang,small,bf]{caption}
\setlength{\captionmargin}{20pt}

\makeatletter
\def\cleardoublepage{\clearpage\if@twoside \ifodd\c@page\else
\hbox{}
% \vspace*{\fill}
% \begin{center}
% This page intentionally contains only this sentence.
% \end{center}% Make numbering specific to the Appendix
\newtheorem{Atheorem}{Theorem}[chapter]
\newtheorem{Aproperty}[Atheorem]{Property}
\newtheorem{Acondition}[Atheorem]{Condition}
\newtheorem{Aproposition}[Atheorem]{Proposition}
\newtheorem{Aaxiom}[Atheorem]{Axiom}
\newtheorem{Alemma}[Atheorem]{Lemma}
\newtheorem{Acorollary}[theorem]{Corollary}

% \vspace{\fill}
\thispagestyle{empty}
\newpage
\if@twocolumn\hbox{}\newpage\fi\fi\fi}
\makeatother

%\author{Julien Arino}
%\address{University of Manitoba}
%\title{MATH 8430\\ Lecture Notes}
\title{University of Manitoba\\ Math 3820 -- Winter 2009}
\author{Assignment 3\\ Preparing the analysis of the model}
\date{Due Thursday, April 2, 2009}

\renewcommand{\abstractname}{Instructions}
%%%%%%%%%%%%%%%%
%%%%%%%%%%%%%%%%
%%%%%%%%%%%%%%%%
%%%%%%%%%%%%%%%%
%%%%%%%%%%%%%%%%
%%%%%%%%%%%%%%%%
\begin{document}

\maketitle
\begin{abstract}
This assignment is due Thursday, April 2. Note that I will accept files, so if
you want to typeset or scan your answers, this is fine (pdf is preferred).
\end{abstract}
The aim of this assignment is to make sure you start thinking about and working on the mathematical and numerical aspects of the project. Given the model you presented in Assignment 2, you will have to address problems in two directions. At this point, you do not yet have to carry out the steps that you describe here if they are long, but you have to effectively set them up. Carry out short computations, write down the matlab/octave/whatever code that will be required, or use pseudo-code.
\begin{description}
\item[Mathematical analysis]
Conducting some mathematical analysis of your system might be difficult, but it is good if you try to do as much as possible.
\begin{enumerate}
\item Can you say anything about how well formulated the model is? In the case of differential equations, you will have to show well-posedness. For models of other types, similar approaches might be relevant. Try to think of questions that might help you ascertain that the model is ``good'' for its purpose.
\item How about the mathematical analysis of the problem? Is it feasible? What tools will you have to use? Can some computations be carried out using Maple?
\end{enumerate}
\item[Numerical simulations] Running numerical simulations on a problem requires some organization.
\begin{enumerate}
\item What programming language will you use for your simulations?
\item What do you want to accomplish with your numerical simulations/numerical work? Numerics should not be carried out without an idea of what they are used for. Say that your analysis shows you there are 4 different cases. You might then want to investigate these 4 different cases more in detail numerically. 
\item In the same direction as above: think of representation. What will you want to plot? Of course, you can think of time/state representations, but some more advanced representations might help you better understand/explain the output of the model. For example: with the logistic map, you can plot $N_t$ as a function of $t$. But a more illustrative plot is the one that shows the limiting behaviour of the solution $N_t$ as a function of the parameter $r$, that is, the bifurcation diagram showing the cascade of bifurcations to chaos.
\item In pseudo-code (for large programs) or in the programming language of your choice, write the main functions that you will use for your simulation. For example: if my problem involves solving a system of differential equations, then I would write down the matlab code for the right hand side of the system. 
\end{enumerate}
\end{description}
The length of this assignment is up to you. It should be at least three pages, but bear in mind that the more you do now, the less you will have to do later (and the more feedback you get from me early on).



\end{document}