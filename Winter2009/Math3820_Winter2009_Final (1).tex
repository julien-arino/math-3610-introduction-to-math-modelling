\documentclass[12pt]{exam}



\usepackage{graphicx}% Include figure files
\usepackage{dcolumn}% Align table columns on decimal point
\usepackage{bm}% bold math
\usepackage{latexsym}
\usepackage{amsmath,amssymb,amsthm,amsfonts}
\usepackage{subfigure}

\begin{document}
\pagestyle{head} \extraheadheight{1in}\runningheadrule
\firstpageheader{$\mbox{}$\\DATE: April 14, 2009\\PAPER NO.: - \\
DEPARTMENT $\&$ COURSE NO.: MATH 3820\\EXAMINATION: Intro. Math. Modelling}{\bf UNIVERSITY OF MANITOBA\\
$\mbox{}$
\\$\mbox{}$ \\ $\mbox{}$\\$\mbox{}$}{$\mbox{}$\\Final\\ PAGE NO.: \thepage\ of \numpages\\TIME: 180
minutes\\ EXAMINER: J. Arino}\runningheadrule
\runningheader{$\mbox{}$\\DATE: April 14, 2009\\PAPER NO.: - \\
DEPARTMENT $\&$ COURSE NO.: MATH 3820\\EXAMINATION: Intro. Math. Modelling}{\bf UNIVERSITY OF MANITOBA\\
$\mbox{}$
\\$\mbox{}$ \\ $\mbox{}$\\$\mbox{}$}{$\mbox{}$\\Final\\ PAGE NO.: \thepage\ of \numpages\\TIME: 180
minutes\\ EXAMINER: J. Arino} \vspace{-2.5cm}
\begin{center}
\begin{tabular}{cp{15.5cm}}
\hline
&\\
\end{tabular}
\end{center}
%\footer{}{Page \thepage\ of \numpages}%
%{\iflastpage{End of exam.}{Please go on to the next page\ldots}}

\addpoints

This is a 180 minutes exam, with \numquestions\ questions for a total of \numpoints\; marks. Lecture Notes are allowed.
{\sc Please show your work clearly.} A correct answer without explanation will not get full marks.
\vspace{0.1cm}

\begin{center}
\begin{tabular}{cp{15.5cm}}
\hline
&\\
\end{tabular}
\end{center}





\begin{questions}

\question[20]\textbf{(An arms race model)}
Two countries are involved in an arms race. Let $x_i(t)$ be the defense expenditure of country $i$ at time $t$. For a given country $i=1,2$, in the absence of the other country, the expenditure tends to decrease at a rate $-a_{ii}$ proportional to the current expenditure. Each country also spends a fixed amount $p_i$ on military equipment. When other countries are present, country $i$ buys military equipment at a rate $a_{ij}$ proportional to the expenditure of country $j$.
\begin{parts}
\part Explain briefly why the model
\begin{equation}\label{sys:arms_race}
\begin{aligned}
x_1' &= -a_{11}x_1+a_{12}x_2+p_1 \\
x_2' &= a_{21}x_1-a_{22}x_2+p_2 
\end{aligned}
\end{equation}
with $a_{ij}>0$, is an appropriate model for the arms race described above.
\part Write \eqref{sys:arms_race} in vector form.
\part Discuss the well-posedness of \eqref{sys:arms_race} (existence and uniqueness of solutions, nonnegativity, boundedness..).
\part Suppose first that $p_i=0$ for $i=1,2$. Determine the equilibrium of \eqref{sys:arms_race}.
\part From now on, assume that $p_i>0$. Determine the equilibrium of \eqref{sys:arms_race}.
\part The matrix of the system in vector form is diagonalizable. Find this diagonal form, and express the general solution to \eqref{sys:arms_race}.
\end{parts}


\question[20]
The probability that a machine functions without failure today is:
\begin{enumerate}\renewcommand{\labelenumi}{\roman{enumi})}
\item 0.7 if the machine functioned without failure yesterday and the day before yesterday,
\item 0.5 if the machine functioned without failure yesterday, but not the day before yesterday,
\item 0.4 if the machine functioned without failure the day before yesterday, but not yesterday,
\item 0.2 if the machine had failures yesterday and the day before yesterday.
\end{enumerate}
Answer the following questions.
\begin{parts}
\part Explain why the chain with two states, ``machine functions without failure at day $n$'' and ``machine not functions without failure at day $n$'' is not a Markov chain.
\part What states should you consider to have a Markov chain? From now on, consider the Markov chain with those states.
\part Find the transition matrix associated to the Markov chain.
\part Draw the transition graph associated to the Markov chain.
\part Is the chain regular?
\part Calculate the probability that the machine will function correctly tomorrow, given that it functioned correctly yesterday and the day before yesterday.
\end{parts}

\question[15]
Consider a Markov chain with transition matrix
\[
P=
\begin{pmatrix}
1/2 & 1/4 & 1/4 \\
\alpha & 1-\alpha & 0 \\
0 & \alpha & 1-\alpha
\end{pmatrix},
\]
where $0\leq\alpha\leq 1$. 
\begin{parts}
\part For what values of $\alpha$ is the Markov chain regular?
\part Calculate the equilibrium probability distribution for the values of $\alpha$ found in (a).
\end{parts}

\question[10]
Show that the two-dimensional system
\begin{align*}
x_{t+1} &= x_t(1+x_t+y_t)/3 \\
y_{t+1} &= y_t(1-x_t+y_2)/2
\end{align*}
has four fixed points, but only one that is locally stable.

\question[20]
Consider the following competition model for two species with population sizes $x$ and $y$:
\begin{equation}\label{sys:compet}
\begin{aligned}
x' &= 2x(1-3x-y) \\
y' &= 6y(1-2x-4y).
\end{aligned}
\end{equation}
\begin{parts}
\part Discuss the well-posedness of \eqref{sys:compet}.
\part Sketch the phase plane, including the nullclines and equilibria. Show the general direction of flow at different parts of the phase plane.
\part Study the stability of the equilibria.
\part If $x(0),y(0)>0$, find $\lim_{t\to\infty}(x(t),y(t))$, if this limit exists.
\end{parts}


\question[30] \textbf{(Spreading rumors)}
Construct a model for spreading rumors making use of the following assumptions.  The total population size $N$ is
constant with no one entering or leaving the population.  The population
is split into three subgroups: the uncool (who have not heard the rumor),
the rumor spreaders, and the cool (who have heard the rumor, but are no
longer spreading it).  Let the sizes of these groups be given by $U$,
$R$, and $C$ respectively.

As soon as someone in the uncool group hears the rumor, they enter the
group of rumor spreaders.  Each rumor spreader tells the rumor at rate
$k$.  Thus, the rate at which the rumor is being told is $k$ times the
number of rumor spreaders.  So, the rate at which uncool people hear the
rumor is $k$ times the number of rumor spreaders times the \textit{fraction} of the population consisting of uncool people.

When a rumor spreader tells the rumor to someone who has already heard
it (i.e. someone in the rumor spreading group or the cool group), they
stop spreading the rumor and enter the cool group.  Thus, the rate at
which rumor spreaders enter the cool group is $k$ times the number of
rumor spreaders times the \textit{fraction} of the population consisting
of rumor spreaders and cool people.
\begin{parts}
\part Draw the transfer diagram.
\part Write down the differential equations.
\part Discuss the well-posedness of your system.
\part Rewrite the equations for $U'$ and $R'$
without using the variable $C$ by making the substitution $C=N-U-R$.
\part Find the equilibria for the two dimensional system given in part (d).
\part Suppose that when the rumor starts, it is initially known by a
very small fraction (much less than half) of the population and that
this group of people are all rumor spreaders.  Suppose that everyone
else in the population is in the uncool group.  
\begin{parts}
\part Initially, this should give $R'>0$ and $R$ should increase. Check that your equations in part (d) support this.
\part Then $U$ will decrease (from nearly $N$) until the rumor dies out.
Check that your equations in part (d) support this.
\part At what value of $U$ does $R$ stop increasing?
\part Eventually the rumor will die out when $R$ goes to zero.
When the rumor dies out, are there more people in the uncool
group or in the cool group?
\end{parts}
\part For bonus marks, study the stability of the equilibria of the system found in part (d).
\part Consider the state variables $u,r$ and $c$ representing the \emph{proportion} of $U,R$ and $C$ in the population. Write your system in terms of these variables.
\end{parts}



\end{questions}

\end{document}
