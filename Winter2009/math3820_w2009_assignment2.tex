\documentclass[12pt]{article}

\usepackage[active]{srcltx}

\usepackage{amsmath,amsfonts,amssymb,amsthm}
\usepackage{easybmat}
\usepackage{graphicx}
%\usepackage[hypertex]{hyperref}
%\usepackage[pdftex]{hyperref}
\usepackage{fancyhdr}
\usepackage{subfigure}

%\usepackage{makeidx}
%\makeindex

\theoremstyle{plain}
\newtheorem{theorem}{Theorem}
\newtheorem{property}[theorem]{Property}
\newtheorem{condition}[theorem]{Condition}
\newtheorem{proposition}[theorem]{Proposition}
\newtheorem{axiom}[theorem]{Axiom}
\newtheorem{lemma}[theorem]{Lemma}
\newtheorem{corollary}[theorem]{Corollary}
%%
% % Make numbering specific to the Appendix
% \newtheorem{Atheorem}{Theorem}[chapter]
% \newtheorem{Aproperty}[Atheorem]{Property}
% \newtheorem{Acondition}[Atheorem]{Condition}
% \newtheorem{Aproposition}[Atheorem]{Proposition}
% \newtheorem{Aaxiom}[Atheorem]{Axiom}
% \newtheorem{Alemma}[Atheorem]{Lemma}
% \newtheorem{Acorollary}[theorem]{Corollary}

% Make definitions non italicized
%\theoremstyle{definition}
\newtheorem{definition}[theorem]{Definition}
% Make numbering specific to the Appendix
%\newtheorem{Adefinition}[Atheorem]{Definition}
\newenvironment{defi}{\vskip0.2cm\addtocounter{theorem}{1}\par\noindent\bf Definition~\arabic{section}.\arabic{subsection}.\arabic{theorem}\rm.}{\hfill{$\circ$}\par\vskip0.25cm}


%% Example environment and counter.
\newcounter{cmpt_exercise}
\newenvironment{exercise}{\addtocounter{cmpt_exercise}{1}\vskip0.2cm\par\noindent\begin{small}\bf Exercise~\arabic{cmpt_exercise}\,\,\rm --}{\hfill{$\circ$}\end{small}\par\vskip0.25cm}
\newenvironment{problem}{\addtocounter{cmpt_exercise}{1}\vskip0.2cm\par\noindent{\Large\bf Problem~\arabic{cmpt_exercise}\,\,\rm --}}{\par\vskip0.25cm}


\newenvironment{example}{\vskip0.2cm\par\noindent\begin{small}\bf Example\,\,\rm --}{\hfill{$\diamond$}\end{small}\par\vskip0.25cm}
\newenvironment{remark}{\vskip0.2cm\par\noindent\begin{small}\bf Remark\,\,\rm --}{\hfill{$\circ$}\end{small}\par\vskip0.25cm}
%\newenvironment{aparte}[1]{\vskip0.2cm\par\noindent\begin{quote}\begin{small}\bf Apart\'e : #1\,\,\rm --}{\hfill{$\circ$}\end{small}\end{quote}\par\vskip0.25cm}
\newenvironment{aparte}[1]{\vskip0.3cm\par\begin{center}\begin{tabular}{|p{0.9\textwidth}|}\hline{\bf Apart\'e : #1}}{\\ \hline\end{tabular}\end{center}\par\vskip0.25cm}

\renewcommand{\labelenumi}{\roman{enumi})}
\renewcommand{\labelenumii}{\alph{enumii})}
\newcommand{\espv}{\vspace{.5\baselineskip}}
\def\IR{\mathbb{R}}
\def\IC{\mathbb{C}}
\def\IN{\mathbb{N}}
\def\IQ{\mathbb{Q}}
\def\IZ{\mathbb{Z}}
\def\rank{\textrm{rank }}
\def\Sp{\textrm{Sp }}
\def\Span{\textrm{Span }}
\def\Tr{\textrm{Tr }}
\def\D{\mathcal{D}}
\def\I{\mathcal{I}}
\def\U{\mathcal{U}}
\def\R{\mathcal{R}}
\def\Q{\mathcal{Q}}
\def\O{\mathcal{O}}
\def\Mn{\mathcal{M}_n}
\def\NN#1{\|#1\|}
\def\N3#1{|\!|\!|#1|\!|\!|}
\def\diag{\textrm{diag}}
\def\tr{\textrm{tr}}
\def\ker{\textrm{Ker }}

\def\M{\mathcal{M}}

\setlength{\textwidth}{17cm} 
\addtolength{\oddsidemargin}{-1.5cm}
\setlength{\textheight}{24cm}
\addtolength{\topmargin}{-2.5cm} 
\setlength{\headheight}{25.3pt}

%% Fancyhdr related stuff
\pagestyle{fancy}
\lhead{MATH 3820 -- Intro Math Modelling -- Assignment 2}
\rhead{\thepage}
\cfoot{}

\usepackage[hang,small,bf]{caption}
\setlength{\captionmargin}{20pt}

\makeatletter
\def\cleardoublepage{\clearpage\if@twoside \ifodd\c@page\else
\hbox{}
% \vspace*{\fill}
% \begin{center}
% This page intentionally contains only this sentence.
% \end{center}% Make numbering specific to the Appendix
\newtheorem{Atheorem}{Theorem}[chapter]
\newtheorem{Aproperty}[Atheorem]{Property}
\newtheorem{Acondition}[Atheorem]{Condition}
\newtheorem{Aproposition}[Atheorem]{Proposition}
\newtheorem{Aaxiom}[Atheorem]{Axiom}
\newtheorem{Alemma}[Atheorem]{Lemma}
\newtheorem{Acorollary}[theorem]{Corollary}

% \vspace{\fill}
\thispagestyle{empty}
\newpage
\if@twocolumn\hbox{}\newpage\fi\fi\fi}
\makeatother

%\author{Julien Arino}
%\address{University of Manitoba}
%\title{MATH 8430\\ Lecture Notes}
\title{University of Manitoba\\ Math 38200 -- Winter 2009}
\author{Assignment 2\\ Explanation of the model}
\date{Due Thursday, March 19 2009}

\renewcommand{\abstractname}{Instructions}
%%%%%%%%%%%%%%%%
%%%%%%%%%%%%%%%%
%%%%%%%%%%%%%%%%
%%%%%%%%%%%%%%%%
%%%%%%%%%%%%%%%%
%%%%%%%%%%%%%%%%
\begin{document}

\maketitle
\begin{abstract}
This assignment is due Thursday, March 19. Note that I will accept files, so if you want to typeset or scan your answers, this is fine (pdf is preferred).
\end{abstract}
Now that you have chosen a topic, you should try to understand and explain how the model was formulated. In this assignment, you should do this. The following steps are required. In the following, I list these points, and show some sample questions that you may want to address in each point. This assignment should count a minimum of 4 or 5 pages (cover page not included).
\begin{enumerate}
\item Position of the problem. Explain, in at most half a page, the problem that the model is formulated for, in general terms. What can a mathematical model bring to the problem?
\item Description of the physical/biological system being modelled. Here, you must go into a much more precise description of the ``real life system'' that your model will represent. Diagrams, if any, are a great help. At this point, there still should not be any mathematics.
\item Model hypotheses. The previous point leads naturally into this one. This is where you start explaining how ``real life'' is going to be reduced to a set of mathematical equations. The first part of this reduction is to state what particular assumptions you are going to make. For example: in a chemostat model, we consider a population of organisms and the concentration of substrate. The growth of organisms is assumed to take the form ..
\item Derivation of the model. How do you go from a certain number of hypotheses to a set of mathematical equations? Note that if your model is derived from another one by simply adding some hypotheses, then you should also explain how the first model is derived. Example: suppose you are considering an SEIRS epidemic model with the infection parameter $\beta$ dependent on time, i.e., $\beta(t)$. Then you should explain how an SEIRS model is obtained, and then discuss how $\beta$ becomes $\beta(t)$; it would not be acceptable to simply state that the SEIRS model with time dependent $\beta$ derives from the ``well known SEIRS model'' by assuming time dependance of $\beta$.
\item Synthesis. An often neglected step, yet a very important one. State the model, the initial conditions (if any). If you made some hypotheses about some functions involved, refer (if possible by equation number) to the definition of these functions, etc. This is what a modeller reading your report will probably look for first: a quickly read summary of your model, in mathematical form.
\end{enumerate}



\end{document}