\documentclass[12pt]{article}

\usepackage[active]{srcltx}

\usepackage{amsmath,amsfonts,amssymb,amsthm}
\usepackage{easybmat}
\usepackage{graphicx}
%\usepackage[hypertex]{hyperref}
%\usepackage[pdftex]{hyperref}
\usepackage{fancyhdr}
\usepackage{subfigure}

\usepackage{makeidx}
\makeindex

\theoremstyle{plain}
\newtheorem{theorem}{Theorem}
\newtheorem{property}[theorem]{Property}
\newtheorem{condition}[theorem]{Condition}
\newtheorem{proposition}[theorem]{Proposition}
\newtheorem{axiom}[theorem]{Axiom}
\newtheorem{lemma}[theorem]{Lemma}
\newtheorem{corollary}[theorem]{Corollary}
%%
% % Make numbering specific to the Appendix
% \newtheorem{Atheorem}{Theorem}[chapter]
% \newtheorem{Aproperty}[Atheorem]{Property}
% \newtheorem{Acondition}[Atheorem]{Condition}
% \newtheorem{Aproposition}[Atheorem]{Proposition}
% \newtheorem{Aaxiom}[Atheorem]{Axiom}
% \newtheorem{Alemma}[Atheorem]{Lemma}
% \newtheorem{Acorollary}[theorem]{Corollary}

% Make definitions non italicized
%\theoremstyle{definition}
\newtheorem{definition}[theorem]{Definition}
% Make numbering specific to the Appendix
%\newtheorem{Adefinition}[Atheorem]{Definition}
\newenvironment{defi}{\vskip0.2cm\addtocounter{theorem}{1}\par\noindent\bf Definition~\arabic{section}.\arabic{subsection}.\arabic{theorem}\rm.}{\hfill{$\circ$}\par\vskip0.25cm}


%% Example environment and counter.
\newcounter{cmpt_exercise}
\newenvironment{exercise}{\addtocounter{cmpt_exercise}{1}\vskip0.2cm\par\noindent\begin{small}\bf Exercise~\arabic{cmpt_exercise}\,\,\rm --}{\hfill{$\circ$}\end{small}\par\vskip0.25cm}
\newenvironment{problem}{\addtocounter{cmpt_exercise}{1}\vskip0.2cm\par\noindent{\Large\bf Problem~\arabic{cmpt_exercise}\,\,\rm --}}{\par\vskip0.25cm}


\newenvironment{example}{\vskip0.2cm\par\noindent\begin{small}\bf Example\,\,\rm --}{\hfill{$\diamond$}\end{small}\par\vskip0.25cm}
\newenvironment{remark}{\vskip0.2cm\par\noindent\begin{small}\bf Remark\,\,\rm --}{\hfill{$\circ$}\end{small}\par\vskip0.25cm}
%\newenvironment{aparte}[1]{\vskip0.2cm\par\noindent\begin{quote}\begin{small}\bf Apart\'e : #1\,\,\rm --}{\hfill{$\circ$}\end{small}\end{quote}\par\vskip0.25cm}
\newenvironment{aparte}[1]{\vskip0.3cm\par\begin{center}\begin{tabular}{|p{0.9\textwidth}|}\hline{\bf Apart\'e : #1}}{\\ \hline\end{tabular}\end{center}\par\vskip0.25cm}

\renewcommand{\labelenumi}{\roman{enumi})}
\renewcommand{\labelenumii}{\alph{enumii})}
\newcommand{\espv}{\vspace{.5\baselineskip}}
\def\IR{\mathbb{R}}
\def\IC{\mathbb{C}}
\def\IN{\mathbb{N}}
\def\IQ{\mathbb{Q}}
\def\IZ{\mathbb{Z}}
\def\rank{\textrm{rank }}
\def\Sp{\textrm{Sp }}
\def\Span{\textrm{Span }}
\def\Tr{\textrm{Tr }}
\def\D{\mathcal{D}}
\def\I{\mathcal{I}}
\def\U{\mathcal{U}}
\def\R{\mathcal{R}}
\def\Q{\mathcal{Q}}
\def\O{\mathcal{O}}
\def\Mn{\mathcal{M}_n}
\def\NN#1{\|#1\|}
\def\N3#1{|\!|\!|#1|\!|\!|}
\def\diag{\textrm{diag}}
\def\tr{\textrm{tr}}
\def\ker{\textrm{Ker }}

\def\M{\mathcal{M}}

\setlength{\textwidth}{17cm} 
\addtolength{\oddsidemargin}{-1.5cm}
\setlength{\textheight}{22cm}
\addtolength{\topmargin}{-2cm} 
\setlength{\headheight}{25.3pt}

%% Fancyhdr related stuff
\pagestyle{fancy}
\lhead{MATH 3820 -- Intro Math Modelling -- Assignment 2}
\rhead{\thepage}
\cfoot{}

\usepackage[hang,small,bf]{caption2}
\setlength{\captionmargin}{20pt}

\makeatletter
\def\cleardoublepage{\clearpage\if@twoside \ifodd\c@page\else
\hbox{}
% \vspace*{\fill}
% \begin{center}
% This page intentionally contains only this sentence.
% \end{center}% Make numbering specific to the Appendix
\newtheorem{Atheorem}{Theorem}[chapter]
\newtheorem{Aproperty}[Atheorem]{Property}
\newtheorem{Acondition}[Atheorem]{Condition}
\newtheorem{Aproposition}[Atheorem]{Proposition}
\newtheorem{Aaxiom}[Atheorem]{Axiom}
\newtheorem{Alemma}[Atheorem]{Lemma}
\newtheorem{Acorollary}[theorem]{Corollary}

% \vspace{\fill}
\thispagestyle{empty}
\newpage
\if@twocolumn\hbox{}\newpage\fi\fi\fi}
\makeatother

%\author{Julien Arino}
%\address{University of Manitoba}
%\title{MATH 8430\\ Lecture Notes}
\title{University of Manitoba\\ Math 38200 -- Winter 2007}
\author{Assignment 2}
\date{Due Thursday, March 29, 2007}

\renewcommand{\abstractname}{Instructions}
%%%%%%%%%%%%%%%%
%%%%%%%%%%%%%%%%
%%%%%%%%%%%%%%%%
%%%%%%%%%%%%%%%%
%%%%%%%%%%%%%%%%
%%%%%%%%%%%%%%%%
\begin{document}

\maketitle
\begin{abstract}
This assignment is due Thursday, March 29. Note that I will accept files, so if you want to typeset or scan your answers, this is fine (pdf is preferred). Since I will be away on March 29, if you are turning in your assignment on paper and not electronically, please take them to the Mathematics Department office (342 Machray Hall).

The aim of this assignment is to make sure that you have chosen your project topic and have started working on it. It will be marked as a regular assignment, and might also generate feedback from me, if I find that the topic definition needs more work.
\end{abstract}

\vskip1cm
Remember a few facts about the project:
\begin{itemize}
\item It is due on April 23, same day as the final for our class.
\item You will have other finals to think about at that time, ours among others.
\item If you get started now, and do a little regularly, it will be much easier than leaving a big load of work until the last moment.
\end{itemize}
Therefore, it is important that you get your project started and organize your ideas. This is the aim of this assignment.

\vskip1cm
The text of this assignment is thus extremely simple:
\begin{center}
{\bf Present your project in 2 or 3 pages (no more than 3 pages).}
\end{center}
The following are (in random order) questions that I should be able to answer after reading your assignment: 
\begin{itemize}
\item What is the topic?
\item What is interesting about this topic?
\item What type of modelling framework will be used (discrete-time, ODE, DDE, PDE, other..)?
\item What are one or two well known equations in that domain?
\item What mathematical techniques can be used for this project?
\item What numerical techniques can be used for this project?
\end{itemize}
and any other things that you may think of.. 

\vskip1cm
I will pay attention to form: I am not in your mind, so equations or facts thrown at random mean nothing to me (even if I happen to known them). Note that you {\bf should not use more than 3 pages}. Organize things to fit everything in these 3 pages.

\vskip1cm
Lastly, this document is not \emph{binding}: if, when working on your project later, you realize that some of the things you had said in this assignment turn out to be too easy or too hard or not interesting or .., and you end up doing something different from what you had said here, that will have no impact on the marking of your project.


\end{document}