\documentclass[12pt]{article}

\usepackage[active]{srcltx}

\usepackage{amsmath,amsfonts,amssymb,amsthm}
\usepackage{easybmat}
\usepackage{graphicx}
%\usepackage[hypertex]{hyperref}
%\usepackage[pdftex]{hyperref}
\usepackage{fancyhdr}
\usepackage{subfigure}

\usepackage{makeidx}
\makeindex

\theoremstyle{plain}
\newtheorem{theorem}{Theorem}
\newtheorem{property}[theorem]{Property}
\newtheorem{condition}[theorem]{Condition}
\newtheorem{proposition}[theorem]{Proposition}
\newtheorem{axiom}[theorem]{Axiom}
\newtheorem{lemma}[theorem]{Lemma}
\newtheorem{corollary}[theorem]{Corollary}
%%
% % Make numbering specific to the Appendix
% \newtheorem{Atheorem}{Theorem}[chapter]
% \newtheorem{Aproperty}[Atheorem]{Property}
% \newtheorem{Acondition}[Atheorem]{Condition}
% \newtheorem{Aproposition}[Atheorem]{Proposition}
% \newtheorem{Aaxiom}[Atheorem]{Axiom}
% \newtheorem{Alemma}[Atheorem]{Lemma}
% \newtheorem{Acorollary}[theorem]{Corollary}

% Make definitions non italicized
%\theoremstyle{definition}
\newtheorem{definition}[theorem]{Definition}
% Make numbering specific to the Appendix
%\newtheorem{Adefinition}[Atheorem]{Definition}
\newenvironment{defi}{\vskip0.2cm\addtocounter{theorem}{1}\par\noindent\bf Definition~\arabic{section}.\arabic{subsection}.\arabic{theorem}\rm.}{\hfill{$\circ$}\par\vskip0.25cm}


%% Example environment and counter.
\newcounter{cmpt_exercise}
\newenvironment{exercise}{\addtocounter{cmpt_exercise}{1}\vskip0.2cm\par\noindent\begin{small}\bf Exercise~\arabic{cmpt_exercise}\,\,\rm --}{\hfill{$\circ$}\end{small}\par\vskip0.25cm}
\newenvironment{problem}{\addtocounter{cmpt_exercise}{1}\vskip0.2cm\par\noindent{\Large\bf Problem~\arabic{cmpt_exercise}\,\,\rm --}}{\par\vskip0.25cm}


\newenvironment{example}{\vskip0.2cm\par\noindent\begin{small}\bf Example\,\,\rm --}{\hfill{$\diamond$}\end{small}\par\vskip0.25cm}
\newenvironment{remark}{\vskip0.2cm\par\noindent\begin{small}\bf Remark\,\,\rm --}{\hfill{$\circ$}\end{small}\par\vskip0.25cm}
%\newenvironment{aparte}[1]{\vskip0.2cm\par\noindent\begin{quote}\begin{small}\bf Apart\'e : #1\,\,\rm --}{\hfill{$\circ$}\end{small}\end{quote}\par\vskip0.25cm}
\newenvironment{aparte}[1]{\vskip0.3cm\par\begin{center}\begin{tabular}{|p{0.9\textwidth}|}\hline{\bf Apart\'e : #1}}{\\ \hline\end{tabular}\end{center}\par\vskip0.25cm}

\renewcommand{\labelenumi}{\roman{enumi})}
\renewcommand{\labelenumii}{\alph{enumii})}
\newcommand{\espv}{\vspace{.5\baselineskip}}
\def\IR{\mathbb{R}}
\def\IC{\mathbb{C}}
\def\IN{\mathbb{N}}
\def\IQ{\mathbb{Q}}
\def\IZ{\mathbb{Z}}
\def\rank{\textrm{rank }}
\def\Sp{\textrm{Sp }}
\def\Span{\textrm{Span }}
\def\Tr{\textrm{Tr }}
\def\D{\mathcal{D}}
\def\I{\mathcal{I}}
\def\U{\mathcal{U}}
\def\R{\mathcal{R}}
\def\Q{\mathcal{Q}}
\def\O{\mathcal{O}}
\def\Mn{\mathcal{M}_n}
\def\NN#1{\|#1\|}
\def\N3#1{|\!|\!|#1|\!|\!|}
\def\diag{\textrm{diag}}
\def\tr{\textrm{tr}}
\def\ker{\textrm{Ker }}

\def\M{\mathcal{M}}

\setlength{\textwidth}{17cm} 
\addtolength{\oddsidemargin}{-1.5cm}
\setlength{\textheight}{22cm}
\addtolength{\topmargin}{-2cm} 
\setlength{\headheight}{25.3pt}

%% Fancyhdr related stuff
\pagestyle{fancy}
\lhead{MATH 3820 -- Intro Math Modelling -- Assignment 1}
\rhead{\thepage}
\cfoot{}

\usepackage[hang,small,bf]{caption2}
\setlength{\captionmargin}{20pt}

\makeatletter
\def\cleardoublepage{\clearpage\if@twoside \ifodd\c@page\else
\hbox{}
% \vspace*{\fill}
% \begin{center}
% This page intentionally contains only this sentence.
% \end{center}% Make numbering specific to the Appendix
\newtheorem{Atheorem}{Theorem}[chapter]
\newtheorem{Aproperty}[Atheorem]{Property}
\newtheorem{Acondition}[Atheorem]{Condition}
\newtheorem{Aproposition}[Atheorem]{Proposition}
\newtheorem{Aaxiom}[Atheorem]{Axiom}
\newtheorem{Alemma}[Atheorem]{Lemma}
\newtheorem{Acorollary}[theorem]{Corollary}

% \vspace{\fill}
\thispagestyle{empty}
\newpage
\if@twocolumn\hbox{}\newpage\fi\fi\fi}
\makeatother

%\author{Julien Arino}
%\address{University of Manitoba}
%\title{MATH 8430\\ Lecture Notes}
\title{University of Manitoba\\ Math 38200 -- Winter 2007}
\author{Assignment 1}
\date{Due Tuesday, February 20, 2007}

\renewcommand{\abstractname}{Instructions}
%%%%%%%%%%%%%%%%
%%%%%%%%%%%%%%%%
%%%%%%%%%%%%%%%%
%%%%%%%%%%%%%%%%
%%%%%%%%%%%%%%%%
%%%%%%%%%%%%%%%%
\begin{document}

\maketitle
\begin{abstract}
This assignment is due Tuesday, February 20. Note that I will accept files, so if you want to typeset or scan your answers, this is fine (pdf is preferred).

Try to be clear in your answers. If you use a result in the course notes or from another source, please indicate clearly the reference. 
\end{abstract}

\vskip1cm
\noindent
The Ricker model of growth of a single population takes the form
\begin{equation}\label{eq:Ricker}
N_{t+1}=N_t e^{r\left(1-\frac{N_t}K\right)},
\end{equation}
with $r,K>0$ and initial condition $N_0>0$. The aim of this assignment is to study the behavior of \eqref{eq:Ricker}.

\vskip1cm
\noindent
{\bf 1.} Define $x_t=N_t/K$, and show that the difference equation \eqref{eq:Ricker} then takes the form, as a function of the dimensionless variable $x_t$,
\begin{equation}\label{eq:Ricker_dimless}
x_{t+1}=x_t e^{r(1-x_t)},
\end{equation}

\vskip0.4cm
\noindent
{\bf 2.} Show that if $x_0>0$ (where $x_0=N_0/K$), there holds that $x_t>0$ for all $t$ in \eqref{eq:Ricker_dimless}.

\vskip0.4cm
\noindent
{\bf 3.} 
Determine the fixed points of \eqref{eq:Ricker_dimless}, as well as their stability as a function of the parameter $r$.

\vskip0.4cm
\noindent
{\bf 4.} 
Show that \eqref{eq:Ricker_dimless} has no points of period 2 for $0<r<2$. [Hint: use Theorem~\ref{th:nonexistFP2} with $I=(0,\infty)$.]

\vskip0.4cm
\noindent
{\bf 5.} 
Try to find 2-periodic points of \eqref{eq:Ricker_dimless} analytically (show your work). Then, do so using a numerical software. Under what conditions are these periodic points stable? Evaluate the stability of the points you found for a few sample values of $r$, using a numerical software.

\vskip0.4cm
\noindent
{\bf 6.} 
Using numerical software, draw a bifurcation diagram for \eqref{eq:Ricker_dimless}, for $r$ varying in $(0,5]$. What do you observe?


\begin{center}
\bf A useful result
\end{center}
\begin{theorem}\label{th:nonexistFP2}
Consider the function $f:I\to I$, and assume that $f'$ is continuous on $I$. If $1+f'(x)\neq 0$ for all $x\in I$, then $x_{t+1}=f(x_t)$ has no periodic point of period 2 in $I$.
\end{theorem}

\end{document}