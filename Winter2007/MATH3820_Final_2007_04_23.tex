\documentclass[12pt]{exam}



%\usepackage{graphicx}% Include figure files
%\usepackage{dcolumn}% Align table columns on decimal point
%\usepackage{bm}% bold math
\usepackage{latexsym}
\usepackage{amsmath,amssymb,amsthm,amsfonts}
%\usepackage{subfigure}

%\usepackage{titlefoot}

%\usepackage[displaymath]{lineno}

%

%\usepackage{fancyhdr}
%
%\setlength{\textwidth}{6.9in} \setlength{\textheight}{9.4in}
%\addtolength{\oddsidemargin}{-3cm} \addtolength{\topmargin}{-0.2in}
%\setlength{\headheight}{5pt} \setlength{\headsep}{10pt}

\def\IR{\mathbb{R}}

\begin{document}
\pagestyle{head} \extraheadheight{1in}\runningheadrule
\firstpageheader{$\mbox{}$\\DATE: April 23, 2007\\PAPER NO.: 641 \\
DEPARTMENT $\&$ COURSE NO.: MATH 3820 \\EXAMINATION: Intro. Math. Model.}{\bf
UNIVERSITY OF MANITOBA\\
$\mbox{}$
\\$\mbox{}$ \\ $\mbox{}$\\$\mbox{}$}{$\mbox{}$\\FINAL EXAMINATION\\ PAGE NO.:
\thepage\ of \numpages\\TIME: 3
hours\\ EXAMINER: J. Arino}\runningheadrule
\runningheader{$\mbox{}$\\DATE: April 23, 2007\\PAPER NO.: 641 \\
DEPARTMENT $\&$ COURSE NO.: MATH 3820 \\EXAMINATION: Intro. Math. Model.}{\bf
UNIVERSITY OF MANITOBA\\
$\mbox{}$
\\$\mbox{}$ \\ $\mbox{}$\\$\mbox{}$}{$\mbox{}$\\FINAL EXAMINATION\\ PAGE NO.:
\thepage\ of \numpages\\TIME: 3
hours\\ EXAMINER: J. Arino} \vspace{-2.5cm}
\begin{center}
\begin{tabular}{cp{15.5cm}}
\hline
&\\
\end{tabular}
\end{center}
%\footer{}{Page \thepage\ of \numpages}%
%{\iflastpage{End of exam.}{Please go on to the next page\ldots}}

\addpoints
%\bibliographystyle{plain}
%\pagestyle{fancy} \chead{} \lhead{} \rhead{} \cfoot{}




%\vspace{.2cm}
%
%{\bf NAME:} \vspace{0.5cm}
%
%{\bf STUDENT \#:}


\vspace{0.1in} \hbox to \textwidth{NAME:\enspace\hrulefill}
\vspace{0.2in} \hbox to \textwidth{STUDENT NUMBER:\enspace\hrulefill
 SEAT NUMBER:\enspace\hrulefill} \vspace{0.2in} \hbox to
\textwidth{SIGNATURE:\enspace\hrulefill}
\hspace{1.75in}(I understand that cheating is a serious offense)
\vspace{.5in}

This is a 3 hours exam. Cell phones are
not allowed. Calculators and notes are allowed; books are {\bf not} allowed.

\vspace{.1in} This exam has a title page and
2 pages of questions.

\vspace{.2in} {\bf PLEASE SHOW YOUR WORK CLEARLY.}
A correct but unclear answer will not get full marks, nor will a correct answer
that does not give some detail of the method used to obtain the solution.


\vspace{.5in}
\begin{center}
\gradetable[v]
\end{center}



\begin{questions}
\newpage
\question[20] We want to model the survival of a population of whales. We assume
that if the number of whales falls below a minimum survival level $m$, then the
species becomes extinct. In addition, we assume that the population is limited
by the carrying capacity $M$ of the environment, that is, if the population is
above $M$, it will experience a decline because the environment cannot sustain
that large a population level.
\begin{parts}
\part Let $a_t$ represent the whale population after $t$ years. Discuss the
model
\[
a_{t+1}=a_t+k(M-a_t)(a_t-m),
\]
where $k>0$. Does it make sense in terms of the description above?
\part Find the fixed points of the model, and determine their stability.
\part Assume that $M=5000$, $m=100$ and $k=0.0001$. Perform a graphical
stability analysis.
\part The model has two serious shortcomings. What are they? [Hint: Consider
what happens when $a_0<m$, and when $a_0\gg M$.]
\end{parts}

\vskip1cm
\question[10]
We consider the partial differential equation
\begin{equation}\label{eq:diffusion}
u_t=Du_{xx},
\end{equation}
with $D>0$.
\begin{parts}
\part Show that the function
\[
g(x,t)=\frac{1}{2\sqrt{\pi Dt}}e^{-\frac{x^2}{4Dt}}
\]
is a solution to the diffusion equation \eqref{eq:diffusion}.
\part Verify that $g(x,t)\geq 0$ for all $t\geq 0$ and $x\in\IR$. Investigate
the limits as $x\to\pm\infty$ and $t\to\infty$.
\end{parts}

\vskip1cm
\question[15]
A forest ecosystem is observed for several years. When a tree dies, the type of
tree that replaces it is recorded. The following table is obtained:
\begin{center}
\begin{tabular}{c||c|c|c|c|c|}
& RO & HI & TU & RM & BE \\
\hline
Red oak (RO) & 0.12 & 0.12 & 0.12 & 0.42 & 0.22 \\
Hickory (HI) & 0.14 & 0.05 & 0.10 & 0.53 & 0.18 \\
Tulip tree (TU) & 0.12 & 0.08 & 0.10 & 0.32 & 0.38 \\
Red maple (RM) & 0.12 & 0.28 & 0.05 & 0.20 & 0.35 \\
Beech (BE) & 0.13 & 0.27 & 0.08 & 0.19 & 0.33
\end{tabular}
\end{center}
\begin{parts}
\part Is this a Markov chain?
\part Assume that we start with a forest comprising only red oaks. What are the
first 2 iterates of the process.
\part Is this Markov chain regular?
\part How would you obtain the equilibrium value of the distribution of trees?
Do not do the computations.
\end{parts}

\vskip1cm
\question[20]
We consider models of a fishery. In the absence of fishing, the fish population
grows logistically,
\[
N'=rN\left(1-\frac NK\right),
\]
with $r,K>0$. We add terms to this equation to represent the
harvesting of the population by fishing. This gives us the following three
models:
\begin{equation}\label{eq:fish1}
N'=rN\left(1-\frac NK\right)-H_1,
\end{equation}
\begin{equation}\label{eq:fish2}
N'=rN\left(1-\frac NK\right)-H_2N,
\end{equation}
and
\begin{equation}\label{eq:fish3}
N'=rN\left(1-\frac NK\right)-H_3\frac{N}{A+N},
\end{equation}
where $H_1,H_2,H_3,A>0$.
\begin{parts}
\part For each model, give an interpretation of the fishing term. How do these
terms differ? How do you interpret the parameters $H_1,H_2,H_3,A$?
\part Why is \eqref{eq:fish1} not realistic?
\part Which of \eqref{eq:fish2} and \eqref{eq:fish3} do you think is best?
\end{parts}
[Hint: In this exercise, although it is not asked explicitly, you may want to
study the models \eqref{eq:fish1}, \eqref{eq:fish2} and \eqref{eq:fish3} more in
detail (fixed points, stability, conditions leading to the extinction of the
population, etc.).]




\end{questions}
\end{document}
