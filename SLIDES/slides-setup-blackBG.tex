\usetheme{default}
\input{slides_setup_colour_independent.tex}

%% Listings
\usepackage{listings}
\definecolor{mygreen}{rgb}{0,0.6,0}
\definecolor{mygray}{rgb}{0.5,0.5,0.5}
\definecolor{mymauve}{rgb}{0.58,0,0.82}
\definecolor{mygold}{rgb}{1,0.843,0}
\definecolor{myblue}{rgb}{0.537,0.812,0.941}

\definecolor{lgreen}{rgb}{0.6,0.9,.6}
\definecolor{lred}{rgb}{1,0.5,.5}

\lstloadlanguages{R}
\lstset{ %
  language=R,
  backgroundcolor=\color{black!95},   % choose the background color
  basicstyle=\footnotesize\ttfamily,        % size of fonts used for the code
  breaklines=true,                 % automatic line breaking only at whitespace
  captionpos=b,                    % sets the caption-position to bottom
  commentstyle=\color{mygreen},    % comment style
  escapeinside={\%*}{*)},          % if you want to add LaTeX within your code
  keywordstyle=\color{myblue},       % keyword style
  stringstyle=\color{mygold},     % string literal style
  keepspaces=true,
  columns=fullflexible,
  tabsize=4,
}
% Could also do (in lstset)
% basicstyle==\fontfamily{pcr}\footnotesize


%%%%%%% 
%% Definitions in yellow boxes
\usepackage{etoolbox}
\setbeamercolor{block title}{use=structure,fg=structure.fg,bg=structure.fg!05!bg}
\setbeamercolor{block body}{parent=normal text,use=block title,bg=block title.bg!20!bg}

\BeforeBeginEnvironment{definition}{%
	\setbeamercolor{block title}{fg=black,bg=yellow!20!white}
	\setbeamercolor{block body}{fg=black, bg=yellow!05!white}
}
\AfterEndEnvironment{definition}{
	\setbeamercolor{block title}{use=structure,fg=structure.fg,bg=structure.fg!20!bg}
	\setbeamercolor{block body}{parent=normal text,use=block title,bg=block title.bg!50!bg, fg=black}
}
\BeforeBeginEnvironment{importanttheorem}{%
	\setbeamercolor{block title}{fg=black,bg=red!20!white}
	\setbeamercolor{block body}{fg=black, bg=red!05!white}
}
\AfterEndEnvironment{importanttheorem}{
	\setbeamercolor{block title}{use=structure,fg=structure.fg,bg=structure.fg!20!bg}
	\setbeamercolor{block body}{parent=normal text,use=block title,bg=block title.bg!50!bg, fg=black}
}
\BeforeBeginEnvironment{theorem}{%
	\setbeamercolor{block title}{fg=white,bg=red!30!black}
	\setbeamercolor{block body}{fg=white, bg=red!10!black}
}
\AfterEndEnvironment{theorem}{
	\setbeamercolor{block title}{use=structure,fg=structure.fg,bg=structure.fg!20!bg}
	\setbeamercolor{block body}{parent=normal text,use=block title,bg=block title.bg!50!bg, fg=black}
}
\BeforeBeginEnvironment{importantproperty}{%
	\setbeamercolor{block title}{fg=black,bg=red!50!white}
	\setbeamercolor{block body}{fg=black, bg=red!30!white}
}
\AfterEndEnvironment{importantproperty}{
	\setbeamercolor{block title}{use=structure,fg=structure.fg,bg=structure.fg!20!bg}
	\setbeamercolor{block body}{parent=normal text,use=block title,bg=block title.bg!50!bg, fg=black}
}


\tikzstyle{cloud} = [draw, 
ellipse,
fill=red!20, 
node distance=0.87cm,
minimum height=2em]
\tikzstyle{line} = [draw, 
-latex', 
color=yellow]


% Beginning of a section
% \AtBeginSection[]{
% 	{
% 		\setbeamercolor{background canvas}{bg=orange!10}
% 		\begin{frame}[noframenumbering,plain]
% 			\framesubtitle{\nameofthepart Chapter \insertromanpartnumber \ -- \iteminsert{\insertpart}}
% 			\tableofcontents[currentsection,currentsubsection]
% 		\end{frame}
% 	\addtocounter{page}{-1}
% 	%\addtocounter{framenumber}{-1} 
% 	}
% }


%%% SLIDES COLOURING

%\usecolortheme{owl}

\setbeamerfont{frametitle}{series=\bfseries}
\setbeamercolor{frametitle}{fg=black!05,bg=black}

\setbeamerfont{framesubtitle}{size=\normalfont\tiny}
\setbeamercolor{framesubtitle}{fg=black!05}

\setbeamercolor{background canvas}{bg=black}
\setbeamercolor{normal text}{fg=black!10}



\definecolor{bottomcolour}{rgb}{0.32,0.3,0.38}
\definecolor{middlecolour}{rgb}{0.08,0.08,0.16}
\definecolor{mycolor}{rgb}{0.4,0.4, 0.4}
% Beginning of a section
\AtBeginSection[]{
	{
		%\setbeamercolor{background canvas}[vertical shading][top=bottomcolour, middle=middlecolour, bottom=black]
		\setbeamertemplate{background canvas}[vertical shading][bottom=bottomcolour,top=black!20]
    % \setbeamertemplate{background canvas}{
    %   \begin{tikzpicture}%[remember picture,overlay]
    %     \shade[top color=yellow!75!green!33,
    %     bottom color=blue!66!green!33,
    %     middle color=blue!6!green!33]
    %   \end{tikzpicture}
    % }
    \begin{frame}[noframenumbering,plain]
			\framesubtitle{\nameofthepart Chapter \insertromanpartnumber \ -- \iteminsert{\insertpart}}
			\tableofcontents[currentsection,currentsubsection]
		\end{frame}
	\addtocounter{page}{-1}
	}
}
% Beginning of a section
\AtBeginSubsection[]{
	{
		%\setbeamercolor{background canvas}[vertical shading][top=bottomcolour, middle=middlecolour, bottom=black]
		%\setbeamertemplate{background canvas}[vertical shading][bottom=bottomcolour,top=black!20]
    \setbeamertemplate{background canvas}{
      \begin{tikzpicture}%[remember picture,overlay]
        \shade[top color=yellow!75!green!33,
        bottom color=blue!66!green!33,
        middle color=blue!6!green!33]
      \end{tikzpicture}
    }
    \begin{frame}[noframenumbering,plain]
			\framesubtitle{\nameofthepart Chapter \insertromanpartnumber \ -- \iteminsert{\insertpart}}
			\tableofcontents[currentsection,currentsubsection]
		\end{frame}
	\addtocounter{page}{-1}
	}
}

% Colours for special pages
\def\extraContent{yellow!20}

%% Allow to change slide colour
%% From: https://tex.stackexchange.com/questions/8043/change-the-background-color-of-a-frame-in-beamer
\defbeamertemplate*{background canvas}{mydefault}{%
  \ifbeamercolorempty[bg]{background canvas}{}{\color{bg}\vrule width\paperwidth height\paperheight}% copied beamer default here
}
\defbeamertemplate*{background canvas}{bg}{%
  \color{lightgray!20}\vrule width\paperwidth height\paperheight% added bg color
}
\BeforeBeginEnvironment{frame}{%
  \setbeamertemplate{background canvas}[mydefault]%
}
\makeatletter
\define@key{beamerframe}{bg}[true]{%
  \setbeamertemplate{background canvas}[bg]%
}
\makeatother
% Use with
%\begin{frame}
% \frametitle{Normal}
%\end{frame} 
%\begin{frame}[bg]
% \frametitle{With bg}
%\end{frame}


%% Vertical alignment on pages
%% From: https://tex.stackexchange.com/questions/148365/how-do-i-ask-beamer-to-exactly-fill-up-a-slide
%% Turn on with
%% \stretchon
%% (outside slide), and off with
%% \stretchoff
% \def\itemsymbol{$\blacktriangleright$}
% \let\svpar\par
% \let\svitemize\itemize
% \let\svenditemize\enditemize
% \let\svitem\item
% \let\svcenter\center
% \let\svendcenter\endcenter
% \let\svcolumn\column
% \let\svendcolumn\endcolumn
% \def\newitem{\renewcommand\item[1][\itemsymbol]{\vfill\svitem[##1]}}%
% \def\newpar{\def\par{\svpar\vfill}}%
% \newcommand\stretchon{%
%   \newpar%
%   \renewcommand\item[1][\itemsymbol]{\svitem[##1]\newitem}%
%   \renewenvironment{itemize}%
%     {\svitemize}{\svenditemize\newpar\par}%
%   \renewenvironment{center}%
%     {\svcenter\newpar}{\svendcenter\newpar}%
%   \renewenvironment{column}[2]%
%     {\svcolumn{##1}\setlength{\parskip}{\columnskip}##2}%
%     {\svendcolumn\vspace{\columnskip}}%
% }
% \newcommand\stretchoff{%
%   \let\par\svpar%
%   \let\item\svitem%
%   \let\itemize\svitemize%
%   \let\enditemize\svenditemize%
%   \let\center\svcenter%
%   \let\endcenter\svendcenter%
%   \let\column\svcolumn%
%   \let\endcolumn\svendcolumn%
% }
% \newlength\columnskip
% \columnskip 0pt
