\documentclass[12pt]{exam}

\input xy
\xyoption{all}


%\usepackage{graphicx}% Include figure files
%\usepackage{dcolumn}% Align table columns on decimal point
%\usepackage{bm}% bold math
\usepackage{latexsym}
\usepackage{amsmath,amssymb,amsthm,amsfonts}
%\usepackage{subfigure}

%\usepackage{titlefoot}

%\usepackage[displaymath]{lineno}

%

%\usepackage{fancyhdr}
%
%\setlength{\textwidth}{6.9in} \setlength{\textheight}{9.4in}
%\addtolength{\oddsidemargin}{-3cm} \addtolength{\topmargin}{-0.2in}
%\setlength{\headheight}{5pt} \setlength{\headsep}{10pt}

\def\IR{\mathbb{R}}
\def\dxdt{\frac{dx}{dt}}
\def\dydt{\frac{dy}{dt}}

\begin{document}
\pagestyle{head} \extraheadheight{1in}\runningheadrule
\firstpageheader{$\mbox{}$\\DATE: April 16, 2008\\PAPER NO.: 249 \\
DEPARTMENT $\&$ COURSE NO.: MATH 3820 \\EXAMINATION: Intro. Math. Model.}{\bf
UNIVERSITY OF MANITOBA\\
$\mbox{}$
\\$\mbox{}$ \\ $\mbox{}$\\$\mbox{}$}{$\mbox{}$\\FINAL EXAMINATION\\ PAGE NO.:
\thepage\ of \numpages\\TIME: 3
hours\\ EXAMINER: J. Arino}\runningheadrule
\runningheader{$\mbox{}$\\DATE: April 16, 2008\\PAPER NO.: 249 \\
DEPARTMENT $\&$ COURSE NO.: MATH 3820 \\EXAMINATION: Intro. Math. Model.}{\bf
UNIVERSITY OF MANITOBA\\
$\mbox{}$
\\$\mbox{}$ \\ $\mbox{}$\\$\mbox{}$}{$\mbox{}$\\FINAL EXAMINATION\\ PAGE NO.:
\thepage\ of \numpages\\TIME: 3
hours\\ EXAMINER: J. Arino} \vspace{-2.5cm}
\begin{center}
\begin{tabular}{cp{15.5cm}}
\hline
&\\
\end{tabular}
\end{center}
%\footer{}{Page \thepage\ of \numpages}%
%{\iflastpage{End of exam.}{Please go on to the next page\ldots}}

\addpoints
%\bibliographystyle{plain}
%\pagestyle{fancy} \chead{} \lhead{} \rhead{} \cfoot{}




%\vspace{.2cm}
%
%{\bf NAME:} \vspace{0.5cm}
%
%{\bf STUDENT \#:}


This is a 3 hours exam. Cell phones are
not allowed. Calculators and notes are allowed; books are {\bf not} allowed.

\vspace{.1in} This exam has \numquestions\;questions for a total of \numpoints\;marks.

\vspace{.2in} \textbf{ANSWER IN THE BOOKLET AND SHOW YOUR WORK CLEARLY.}
A correct but unclear answer will not get full marks, nor will a correct answer
that does not give some detail of the method used to obtain the solution.
\begin{center}
\begin{tabular}{cp{15.5cm}}
\hline
&\\
\end{tabular}
\end{center}



\begin{questions}
\question[20]
Consider the following Lotka-Volterra model for competition
\begin{equation*}
\begin{aligned}
\dxdt &= \alpha x (1 - 2 x -   y)			\\
\dydt &= \beta  y \left(1 - \frac 15 x - \frac 19 y\right)
\end{aligned}
\end{equation*}
where $\alpha$ and $\beta$ are positive constants.
\begin{parts}
\item Find the (biologically meaningful) equilibria.
\item By considering the Jacobian matrix, determine whether each
equilbrium is stable or unstable.
\item Sketch the phase plane including the nullclines, equilibria
and the general direction of flow at different points.
\end{parts}
\question[10]
Consider the general Lotka-Volterra model for competition
between two species:
\begin{equation*}
\begin{aligned}
\dxdt &= r_1 x (1 - a_{11} x - a_{12} y)	\\
\dydt &= r_2 y (1 - a_{21} x - a_{22} y).
\end{aligned}
\end{equation*}
Modify the above equations to produce a model for the situation
where species $x$ and $y$ help each other but there is intraspecific competition for both species.


\question[20]
Consider an infectious disease for which the population is
to be divided into four subgroups: susceptibles, exposed (but
not yet infectious), infectious and recovered.  Let the sizes of
these four groups be given by $S$, $E$, $I$, and $R$, respectively.
Assume that the average number of contacts that each susceptible
has with infectives is $c$ times the number of infectives.  Let the
probability that a contact between an infective and a susceptible
be given by $\beta$.  Suppose that the average durations spent in
the exposed and infectious classes are $1/\epsilon$ and $1/\gamma$
respectively.  Assume that there is a constant rate birth rate
$B$, with all newborns entering the susceptible class.  Assume
that the death rate (not directly related to the disease is)
for each population subgroup is proportional to the size of
that subgroup.
\begin{parts}
\item Draw the transfer diagram.
\item Give differential equations which describe the rate of change
of the number of susceptibles and the number of infectives.
\item Find the disease-free equilibrium ($I=0$).
\item Find the endemic equilibrium ($I>0$) and the basic reproduction number $\mathcal{R}_0$. [Hint: the latter is such that the disease free equilibrium undergoes a bifurcation at $\mathcal{R}_0=1$.]
\end{parts}


\question[20]
Consider the $SIR$ model where a fraction $p$ of all
newborns are successfully vaccinated, given by the transfer
diagram below.  Assume that the model is for a town of size
$50,000$ and that the total population size is constant.
Suppose $c=.01$ contacts per month per infective per susceptible,
the probability of transmission given that a contact occurs is
$\beta = .03$, the average duration of infectivity (before recovering)
is $\frac 12$ month, the life expectancy is $70$ years.

$$
\xymatrix{
\ar[d]_{(1-p)bN} &&&& \ar[d]^{pbN}	\\
S \ar[rr]^{c \beta SI} \ar[d]^{d S} & &
I \ar[rr]^{\gamma I} \ar[d]^{d I} & &
R \ar[d]^{d R} & &   \\
&&&&&&
}
$$
\begin{parts}
\item Write down the differential equations for this model.
\item Find the basic reproduction number $\mathcal{R}_0$ in the absence of vaccination (i.e. $p=0$).
\item Find the reproduction number with vaccination $\mathcal{R}_{0,vac}$ (as a function of $p$).
\item What fraction of newborns has to be {\it successfully}
vaccinated in order to make $\mathcal{R}_{0,vac}$ less than one?
\item Suppose the vaccine is successful in $95 \%$ of the cases in which it is used.  What fraction of newborns has to be vaccinated to make $\mathcal{R}_{0,vac}$ less than one?
\end{parts}


\question[20]
A gambler plays repeatedly a game in which on each play he wins one dollar with probability $p$ and loses one dollar with probability $q=1-p$. 

The \emph{Gambler's Ruin problem} consists in finding the probability $w_x$ of winning an amount $T$ before losing everything, starting with state $x$. 
\begin{parts}
\item Show that this problem may be considered to be an absorbing Markov chain with states $0,1,2,\ldots,T$ with $0$ and $T$ absorbing states.
\item Show that $w_x$ satisfies the following conditions:
\begin{itemize}
\item[(i)] $w_x = pw_{x+1} + qw_{x-1}$ for $x=1,\ldots,T-1$.
\item[(ii)] $w_0 = 0$.
\item[(iii)] $w_T = 1$.
\end{itemize}
\item Show that conditions (i), (ii) and (iii) determine $w_x$.
\item Show that, if $p=q=1/2$, then
\[
w_x = \frac xT
\]
satisfies (i), (ii), and (iii) and hence is the solution. 
\item If $p\neq q$, show that
\[
w_x=\frac{(q/p)^x-1}{(q/p)^T-1}
\]
satisfies conditions (i), (ii) and (iii) and hence gives the probability of the gambler winning.
\end{parts}

 


\end{questions}
\end{document}
