\documentclass[12pt]{article}

\usepackage[active]{srcltx}

\usepackage{amsmath,amsfonts,amssymb,amsthm}
\usepackage{easybmat}
\usepackage{graphicx}
%\usepackage[hypertex]{hyperref}
%\usepackage[pdftex]{hyperref}
\usepackage{fancyhdr}
\usepackage{subfigure}

%\usepackage{makeidx}
%\makeindex

\theoremstyle{plain}
\newtheorem{theorem}{Theorem}
\newtheorem{property}[theorem]{Property}
\newtheorem{condition}[theorem]{Condition}
\newtheorem{proposition}[theorem]{Proposition}
\newtheorem{axiom}[theorem]{Axiom}
\newtheorem{lemma}[theorem]{Lemma}
\newtheorem{corollary}[theorem]{Corollary}
%%
% % Make numbering specific to the Appendix
% \newtheorem{Atheorem}{Theorem}[chapter]
% \newtheorem{Aproperty}[Atheorem]{Property}
% \newtheorem{Acondition}[Atheorem]{Condition}
% \newtheorem{Aproposition}[Atheorem]{Proposition}
% \newtheorem{Aaxiom}[Atheorem]{Axiom}
% \newtheorem{Alemma}[Atheorem]{Lemma}
% \newtheorem{Acorollary}[theorem]{Corollary}

% Make definitions non italicized
%\theoremstyle{definition}
\newtheorem{definition}[theorem]{Definition}
% Make numbering specific to the Appendix
%\newtheorem{Adefinition}[Atheorem]{Definition}
\newenvironment{defi}{\vskip0.2cm\addtocounter{theorem}{1}\par\noindent\bf Definition~\arabic{section}.\arabic{subsection}.\arabic{theorem}\rm.}{\hfill{$\circ$}\par\vskip0.25cm}


%% Example environment and counter.
\newcounter{cmpt_exercise}
\newenvironment{exercise}{\addtocounter{cmpt_exercise}{1}\vskip0.2cm\par\noindent\begin{small}\bf Exercise~\arabic{cmpt_exercise}\,\,\rm --}{\hfill{$\circ$}\end{small}\par\vskip0.25cm}
\newenvironment{problem}{\addtocounter{cmpt_exercise}{1}\vskip0.2cm\par\noindent{\Large\bf Problem~\arabic{cmpt_exercise}\,\,\rm --}}{\par\vskip0.25cm}


\newenvironment{example}{\vskip0.2cm\par\noindent\begin{small}\bf Example\,\,\rm --}{\hfill{$\diamond$}\end{small}\par\vskip0.25cm}
\newenvironment{remark}{\vskip0.2cm\par\noindent\begin{small}\bf Remark\,\,\rm --}{\hfill{$\circ$}\end{small}\par\vskip0.25cm}
%\newenvironment{aparte}[1]{\vskip0.2cm\par\noindent\begin{quote}\begin{small}\bf Apart\'e : #1\,\,\rm --}{\hfill{$\circ$}\end{small}\end{quote}\par\vskip0.25cm}
\newenvironment{aparte}[1]{\vskip0.3cm\par\begin{center}\begin{tabular}{|p{0.9\textwidth}|}\hline{\bf Apart\'e : #1}}{\\ \hline\end{tabular}\end{center}\par\vskip0.25cm}

\renewcommand{\labelenumi}{\roman{enumi})}
\renewcommand{\labelenumii}{\alph{enumii})}
\newcommand{\espv}{\vspace{.5\baselineskip}}
\def\IR{\mathbb{R}}
\def\IC{\mathbb{C}}
\def\IN{\mathbb{N}}
\def\IQ{\mathbb{Q}}
\def\IZ{\mathbb{Z}}
\def\rank{\textrm{rank }}
\def\Sp{\textrm{Sp }}
\def\Span{\textrm{Span }}
\def\Tr{\textrm{Tr }}
\def\D{\mathcal{D}}
\def\I{\mathcal{I}}
\def\U{\mathcal{U}}
\def\R{\mathcal{R}}
\def\Q{\mathcal{Q}}
\def\O{\mathcal{O}}
\def\Mn{\mathcal{M}_n}
\def\NN#1{\|#1\|}
\def\N3#1{|\!|\!|#1|\!|\!|}
\def\diag{\textrm{diag}}
\def\tr{\textrm{tr}}
\def\ker{\textrm{Ker }}

\def\M{\mathcal{M}}

\setlength{\textwidth}{17cm} 
\addtolength{\oddsidemargin}{-1.5cm}
\setlength{\textheight}{22cm}
\addtolength{\topmargin}{-2cm} 
\setlength{\headheight}{25.3pt}

%% Fancyhdr related stuff
\pagestyle{fancy}
\lhead{MATH 3820 -- Intro Math Modelling -- Final Examination solutions}
\rhead{\thepage}
\cfoot{}

\usepackage[hang,small,bf]{caption}
\setlength{\captionmargin}{20pt}

\makeatletter
\def\cleardoublepage{\clearpage\if@twoside \ifodd\c@page\else
\hbox{}
% \vspace*{\fill}
% \begin{center}
% This page intentionally contains only this sentence.
% \end{center}% Make numbering specific to the Appendix
\newtheorem{Atheorem}{Theorem}[chapter]
\newtheorem{Aproperty}[Atheorem]{Property}
\newtheorem{Acondition}[Atheorem]{Condition}
\newtheorem{Aproposition}[Atheorem]{Proposition}
\newtheorem{Aaxiom}[Atheorem]{Axiom}
\newtheorem{Alemma}[Atheorem]{Lemma}
\newtheorem{Acorollary}[theorem]{Corollary}

% \vspace{\fill}
\thispagestyle{empty}
\newpage
\if@twocolumn\hbox{}\newpage\fi\fi\fi}
\makeatother

%\author{Julien Arino}
%\address{University of Manitoba}
%\title{MATH 8430\\ Lecture Notes}
\title{University of Manitoba\\ Math 38200 -- Winter 2008}
\author{Final Examination}
\date{Solutions}

\renewcommand{\abstractname}{Remarks}
%%%%%%%%%%%%%%%%
%%%%%%%%%%%%%%%%
%%%%%%%%%%%%%%%%
%%%%%%%%%%%%%%%%
%%%%%%%%%%%%%%%%
%%%%%%%%%%%%%%%%
\begin{document}
\maketitle

\textbf{1.a.}
Equilibria satisfy $x'=y'=0$, so at the equilibrium, we must have
\begin{subequations}\label{sol_final_w2008:exo1_EP}
\begin{align}
\alpha x(1-2x-y) &=0 \label{sol_final_w2008:exo1_EP_1}\\
\beta y\left(1-\frac 15x-\frac 19y\right) &=0.\label{sol_final_w2008:exo1_EP_2}
\end{align}
\end{subequations}
From \eqref{sol_final_w2008:exo1_EP_1}, $x=0$ or $y=1-2x$. Substituting the value $x=0$ into \eqref{sol_final_w2008:exo1_EP_2}, we get $y=0$ and $y=9$, so we have the two equilibria $(x,y)=(0,0)$ and $(x,y)=(0,9)$. Substituting the value $y=1-2x$ into \eqref{sol_final_w2008:exo1_EP_2}, we get
\[
\beta(1-2x)\left(
1-\frac 15 x-\frac 19 (1-2x)
\right)=0,
\]
so $x=1/2$ (and thus $y=1-2x=0$) or 
\[
45-9x-5(1-2x)=0,
\]
that is $-40=x$. This latter point is not biologically relevant, so the only other equilibrium we find this way is the point $(x,y)=(1/2,0)$.

\textbf{1.b.}
The Jacobian matrix at an arbitrary point $(x,y)$ takes the form
\begin{equation}\label{sol_final_w2008:exo1_jacobian}
J=
\begin{pmatrix}
\alpha-4\alpha x-\alpha y & -\alpha x \\
-\dfrac 15\beta y & \beta-\dfrac 15\beta x -\dfrac 29\beta y
\end{pmatrix}.
\end{equation}
Evaluated at $(x,y)=(0,0)$, \eqref{sol_final_w2008:exo1_jacobian} is
\begin{equation}\label{sol_final_w2008:exo1_jacobian_EP1}
J=
\begin{pmatrix}
\alpha & 0 \\
0 & \beta
\end{pmatrix},
\end{equation}
so eigenvalues are $\alpha$ and $\beta$. Since $\alpha$ and $\beta$ are positive, $(0,0)$ is unstable. Evaluated at $(x,y)=(0,9)$, \eqref{sol_final_w2008:exo1_jacobian} is
\begin{equation}\label{sol_final_w2008:exo1_jacobian_EP2}
J=
\begin{pmatrix}
-8\alpha & 0 \\
-\dfrac 95\beta & -\beta
\end{pmatrix}.
\end{equation}
Eigenvalues of \eqref{sol_final_w2008:exo1_jacobian_EP2} are $-\beta$ and $-8\alpha$, so the equilibrium is locally asymptotically stable. Finally, at $(x,y)=(1/2,0)$, \eqref{sol_final_w2008:exo1_jacobian} is
\begin{equation}\label{sol_final_w2008:exo1_jacobian_EP3}
J=
\begin{pmatrix}
-\alpha & -\dfrac\alpha2 \\
0 & \frac 9{10}\beta
\end{pmatrix}.
\end{equation}
Therefore, eigenvalues of \eqref{sol_final_w2008:exo1_jacobian_EP3} are $-\alpha$ and $9\beta/10$, so the equilibrium point $(1/2,0)$ is unstable (a saddle point).

\textbf{1.c.}



\vskip1cm
\textbf{2.} It suffices to assume that $r_1,r_2,a_{11}$ and $a_{21}$ are positive parameters, while $a_{12}$ and $a_{22}$, which describe the interactions between species, are negative parameters.

\vskip1cm
\textbf{3.a.}

\textbf{3.b.}
\begin{subequations} \label{sol_final_w2008:exo3_sys}
\begin{align}
S' &= B-c\beta SI-dS \label{sol_final_w2008:exo3_sysS} \\
E' &= c\beta SI -\varepsilon E-dE \label{sol_final_w2008:exo3_sysE}\\
I' &= \varepsilon E-\gamma I-dI \label{sol_final_w2008:exo3_sysI}\\
R' &= \gamma I -dR \label{sol_final_w2008:exo3_sysR}
\end{align}
\end{subequations}
(The question only asked for equations \eqref{sol_final_w2008:exo3_sysS} and \eqref{sol_final_w2008:exo3_sysI}, but for the rest of the exercise, it was better to give all four.)

\textbf{3.c.}
Setting $I=0$ in \eqref{sol_final_w2008:exo3_sys} and assuming that the system is at equilibrium, i.e., $S'=E'=I'=R'=0$, we see from \eqref{sol_final_w2008:exo3_sysE} that at the disease free equilibrium, $E=0$, and from \eqref{sol_final_w2008:exo3_sysR}, at the disease free equilibrium, $R=0$. Finally, substituting $I=0$ in \eqref{sol_final_w2008:exo3_sysS}, $S=B/d$. So the disease free equilibrium is $(S,E,I,R)=(B/d,0,0,0)$.

\textbf{3.d.} 
We now seek an equilibrium solution such that $I>0$. From \eqref{sol_final_w2008:exo3_sysI}, at this equilibrium,
\[
E=\frac{\gamma+d}{\varepsilon}\;I
\]
and from \eqref{sol_final_w2008:exo3_sysR},
\[
R=\frac\gamma d\;I.
\]
Substituting the value found for $E$ in \eqref{sol_final_w2008:exo3_sysI}, we have that at the equilibrium,
\[
\left(c\beta S-\varepsilon\frac{\gamma+d}{\varepsilon}
-d\frac{\gamma+d}{\varepsilon}\right)I=0.
\]
Since we are assuming $I>0$ (the case $I=0$ having already been treated earlier), we must have
\[
c\beta S-(\varepsilon+d)\frac{\gamma+d}{\varepsilon}=0,
\]
that is
\[
S=\frac{\varepsilon+d}{c\beta}\;\frac{\gamma+d}{\varepsilon}.
\]
Finally, from \eqref{sol_final_w2008:exo3_sysS},
\[
I=\frac{B-dS}{c\beta S},
\]
which, when substituting the expression found for $S$ above, gives
\begin{align*}
I &=
\frac{B-d\frac{\varepsilon+d}{c\beta}\;\frac{\gamma+d}{\varepsilon}}
{c\beta\frac{\varepsilon+d}{c\beta}\;\frac{\gamma+d}{\varepsilon}} \\
&= \frac{B-d\frac{\varepsilon+d}{c\beta}\;\frac{\gamma+d}{\varepsilon}}
{(\varepsilon+d)\;\frac{\gamma+d}{\varepsilon}} \\
&=
\frac{Bc\beta\varepsilon-d(c+d)(\gamma+d)}{c\beta\varepsilon}\;
\frac{\varepsilon}{(\varepsilon+d)(\gamma+d)} \\
&=
\frac{Bc\beta\varepsilon-d(c+d)(\gamma+d)}{c\beta(\varepsilon+d)(\gamma+d)}.
\end{align*}
Substituting this expression into those found for $E$ and $R$ gives
\[
E=\frac{Bc\beta\varepsilon-d(c+d)(\gamma+d)}{c\beta\varepsilon(\gamma+d)}
\]
and
\[
R=\frac{Bc\beta\varepsilon\gamma-d\gamma(c+d)(\gamma+d)}{cd\beta(\varepsilon+d)(\gamma+d)}
\]



\end{document}