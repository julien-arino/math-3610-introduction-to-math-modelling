\documentclass[reqno,12pt]{amsart}
%\documentclass[final,letter]{siamltex}
%\documentstyle{amsppt}


\usepackage{amssymb}
\usepackage{amscd}
\usepackage{graphicx}
\usepackage{overcite}

\input xy
\xyoption{all}

\setlength{\oddsidemargin}{0.0in}
\setlength{\evensidemargin}{0.0in}
\setlength{\textwidth}{6.5in}

\def\eee{\textrm{e}}
\def\dxdt{\frac{dx}{dt}}
\def\dydt{\frac{dy}{dt}}
\def\dHdt{\frac{dH}{dt}}
\def\dLdt{\frac{dL}{dt}}

\begin{document}
\title{M2E03 - Introduction to Modelling - Assignment 4}
\maketitle

\begin{center}
% \pagestyle{myheadings} 
% \thispagestyle{empty}

% \textsc{\bf M2E03 - Introduction to Modelling - Assignment 4}

\textsc{Due in class - 11:00 am, Friday, November 28}

\textsc{Late assignments will receive a grade of zero.}
\end{center}

\vspace{1cm}

\noindent
1.  Consider a non-fatal infectious disease (like the common cold) for
which each individual in the population is either susceptible or infective.
Assume that the average number of contacts that each susceptible has
with infectives is $c$ times the number of infectives.  Let the probability
that a contact between an infective and a susceptible results in transmission
of the disease be given by $\beta$.  Let the average duration of infectivity
be given by $1/\gamma$.  Assume that the rate of birth is proportional to
the total population size, with all new individuals being added to the
susceptible class, and that the death rate (not directly related to the
disease) for each population subgroup is proportional to the size of that
subgroup.

\bigskip
\noindent
a.  Draw the transfer diagram.

\bigskip
\noindent
b.  Give differential equations which describe the rate of change
of the number of susceptibles and the number of infectives.

\bigskip
\noindent
c.  Make the additional assumption that the total population size
remains constant.  Rewrite the differential equation.

\bigskip
\noindent
d.  For the situation for which the population size is constant, do
the following.

i.\;\;  Find the equilibria and determine when each equilibrium is stable.

ii.\;  Find the basic reproduction number $R_0$.

iii. Draw the bifurcation diagram ($I$ at the equilibria versus $R_0$).

\newpage
\noindent
2.  Consider an infectious disease for which the population is
to be divided into four subgroups: susceptibles, exposed (but
not yet infectious), infectious and recovered.  Let the sizes of
these four groups be given by $S$, $E$, $I$, and $R$, respectively.
Assume that the average number of contacts that each susceptible
has with infectives is $c$ times the number of infectives.  Let the
probability that a contact between an infective and a susceptible
be given by $\beta$.  Suppose that the average durations spent in
the exposed and infectious classes are $1/\epsilon$ and $1/\gamma$
respectively.  Assume that there is a constant rate birth rate
$B$, with all newborns entering the susceptible class.  Assume
that the death rate (not directly related to the disease is)
for each population subgroup is proportional to the size of
that subgroup.

\bigskip
\noindent
a.  Draw the transfer diagram.

\bigskip
\noindent
b.  Give differential equations which describe the rate of change
of the number of susceptibles and the number of infectives.

\bigskip
\noindent
c.  Find the disease-free equilibrium ($I=0$).

\bigskip
\noindent
d.  Find the endemic equilibrium ($I>0$) and $R_0$.


\newpage
\noindent
3.  Consider the $SIR$ model where a fraction $p$ of all
newborns are successfully vaccinated, given by the transfer
diagram below.  Assume that the model is for a town of size
$50,000$ and that the total population size is constant.
$$
\xymatrix{
\ar[d]_{(1-p)bN} &&&& \ar[d]^{pbN}	\\
S \ar[rr]^{c \beta SI} \ar[d]^{d S} & &
I \ar[rr]^{\gamma I} \ar[d]^{d I} & &
R \ar[d]^{d R} & &   \\
&&&&&&
}
$$

\bigskip
\noindent
a.  Write down the differential equations for this model.

\bigskip
Suppose $c=.01$ contacts per month per infective per susceptible,
the probability of transmission given that a contact occurs is
$\beta = .03$, the average duration of infectivity (before recovering)
is $\frac 12$ month, the life expectancy is $70$ years.
\bigskip

\noindent
b.  Find $R_0$ in the absence of vaccination (i.e. $p=0$).

\bigskip
\noindent
c.  Find $R_{0,vac}$ (as a function of $p$).

\bigskip
\noindent
d.  What fraction of newborns has to be {\it successfully}
vaccinated in order to make $R_{0,vac}$ less than one?

\bigskip
\noindent
e.  Suppose the vaccine is successful in $95 \%$ of the cases in which
it is used.  What fraction of newborns has to be vaccinated to make
$R_{0,vac}$ less than one?




\newpage
\noindent
4.  {\bf Spreading Rumors:}  Construct a model for spreading rumors making
use of the following assumptions.  The total population size $N$ is
constant with noone entering or leaving the population.  The population
is split into three subgroups: the uncool (who have not heard the rumor),
the rumor spreaders, and the cool (who have heard the rumor, but are not
currently spreading it).  Let the sizes of these groups be given by $U$,
$R$, and $C$ respectively.

As soon as someone in the uncool group hears the rumor, they enter the
group of rumor spreaders.  Each rumor spreader tells the rumor at rate
$k$.  Thus, the rate at which the rumor is being told is $k$ times the
number of rumor spreaders.  So, the rate at which uncool people hear the
rumor is $k$ times the number of rumor spreaders times the {\it fraction}
of the population consisting of uncool people.

When a rumor spreader tells the rumor to someone who has already heard
it (i.e. someone in the rumor spreading group or the cool group), they
stop spreading the rumor and enter the cool group.  Thus, the rate at
which rumor spreaders enter the cool group is $k$ times the number of
rumor spreaders times the {\it fraction} of the population consisting
of rumor spreaders and cool people.

\vspace{1cm}

\noindent
a.  Draw the transfer diagram.

\bigskip
\noindent
b.  Write down the differential equations.

\bigskip
\noindent
c.  Rewrite the equations for $\frac{dU}{dt}$ and $\frac{dR}{dt}$
without using the variable $C$ by making the substitution $C=N-U-R$.

\bigskip
\noindent
d.  Find the equilibria for the two dimensional system given in part (c).

\bigskip
\noindent
e.  Suppose that when the rumor starts, it is initially known by a
very small fraction (much less than half) of the population and that
this group of people are all rumor spreaders.  Suppose that everyone
else in the population is in the uncool group.  Then initially, this
should give $\frac{dR}{dt}>0$.  (Check that your equations in part (c)
support this.)  $U$ will decrease (from nearly $N$) until the rumor
dies out.  {\bf At what value of $U$ does $R$ stop increasing?}
Eventually the rumor will die out when $R$ goes to zero. {\bf when the
rumor dies out, are there more people in the uncool group or in the
cool group?}












\end{document}
