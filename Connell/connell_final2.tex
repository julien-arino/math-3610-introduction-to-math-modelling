\documentclass[reqno,12pt]{amsart}

\setlength{\oddsidemargin}{0.0in}
\setlength{\evensidemargin}{0.0in}
\setlength{\textwidth}{6.5in}

\def\eee{\textrm{e}}
\def \dNdt{\frac{dN}{dt}}
\def \dxdt{\frac{dx}{dt}}
\def \dydt{\frac{dy}{dt}}
\def \dXdt{\frac{dX}{dt}}
\def \dYdt{\frac{dY}{dt}}

\begin{document}

\begin{center}
{\bf M2E03 - INTRODUCTION TO MODELLING - EXAM}
\end{center}

\vspace{.5cm}
\noindent
{\bf Please write your solutions in the answer booklet.}
Notes, books and scrap paper are not allowed.
CALCULATORS:  You may only use the McMaster Standard FX-991 Calculator.

\vspace{.5cm}
\noindent
You must sign the class list.
Please put out your Student Card for checking.




\vspace{1cm}


\noindent
{\bf 1.}  Consider a species that lives for two years and may reproduce
at the end of year one or year two.  Assume that the population is
undergoing density independent growth.

\noindent
{\bf a.} Construct a model to describe the population size where

- the mean number of offspring that 0-year olds have the following
year is .6 

- the mean number of offspring that 1-year olds have the following
year is 4

- the probability that a 0-year old survives to be a 1-year old
is 0.2.

\noindent
{\bf b.}  Suppose that the initial population consists of fifty 0-year
olds and ten 1-year olds.  How many 0-year olds and how many
1-year olds will there be one year later.

\noindent
{\bf c.}  Find the growth rate and the stable age distribution.

\vspace{1cm}

\noindent
{\bf 2.}  Consider a single locus with two alleles $A$ and $B$.  Let
$p$ be the frequency of gene $A$, and let $q$ be the frequency of
gene $B$.  Assume that the population is in a mutation-selection
balance where selection favors gene $A$ over gene $B$, but a
fraction $\mu$ of genes of type $A$ mutate to become genes of
type $B$.

Suppose $\mu = .001$ and the relative genotype fitnesses are
$(w_{AA} , w_{AB}, w_{BB}) = (1,0.98,0.98)$.  Find $p$ and $q$.


\vspace{1cm}

\noindent
{\bf 3.}  Consider the following equations which model competition between
two species whose population sizes are given by $x$ and $y$.
$$
\begin{aligned}
\dxdt &= 3x (1 - 2x - y)	\\
\dydt &= 4y (1 - 3x - 2y)
\end{aligned}
$$

\noindent
{\bf a.}  Sketch the phase plane, including the nullclines and the
equilibria (clearly labeled).  Show the general direction of
flow at different parts of the phase plane.

\noindent
{\bf b.}  Assuming $x(0), y(0) > 0$, find
$$
\lim_{t \to \infty} \bigl( x(t), y(t) \bigr).
$$



\vspace{1cm}

\noindent
{\bf 4.}  Consider the following equations which model a predator-prey
relationship between two species whose population sizes are given
by $X$ and $Y$.
$$
\begin{aligned}
\dXdt &= X ( -6 + 3Y)		\\
\dYdt &= Y ( 16 - 2X - 4Y).
\end{aligned}
$$


\noindent
{\bf a.} Which variable represents the population size of the predator
species?  Justify your answer.

\noindent
{\bf b.}  Sketch the phase plane, including the nullclines and the
equilibria (clearly labeled).  Show the general direction of
flow at different parts of the phase plane.

\noindent
{\bf c.}  Use the Jacobian matrix to determine the stability of the
equilibrium for which each population size is positive.



\vspace{1cm}

\noindent
{\bf 5.}  Consider an infectious disease for which each individual
in the population is either susceptible or infective.  The rate
at which susceptibles become infective is $c \beta S I$.  The
rate at which infectives recover from the disease (becoming
susceptible again) is $\gamma I$.  Births occur at the constant
rate $B$.  The death rate for each population subgroup is
proportional to the size of that group, but with a higher death
rate for the infective group.  Thus, the death rate for the
susceptibles is $d S$ and the death rate for the infectives
is $(d+\nu) I$.

\noindent
{\bf a.}  Draw the transfer diagram.

\noindent
{\bf b.}  Give the differential equations which describe the rate of
change of the number of susceptibles and the number of infectives.

\noindent
{\bf c.}  Find the equilibria.

\noindent
{\bf d.}  Find $R_0$.

\noindent
{\bf e.}  Use the Jacobian matrix to determine the stability of
the disease-free equilibrium for $R_0 < 1$.

\noindent
{\bf f.}  Use the Jacobian matrix to determine the stability of
the disease-free equilibrium for $R_0 > 1$.


\vspace{1cm}

\noindent
{\bf 6.a.}  Construct a continuous time model for the population size
$N$ of a country under the following circumstances.  The country
allows immigrants to enter the country at a rate $B$.  Also, the
birth rate $bN$ and the death rate $dN$ are proportional to the
population size, where $d>b$.  (HINT: It may be helpful to draw a
flow diagram, similar to the transfer diagram that is used for an
epidemic model.)

\noindent
{\bf b.}  Find the equilibria and determine the stability of each equilbrium.

\noindent
{\bf c.}  Suppose $b=.009$, $d=.014$, and $B=175,000$.  If the
initial population is $30,000,000$, find
$$
\lim_{t \to \infty} N(t).
$$


\vspace{1cm}

\noindent
{\bf 7.}  For each of the following payoff matrices, determine which pure
strategies are evolutionarily stable.

\noindent
{\bf a.}
$$
\begin{matrix}
  & \begin{matrix} A & B & C \end{matrix} 	\\
\begin{matrix} A \\ B \\ C \end{matrix}		&
\begin{bmatrix}
3 & 2 & 1 	\\
1 & 7 & 1	\\
4 & 6 & 2
\end{bmatrix}
\end{matrix}
$$

\noindent
{\bf b.}
$$
\begin{matrix}
  & \begin{matrix} D & E \end{matrix} 	\\
\begin{matrix} D \\ E \end{matrix}		&
\begin{bmatrix}
3 & 4 	\\
3 & 1
\end{bmatrix}
\end{matrix}
$$



\vspace{1cm}

\noindent
{\bf 8.}  Consider the following game which can be used to model
the behaviour of female lions hunting in pairs.  Suppose that the
lions have killed a gazelle and that the gazelle is interpreted
as having an energy value $v$.  Each lion can either share or be
greedy.  If both lions share, then each lion will eat half the
carcass.  If one is greedy and the other is willing to share,
then the greedy lion will eat $3/4$ of the carcass and leave
the remaining $1/4$ for her partner.  If both lions are greedy,
then they will each end up with half the carcass, but will have
expended energy equal to $3v/4$.

\noindent
{\bf a.}  Give the payoff matrix.

\noindent
{\bf b.}  Find an evolutionarily stable strategy.



\end{document}
