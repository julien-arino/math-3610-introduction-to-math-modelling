\documentclass[reqno,12pt]{amsart}

\setlength{\oddsidemargin}{0.0in}
\setlength{\evensidemargin}{0.0in}
\setlength{\textwidth}{6.5in}

\def\eee{\textrm{e}}
\def \dNdt{\frac{dN}{dt}}
\def \dxdt{\frac{dx}{dt}}
\def \dydt{\frac{dy}{dt}}


\begin{document}
\title{M2E03 - Introduction to Modelling - More Practise Problems}
\maketitle

\noindent
0.  Go back and do the Practise Problems that were handed out
before the midterm.

\vspace{1cm}

\noindent
1.  Consider the following equations which model competition
between two species whose population sizes are given by $x$ and
$y$.
$$
\begin{aligned}
\dxdt &= 5x (1 -  x - 5y)        \\
\dydt &= 2y (1 - 3x - 2y)
\end{aligned}
$$

\noindent
a.  Sketch the phase plane including the nullclines, equilibria
(clearly marked) and the general direction of the flow in different
parts of the plane.

\noindent
b.  Use the Jacobian matrix to determine the stability of each of
the equilibria.


\vspace{1cm}

\noindent
2. a.  Write down a set of equations that gives a model of a predator-prey
interaction between two species.  Assume that in the absence of the prey,
that the predator dies out; and that in the absence of the predator, the
prey limits to a positive constant level.

\noindent
b.  Sketch the phase plane including the nullclines, equilibria
(clearly marked) and the general direction of the flow in different
parts of the plane.

\noindent
c.  Use the Jacobian matrix to determine the stability of each of
the equilibria.

\vspace{1cm}

\noindent
3.  Consider an infectious disease for which each individual in the
population is either susceptible, infective or recovered (and temporarily
immune).  Let the number of people of each of these types be $S$, $I$ and
$R$ respectively.  Assume that the average number of contacts that a
susceptible has with infectives is $c$ times the number of infectives.
Let the probability that a contact between a susceptible and an infective
results in transmission of the disease be $\beta$.  Let the average
duration of infectivity (before recovering) be $1/\gamma$.  Let the
average period of immunity after recovery (before becoming susceptible
again) be $1/\delta$.

Assume that the arrival of new people in the population is through
a constant level of immigration $B$.  Assume that the death rate
(non directly related to the disease) for each population subgroup
is proportional to the size of that subgroup, with the average life
expectancy being $1/d$.

\noindent
a.  Give the transfer diagram.

\noindent
b.  Give the differential equations for $S$, $I$ and $R$.

\noindent
c.  Find the equilibria.

\noindent
d.  Find $R_0$.

\noindent
e.  Let $N = S+I+R$ be the total population size.  Give
the differential equation for $N$.

\noindent
f.  Find the equilibrium values of $N$.  Find the stability.

\noindent
g.  Assuming that $N(0) > 0$, find the limit as $t$ goes to
infinity of $N(t)$.

\noindent
h.  For the following calculations, make sure that all of your
parameters make use of the same time unit.  i.e. use only years
or only months or only days ...

Pick a reasonable value for $d$ based on its definition.
Assuming that the equilibrium value of $N$ that you found in
part {\it f} is the size of the population in a city that is
being studied, pick a reasonable value for population size.
Assuming that the population size is in equilibrium, calculate
$B$.  Based on the definitions of $\gamma$ and $\delta$ pick
reasonable values for each.

Suppose $\beta = .05$ and $R_0 = 10$.  Find $c$.

Based on these numbers, find $S$, $I$ and $R$ at the endemic
equilibrium.

\vspace{1cm}

\noindent
4.  For each of the following payoff matrices, determine which pure
strategies are evolutionarily stable.

\noindent
$$
\begin{matrix}
\begin{matrix}
  & \begin{matrix} A & B & C \end{matrix}       \\
\begin{matrix} A \\ B \\ C \end{matrix}         &
\begin{bmatrix}
5 & 1 & 1       \\
1 & 8 & 4       \\
4 & 5 & 4
\end{bmatrix}
\end{matrix}
& \qquad &
\begin{matrix}
& \begin{matrix} D & E \end{matrix}   \\
\begin{matrix} D \\ E \end{matrix}              &
\begin{bmatrix}
7 & 8   \\
9 & 2
\end{bmatrix}
\end{matrix}
& \qquad &
\begin{matrix}
& \begin{matrix} F & G \end{matrix}   \\
\begin{matrix} F \\ G \end{matrix}              &
\begin{bmatrix}
-1 & 0   \\
1 & 2
\end{bmatrix}
\end{matrix}
\end{matrix}
$$

\vspace{1cm}

\noindent
5.  Consider the following situation.  Two roommates are each
considering the snow-shovelling strategy that they will adopt
for the coming winter.  Each has the option of being Diligent
(and shovelling the snow whenever they see that it needs shovelling)
or Delinquent (and never shovelling the snow).  If the front steps
are not shovelled, then there is a risk of injury to each roommate
which is assigned a cost $C$.  The cost (in time and effort) of
keeping the front steps clear is $E$.

If they are both Diligent, then they split the cost of clearing
the steps.  If one is Diligent and the other is Delinquent, then
the Diligent roommate expends all the energy to clear the steps
and neither roommate is at risk of injury.  If both roommates
are Delinquent, then the steps are never cleared and the expected
cost to each roommate is $C$.

Note that since there is a risk of injury if the steps are not
clear, and energy must be expended in order to clear the steps,
payoffs can be negative.

\noindent
a.  Give the payoff matrix.

\noindent
b.  Find the pure evolutionarily stable strategies (ESS) for $C \le E$.

\noindent
c.  Find an ESS for $C>E$.

\end{document}
