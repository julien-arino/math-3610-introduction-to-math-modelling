\documentclass[reqno,12pt]{amsart}

\usepackage{amssymb}
\usepackage{amscd}
\usepackage{graphicx}
\usepackage{overcite}

\setcounter{MaxMatrixCols}{10}

\theoremstyle{definition}
\theoremstyle{remark}
\numberwithin{equation}{section}
\newcommand{\thmref}[1]{Theorem~\ref{#1}}
\newcommand{\secref}[1]{\S\ref{#1}}
\newcommand{\lemref}[1]{Lemma~\ref{#1}}

\setlength{\oddsidemargin}{0.0in}
\setlength{\evensidemargin}{0.0in}
\setlength{\textwidth}{6.5in}

\def\eee{\textrm{e}}

\begin{document}
\title{M2E03 - Introduction to Modelling - Assignment 1}
\maketitle

\begin{center}
\pagestyle{myheadings} 
\thispagestyle{empty}

\bigskip

\textsc{Due 9:00 am, Wednesday, September 24}

\bigskip

\textsc{Late assignments will receive a grade of zero.}

\bigskip

You do not have to answer the questions that are written in {\it italics}.

\bigskip
\bigskip
\end{center}

1.  Evans and Smith (1952) calculated $r$ for the human louse and
found it to be approximately $0.1$ per day.  Start with the equation
$$
N(t) = N(0) \eee^{rt}
$$
and rearrange to get
$$
t = \frac{\ln \bigl( N(t) / N(0) \bigr)}r.
$$
Using this equation, if we know $r$, $N(0)$, and $N(t)$, we can
find $t$.

Starting with $10$ lice, how long will it take for an exponentially
growing population of lice to reach $1000$?  $100,000$?  $10,000,000$?
$1,000,000,000$?  {\it Does this surprise you?}

\bigskip

2.  In a population (of an imaginary organism that lives upto $2$ years),
the average number of offspring that $0$-year-olds have the following
year is $\frac 12$, the average number of offspring that $1$-year-olds
have the following year is $2$, and the probability that a $0$-year-old
survives to age $1$ is $\frac 12$.  Death is certain after $2$ years.

\noindent
(a) Set up a Leslie matrix model for this population.

\noindent
(b) If the population starts with $1$ adult and no juveniles, find the
number of juveniles, adults and the total population after $3$ years.
Do the same if the population starts with one juvenile and no adults.
{\it Who seems to be `worth' more in terms of future population size
- adults or juveniles?}

\noindent
(c) Find the long term growth rate and the stable age distribution.

\bigskip

3. Compute
$$
\begin{bmatrix}
1 & 2 \\ -3 & 4
\end{bmatrix}
\begin{bmatrix}
2 \\ 5
\end{bmatrix}.
$$

\bigskip

4. For many bird species, the fertility and survivorship of adults
is independent of the age of the adult bird.  Thus, we can think of
the population as composed of two classes, juveniles and adults.
Set up a model based on a $2 \times 2$ matrix to describe the
dynamics of a bird population without density dependence.  Be sure
to define any parameters or variables that you introduce.

\bigskip

5. For a life table, $l_x$ is the fraction of newborn individuals
that survive to age $x$, and $S_x$ is the probability of surviving
from age $x$ to  age $x+1$.

\noindent
(a) Why is $l_0$ always $1$?

\noindent
(b) Express $l_{x+1}$ in terms of $l_x$ and $S_x$?

\bigskip

6. Given the following values of $l_x$ and $m_x$, find $r$ accurate
to two decimal places.  (Hint: Try different values of $r$ in Euler's
equation.  If the sum is too large, try a larger value of $r$; if it
is too small, try a smaller value of $r$.)
$$
\begin{matrix}
x & l_x & m_x \\  \hline
0 & 1.0 & 0   \\
1 & 0.6 & 0   \\
2 & 0.5 & 0   \\
3 & 0.4 & 3   \\
4 & 0.3 & 4   \\
5 &  0  & 0
\end{matrix}
$$
Compute the stable age distribution $c_x$ and the reproductive value
$v_x$ for this population, and graph them against the age $x$.  For
what age $x$ is $v_x$ maximized?  {\it Why?}






\end{document}
