\documentclass[reqno,12pt]{amsart}
%\documentclass[final,letter]{siamltex}
%\documentstyle{amsppt}


\usepackage{amssymb}
\usepackage{amscd}
\usepackage{graphicx}
\usepackage{overcite}

\setlength{\oddsidemargin}{0.0in}
\setlength{\evensidemargin}{0.0in}
\setlength{\textwidth}{6.5in}

\def\eee{\textrm{e}}

\begin{document}
\title{M2E03 - Introduction to Modelling - Assignment 2}
\maketitle

\begin{center}
\pagestyle{myheadings} 
\thispagestyle{empty}

\bigskip

\textsc{Due in class - 9:00 am, Wednesday, October 15}

\bigskip

\textsc{Late assignments will receive a grade of zero.}

% \bigskip
%
% You do not have to answer the questions that are written in {\it italics}.

\bigskip
\bigskip
\end{center}

\noindent
1.  Consider a single locus with two alleles $A$ and $B$, in a
population for which generations are non-overlapping.  Assume
that there is no selection related to this locus.  Suppose
that in the current generation, the genotype frequencies are
$p_{AA} = \frac 13$, $p_{AB} = \frac 13$, and $p_{BB} = \frac 13$.
What will the genotype frequencies be for each of the next three
generations?


\bigskip

\noindent
2. In a population, you find that the frequency of $A$ and $B$ are
$0.95$ and $0.05$, respectively.  Assume that the population is at
equilibrium.

\noindent
(a)  Assuming no mutation, determine parameter combinations that
could lead to this observation on the basis of heterozygote
advantage.  In other words, pick values for $w_{AA}$, $w_{AB}$,
and $w_{BB}$.  You might find it handy to use the parameterization
where $w_{AA} = 1-s$, $w_{AB} = 1$, and $w_{BB} = 1-t$.  In this
case, you need to select a value for $s$ and a value for $t$.
The choices are not unique; to do this you may need to first
pick a value for $s$ and then find $t$.

\noindent
(b)  Again, assume that there is no mutation.  Use the convention
that $w_{AA} = 1-s$, $w_{AB} = 1$, and $w_{BB} = 1-t$.  Find $t$
as a function of $s$ so that the gene frequencies $p$ and $q$ are
as given above.

\noindent
(c)  Determine parameter combinations that could lead to this
observation on the basis of mutation-selection balance with
$A$ recessive.  Here you need to find particular values of
$\mu$ and $s$, noting that $s$ is not the same here as in
parts (a) and (b).  Find $\mu$ as a function of $s$.


\noindent
(d)  Determine parameter combinations that could lead to this
observation on the basis of mutation-selection balance with
$A$ recessive.  Here you need to find particular values of
$\mu$ and $s$.  Find $\mu$ as a function of $s$.


\bigskip

\noindent
3. In many areas of the world where malaria was common, there
are alleles that are lethal when homozygous (giving sickle-cell
anemia) but are advantageous when heterozygous by conferring
resistance to malaria.  (These alleles can reach a frequency as
high as $40 \%$.)

\noindent
(a) If we assume that the fitness of $hh$ individuals is zero,
and the polymorphic equilibrium frequency of $h$ is $0.1$, what
are the relative fitnesses $w_{HH}$ and $w_{Hh}$ of genotypes
$HH$ and $Hh$, respectively.

\noindent
(b) Suppose the frequency of $h$ at the polymorphic equilibrium
is given by $p$.  Let $W = w_{HH}/w_{Hh}$.  Find $W$ as a function
of $p$.  Calculate $W$ when $p=.9$, $p=.8$, $p=.7$ and $p=.6$.
\bigskip

\noindent
4.  Consider a population for which there are non-overlapping
generations.  Let the size of the population at time $t$ be given
by $x_t$.  Since population sizes should not be negative, we are
only interested in the $x_t \ge 0$.  Suppose that due to crowding
effects, the population size follows the rule
$$
x_{t+1} = f(x_t) 
$$
where
$$
f(x) = \mu x (1-x)
$$
and $0 < \mu < 4$.  Find the equilibria for this system.  Note that we
are only interested in non-negative values for the equilibria.  By
considering the derivative of $f$ evaluated at each equilibrium,
determine (when possible) for what values of $\mu$ each equilibrium
is stable.











\end{document}
