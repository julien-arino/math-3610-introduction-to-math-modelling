\documentclass[reqno,12pt]{amsart}

\setlength{\oddsidemargin}{0.0in}
\setlength{\evensidemargin}{0.0in}
\setlength{\textwidth}{6.5in}

\def\eee{\textrm{e}}
\def \dNdt{\frac{dN}{dt}}
\def \dxdt{\frac{dx}{dt}}
\def \dydt{\frac{dy}{dt}}

\begin{document}

\begin{center}
{\bf M2E03 - INTRODUCTION TO MODELLING - MIDTERM}
\end{center}

\vspace{2cm}

\noindent
Name:\_\_\_\_\_\_\_\_\_\_\_\_\_\_\_\_\_\_\_\_\_\_\_\_\_\_\_\_\_\_\_\_\_\_\_\_\_\_\_\_\_\_\_\_\_\_\_\_\_\_\_\_\_\_

\vspace{2cm}

\noindent
Student Number:\_\_\_\_\_\_\_\_\_\_\_\_\_\_\_\_\_\_\_\_\_\_\_\_\_\_\_\_\_\_\_\_\_\_\_\_\_\_\_\_\_

\vspace{1.5cm}
\noindent
Notes, books and scrap paper are not allowed.

\vspace{1.5cm}
\noindent
CALCULATORS:  You may only use the McMaster Standard FX-991 Calculator.

\vspace{1.5cm}
\noindent
You must sign the class list.

\vspace{1.5cm}
\noindent
Please put out your Student Card for checking.


\newpage

\begin{center}
{\bf Formulas}
\end{center}

\vspace{.5cm}
$$
1 = \sum_{x=0}^\infty \eee^{-rx} l_x m_x
$$
\vspace{.5cm}
$$
R_0 = \sum_{x=0}^\infty l_x m_x
$$
\vspace{.5cm}
$$
\frac{v_x}{v_0} =
\frac{ \sum_{y=x}^\infty \eee^{-ry} l_y m_y}{\eee^{-rx} l_x}
$$
\vspace{.5cm}
$$
1 = \int_{x=0}^\infty \eee^{-rx} l(x) m(x) dx
$$
\vspace{.5cm}
$$
R_0 = \int_{x=0}^\infty l(x) m(x) dx
$$
\vspace{.5cm}
$$
\frac{v(x)}{v(0)} =
\frac{ \int_{y=x}^\infty \eee^{-ry} l(y) m(y) dy}{\eee^{-rx} l(x)}
$$
\vspace{.5cm}
$$
p' = p \frac{w_A}{\bar w}
$$
\vspace{.4cm}
$$
w_A = 1-p s
$$
\vspace{.4cm}
$$
w_B = 1-q t
$$
\vspace{.4cm}
$$
\bar w = 1 - p^2 s -q^2 t
$$
\vspace{.4cm}
$$
q = \frac \mu s
$$
\vspace{.4cm}
$$
q = \sqrt{\frac \mu s}
$$








\newpage

\noindent
1.  Consider a species that lives for two years and may reproduce
at the end of year one or year two.  Assume that the population is
undergoing density independent growth.

\noindent
a. Set up the Leslie matrix where

- the mean number of offspring that 0-year olds have the following
year is 1

- the mean number of offspring that 1-year olds have the following
year is 4

- the probability that a 0-year old survives to be a 1-year old
is 0.1.

\noindent
b.  Suppose that the initial population consists of ten 0-year
olds and six 1-year olds.  How many 0-year olds and how many
1-year olds will there be one year later.

\noindent
c.  Find the growth rate and the stable age distribution.

\newpage

\noindent
2.  Consider a species that lives for two years and may reproduce
at the end of year one or year two.  Assume that the population is
undergoing density independent growth and that

- the mean number of offspring that 0-year olds have the following
year is $m_0$

- the mean number of offspring that 1-year olds have the following
year is $m_1$

- the probability that a 0-year old survives to be a 1-year old
is $S_0$.

\noindent
Show that if $m_0>1$ then the growth rate is greater than one.

\newpage

3. a.  Regarding vertical life tables, let $r$ be the
intrinsic rate of increase and let $R_0$ be the net
reproduction rate.  Show that $r>0$ if and only if
$R_0>1$.

\vspace{8cm}

\noindent
b.  Consider following the vertical life table, for which
$x$ is time measured in years, $l_x$ is the fraction of
newborns that survive to age $x$, and $m_x$ is the average
number of offspring that an $x$-year old has in the following
year.
$$
\begin{matrix}
x & l_x  & m_x \\  \hline
0 & 1.0  & 0   \\
1 & 0.95 & 0   \\
2 & 0.9  & 4   \\
3 & 0.6  & 7   \\
4 & 0.3  & 2   \\
5 & 0.1  & 0   \\
6 &  0   & 0   \\
\end{matrix}
$$
Find the net reproduction rate.

\newpage

\noindent
4.a.  Consider a single locus with two alleles $A$ and $B$.  Let
$p$ be the frequency of gene $A$.  For each of the following
sets of fitnesses, state the (biologically meaningful)
equilibrium values of $p$.

\noindent
i.  $(w_{AA},w_{AB},w_{BB}) = (1.2 , 1, 1.3)$.

\vspace{1cm}
\noindent
ii.  $(w_{AA},w_{AB},w_{BB}) = (0.9 , 1, 1.2)$.

\vspace{1cm}
\noindent
iii.  $(w_{AA},w_{AB},w_{BB}) = (0.8 , 1, 0.7)$.

\vspace{1cm}
\noindent
iv.  $(w_{AA},w_{AB},w_{BB}) = (2.7 , 3.0, 2.4)$.

\vspace{1cm}

\noindent
b.  If $(w_{AA},w_{AB},w_{BB}) = (1-s, 1, 1-t)$, then
under what circumstances is there an unstable polymorphic
equilibrium?

\vspace{2cm}

\noindent
5.  Consider one locus with two alleles $Y$ and $Z$.  Suppose
the homozygote fitnesses are $w_{YY} = 1$ and $w_{ZZ} = .999$.
Suppose that a fraction $\mu$ of genes of type $Y$ mutate to
become genes of type $Z$.  Let $y$ be the frequency of gene $Y$
and let $z$ be the frequency of gene $Z$.  Suppose that the
mutation-selection balance occurs when $y=.93$ and $z=.07$.

\noindent
a.  Assuming $Y$ is dominant and $Z$ is recessive, find $\mu$.

\noindent
b.  Assuming $Y$ is recessive and $Z$ is dominant, find $\mu$.

\newpage

\noindent
6. Suppose the differential equation that describes the population
size of a particular species is
$$
\dNdt = 4 N \bigl( 1 - 8 N^3 \bigr)
$$

\noindent
a.  Find the equilibria.

\noindent
b.  Determine the stability of each equilibrium.

\newpage

\noindent
7.  Suppose the size of a particular population is described by
the differential equation
$$
\dNdt = N f(N)
$$
where the graph of $f(N)$ is:

\vspace{6cm}

\noindent
a.  State all equilibrium values of $N$.

\noindent
b.  For which initial values of $N$ do we get
$$
\lim_{t \to \infty} N(t) = 4?
$$

\noindent
c.  For which initial values of $N$ do we get
$$
\lim_{t \to \infty} N(t) = 9?
$$

\newpage

\noindent
8.  Consider two species of plants that are undergoing density
independent growth.  Suppose that a particular annual plant
species produces, on average, $80$ seeds that germinate.  Suppose
that an oak tree (a perennial plant species) produces, on average,
$5$ seeds that germinate.  Let $s_j$ be the probability that a seed
which has germinated survives to produce seeds.  Suppose that the
probability that an oak tree survives to the next year is $.98$.

\noindent
a.  For what value of $s_j$ are the growth rates of the two plant
species the same.

\noindent
b.  Which species grows faster (in population size) for $s_j=.02$.

\noindent
c.  If $s_j=.04$, and each species starts with a population of
$20$, what will the size of each species be after $5$ years.

\newpage

\noindent
9.  Suppose a species of insect has non-overlapping generations
and that the population (measured in billions) in generation $t$
obeys the equation
$$
N_{t+1} = N_t \eee^{4(1- 3 N_t)}.
$$

\noindent
a.  Find all equilibria.

\noindent
b.  Determine the stability of each equilibrium.

\newpage

\noindent
10.  Consider the following competition model for two species with
popuation sizes $x$ and $y$.
\begin{align*}
\dxdt &= 2 x (1 - 3x - y) \\
\dydt &= 6 y (1 - 2x - 4y)
\end{align*}

\noindent
a.  Sketch the phase plane, including the nullclines and the
equilibria (clearly labeled).  Show the general direction of
flow at different parts of the phase plane.

\noindent
b.  If $x(0), y(0) > 0$, then find
$$
\lim_{t \to \infty} \bigl( x(t), y(t) \bigr).
$$



\end{document}
