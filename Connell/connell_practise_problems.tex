\documentclass[reqno,12pt]{amsart}

\setlength{\oddsidemargin}{0.0in}
\setlength{\evensidemargin}{0.0in}
\setlength{\textwidth}{6.5in}

\def\eee{\textrm{e}}
\def \dNdt{\frac{dN}{dt}}
\def \dxdt{\frac{dx}{dt}}
\def \dydt{\frac{dy}{dt}}


\begin{document}
\title{M2E03 - Introduction to Modelling - Practise Problems}
\maketitle

\noindent
1.  Consider a population of insects that is undergoing density
independent population growth, with an initial population of 100.

\noindent
a.  Use a continuous time model to predict the population size
after 6.3 weeks if the intrinsic rate of increase is 0.8 insects
per week.

\noindent
b.  Use a discrete time model (with non-overlapping generations)
to predict the population size after 8 generations if each
individual has an average number of 2.23 offspring in the
next generation.

\vspace{1cm}

\noindent
2.  Consider an animal that lives for two years and many reproduce
at the end of year 1 or year 2.  Assume that the population is
undergoing density independent growth.

\noindent
a. Set up the Leslie matrix where

- the mean number of offspring that 0-year olds have the following
year is 0.6

- the mean number of offspring that 1-year olds have the following
year is 2.1

- the probability that a 0-year old survives to be a 1-year old
is 0.4.

\noindent
b.  Find the growth rate and the stable age distribution.

\noindent
c.  Suppose that the initial population consists of five 0-year
olds and eight 1-year olds.  How may 0-year olds and how many
1-year olds will there be two years later.

\vspace{1cm}

\noindent
3.  Consider an organism that lives for 5 years and can
reproduce at the end of any year from one to five.  Let
the mean number of offspring that a $j$-year old has in
the following year be $m_j$ with

$m_0 = 0.1$

$m_1 = 0.2$

$m_2 = 0.6$

$m_3 = 0.5$

$m_4 = 0.7$

$0 = m_5 = m_6 = m_7 = \dots$.

\noindent
Let the probability that a $j$-year old survives to age $j+1$
be given by $S_j$ with

$S_0 = .8$

$S_1 = .9$

$S_2 = .6$

$S_3 = .3$

$0 = S_4 = S_5 = S_6 = \dots$

\noindent
Find the Leslie matrix.

\vspace{1cm}

\noindent
4. Consider following the vertical life table, for which
$x$ is time measured in years, $l_x$ is the fraction of
newborns that survive to age $x$, and $m_x$ is the average
number of offspring that an $x$-year old has in the following
year.
$$
\begin{matrix}
x & l_x  & m_x \\  \hline
0 & 1.0  & 0   \\
1 & 0.9  & 0   \\
2 & 0.85 & 4   \\
3 & 0.6  & 5   \\
4 & 0.2  & 2   \\
5 & 0.1  & 0   \\
6 &  0   & 0   \\
7 &  0   & 0
\end{matrix}
$$
a.  Find the intrinsic rate of growth $r$.

\noindent
b.  Find the net reproduction rate $R_0$.

\vspace{1cm}

\noindent
5.  Consider a single locus with 2 alleles $A$ and $B$.  Let
$p$ be the frequency of gene $A$.

\noindent
a.  Suppose the fitnesses are given by
$(w_{AA},w_{AB},w_{BB}) = (1.1 , 1, 1.3)$.

I. \hspace{.04cm} Give the (biologically meaningful) equilibrium
values of $p$.

II. For each equilibrium, determine whether it is stable or unstable.

\noindent
b.  Do the same for $(w_{AA},w_{AB},w_{BB}) = (1.2 , 1, 0.7)$.

\noindent
c.  Do the same for $(w_{AA},w_{AB},w_{BB}) = (2.4 , 2, 1.4)$.

\vspace{1cm}

\noindent
6.  Consider one locus with two alleles $A$ and $B$.  Suppose
the homozygote fitnesses are $w_{AA} = 1$ and $w_{BB} = .98$.
Suppose that a fraction $\mu$ of genes of type $A$ mutate to
become genes of type $B$.  Let $p$ be the frequency of gene $A$
and let $q$ be the frequency of gene $B$.  Suppose that the
mutation-selection balance occurs when $p=.92$ and $q=.08$.

\noindent
a.  Assuming $A$ is dominant and $B$ is recessive, find $\mu$.

\noindent
b.  Assuming $A$ is recessive and $B$ is dominant, find $\mu$.

\vspace{1cm}

\noindent
7. Suppose the differential equation that describes the population
size of a particular species is
$$
\dNdt = 2 N \bigl( 1 - 8 N^{\frac 32} \bigr)
$$

\noindent
a.  Find the equilibria.

\noindent
b.  Determine the stability of each equilibrium.

\vspace{1cm}

\noindent
8.  Suppose the size of a particular population is described by
the differential equation
$$
\dNdt = N f(N)
$$
where the graph of $f(N)$ is:

\vspace{8cm}

\noindent
a.  State all equilibrium values of $N$.

\noindent
b.  For which initial values of $N$ do we get
$$
\lim_{t \to \infty} N(t) = 7?
$$

\noindent
c.  For which initial values of $N$ do we get
$$
\lim_{t \to \infty} N(t) = 12?
$$

\vspace{1cm}

\noindent
9.  Suppose a species of butterfly has non-overlapping generations
and that the population (measured in millions) in generation $t$
obeys the equation
$$
N_{t+1} = N_t \eee^{3(1- \frac 15 N_t)}.
$$

\noindent
a.  Find all equilibria.

\noindent
b.  Determine the stability of each equilibrium.

\vspace{1cm}

\noindent
10.  Consider two species of plants that are undergoing density
independent growth.  Suppose that a forget-me-not (an annual plant
species) produces, on average, $52$ seeds that germinate.  Suppose
that an oak tree (a perennial plant species) produces, on average,
$5$ seeds that germinate.  Suppose that the probability that a seed
which has germinated survives to produce seeds is $.02$.  Let $s$
be the probability that an oak tree survives to the next year.

\noindent
a.  For what value of $s$ is the growth rate of the two plant
species the same.

\noindent
b.  Which species grows faster (in population size) for $s=.96$.

\noindent
c.  If $s=.96$, and each species starts with a population of
$100$, what will the size of each species be after $15$ years.

\vspace{1cm}

\noindent
11.  Consider the following competition model for two species with
population sizes $x$ and $y$.
\begin{align*}
\dxdt &= 4 x (1 - 3x -2y) \\
\dydt &= 7 y (1 - \frac 12 x - 4y)
\end{align*}

\noindent
a.  Sketch the phase plane, including the nullclines and the
equilibria (clearly labeled).  Show the general direction of
flow at different parts of the phase plane.

\noindent
b.  If $x(0), y(0) > 0$, then find
$$
\lim_{t \to \infty} \bigl( x(t), y(t) \bigr).
$$



\end{document}
