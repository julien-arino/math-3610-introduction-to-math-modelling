\documentclass[reqno,12pt]{amsart}
%\documentclass[final,letter]{siamltex}
%\documentstyle{amsppt}

\usepackage{nopageno}

\usepackage{amssymb}
\usepackage{amscd}
\usepackage{graphicx}
\usepackage{overcite}

\input xy
\xyoption{all}

\setlength{\oddsidemargin}{0.0in}
\setlength{\evensidemargin}{0.0in}
\setlength{\textwidth}{6.5in}

\def\eee{\textrm{e}}
\def\dxdt{\frac{dx}{dt}}
\def\dydt{\frac{dy}{dt}}
\def\dHdt{\frac{dH}{dt}}
\def\dLdt{\frac{dL}{dt}}

\begin{document}
\title{M2E03 - Introduction to Modelling - Assignment 4}
\maketitle

\begin{center}
% \pagestyle{myheadings} 
% \thispagestyle{empty}

% \textsc{\bf M2E03 - Introduction to Modelling - Assignment 4}

\textsc{Due:  Noon, Wednesday, December 1}

\textsc{Location:  Locker C108 in the basement of Hamilton Hall}


\textsc{Late assignments will receive a grade of zero.}
\end{center}

\vspace{1cm}

\noindent
1.  Consider a non-fatal infectious disease (like the common cold) for
which each individual in the population is either susceptible or infective.
Assume that the average number of contacts that each infective has with
susceptibles is $c$ times the number of susceptibles.  Let the probability
that a contact between an infective and a susceptible results in transmission
of the disease be given by $\beta$.  Let the average duration of infectivity
be $1/\gamma$.  Assume that the rate of birth is proportional to the total
population size, with all new individuals being added to the susceptible
class, and that the death rate (not directly related to the disease) for
each population subgroup is proportional to the size of that subgroup.

\bigskip
\noindent
a.  Draw the transfer diagram.

\bigskip
\noindent
b.  Give differential equations which describe the rate of change
of the number of susceptibles and the number of infectives.

\bigskip
\noindent
c.  Make the additional assumption that the total population size
remains constant.  Rewrite the differential equation.

\bigskip
\noindent
d.  For the situation for which the population size is constant, do
the following.

i.\;\;  Find the equilibria and determine when each equilibrium is stable.

ii.\;  Find the basic reproduction number $R_0$.


\newpage
\noindent
2.  Consider the $SIR$ model where a fraction $p$ of all
newborns are {\bf successfully} vaccinated, given by the transfer
diagram below.  Assume that the model is for a town of size
$50,000$ and that the total population size is constant.
$$
\xymatrix{
\ar[d]_{(1-p)bN} &&&& \ar[d]^{pbN}	\\
S \ar[rr]^{c \beta SI} \ar[d]^{d S} & &
I \ar[rr]^{\gamma I} \ar[d]^{d I} & &
R \ar[d]^{d R} & &   \\
&&&&&&
}
$$

\bigskip
\noindent
a.  Write down the differential equations for this model.

\bigskip
Suppose $c=.01$ contacts per month per infective per susceptible,
the probability of transmission given that a contact occurs is
$\beta = .03$, the average duration of infectivity (before recovering)
is $\frac 12$ month, the life expectancy is $70$ years.
\bigskip

\noindent
b.  Find $R_0$ (as a function of $p$).

\bigskip
\noindent
c.  Find $R_0$ in the absence of vaccination (i.e. $p=0$).

\bigskip
\noindent
d.  What fraction of newborns has to be {\bf successfully}
vaccinated in order to make $R_0$ less than one?

\bigskip
\noindent
e.  Suppose the vaccine is successful in $95 \%$ of the cases in which
it is used.  What fraction of newborns has to be vaccinated to make
$R_0$ less than one?




\newpage
\noindent
3.  {\bf Spreading Rumors:}  Construct a model for spreading rumors making
use of the following assumptions.  The total population size $N$ is
constant with noone entering or leaving the population.  The population
is split into three subgroups: the uncool (who have not heard the rumor),
the rumor spreaders, and the cool (who have heard the rumor, but are no
longer spreading it).  Let the sizes of these groups be given by $U$,
$R$, and $C$ respectively.

As soon as someone in the uncool group hears the rumor, they enter the
group of rumor spreaders.  Each rumor spreader tells the rumor at rate
$k$.  Thus, the rate at which the rumor is being told is $k$ times the
number of rumor spreaders.  So, the rate at which uncool people hear the
rumor is $k$ times the number of rumor spreaders times the {\it fraction}
of the population consisting of uncool people.

When a rumor spreader tells the rumor to someone who has already heard
it (i.e. someone in the rumor spreading group or the cool group), they
stop spreading the rumor and enter the cool group.  Thus, the rate at
which rumor spreaders enter the cool group is $k$ times the number of
rumor spreaders times the {\it fraction} of the population consisting
of rumor spreaders and cool people.

\vspace{1cm}

\noindent
a.  Draw the transfer diagram.

\bigskip
\noindent
b.  Write down the differential equations.

\bigskip
\noindent
c.  Rewrite the equations for $\frac{dU}{dt}$ and $\frac{dR}{dt}$
without using the variable $C$ by making the substitution $C=N-U-R$.

\bigskip
\noindent
d.  Find the equilibria for the two dimensional system given in part (c).

\bigskip
\noindent
e.  Suppose that when the rumor starts, it is initially known by a
very small fraction (much less than half) of the population and that
this group of people are all rumor spreaders.  Suppose that everyone
else in the population is in the uncool group.  Then initially, this
should give $\frac{dR}{dt}>0$ and $R$ should increase.  (Check that
your equations in part (c) support this.)  $U$ will decrease (from
nearly $N$) until the rumor dies out.  (Check that your equations in
part (c) support this.)  {\bf At what value of $U$ does $R$ stop
increasing?} Eventually the rumor will die out when $R$ goes to zero.
{\bf When the rumor dies out, are there more people in the uncool
group or in the cool group?}


\newpage

\noindent
4.  For each of the following payoff matrices, determine which pure
strategies are evolutionarily stable.

\noindent
$$
\begin{matrix}
\begin{matrix}
& \begin{matrix} A & B & C \end{matrix}       \\
\begin{matrix} A \\ B \\ C \end{matrix}         &
\begin{bmatrix}
5 & 1 & 1       \\
1 & 8 & 4       \\
4 & 5 & 4
\end{bmatrix}
\end{matrix}
& \qquad &
\begin{matrix}
& \begin{matrix} D & E \end{matrix}   \\
\begin{matrix} D \\ E \end{matrix}              &
\begin{bmatrix}
7 & 8   \\
9 & 2
\end{bmatrix}
\end{matrix}
& \qquad &
\begin{matrix}
& \begin{matrix} F & G \end{matrix}   \\
\begin{matrix} F \\ G \end{matrix}              &
\begin{bmatrix}
-1 & 0   \\
1 & 2
\end{bmatrix}
\end{matrix}
\end{matrix}
$$

\vspace{3cm}

\noindent
5.  Consider the following situation.  Two roommates are each
considering the snow-shovelling strategy that they will adopt
for the coming winter.  Each has the option of being Responsible
(and shovelling the snow whenever they see that it needs shovelling)
or Delinquent (and never shovelling the snow).  If the front steps
are not shovelled, then there is a risk of injury to each roommate
which is assigned a cost $C$.  The cost (in time and effort) of
keeping the front steps clear is $E$.

If they are both Responsible, then they split the cost of clearing
the steps.  If one is Responsible and the other is Delinquent, then
the Responsible roommate expends all the energy to clear the steps
and neither roommate is at risk of injury.  If both roommates
are Delinquent, then the steps are never cleared and the expected
cost to each roommate is $C$.

Note that since there is a risk of injury if the steps are not
clear, and energy must be expended in order to clear the steps,
payoffs can be negative.

\noindent
a.  Give the payoff matrix.

\noindent
b.  Find the pure evolutionarily stable strategies (ESS) for $C \le E$.

\noindent
c.  Find an ESS for $C>E$.











\end{document}
