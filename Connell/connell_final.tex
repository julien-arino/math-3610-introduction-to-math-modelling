\documentclass[reqno,12pt]{amsart}
\usepackage{nopageno}

\setlength{\oddsidemargin}{0.0in}
\setlength{\evensidemargin}{0.0in}
\setlength{\textwidth}{6.5in}

\def\eee{\textrm{e}}
\def \dNdt{\frac{dN}{dt}}
\def \dxdt{\frac{dx}{dt}}
\def \dydt{\frac{dy}{dt}}
\def \dXdt{\frac{dX}{dt}}
\def \dYdt{\frac{dY}{dt}}

\begin{document}

\begin{center}
{\bf M2E03 - INTRODUCTION TO MODELLING - EXAM}
\end{center}

\vspace{.5cm}
\noindent
{\bf Time:} \hspace{.5cm} 3 Hours
\hspace{7cm}
{\bf Instructor:} \hspace{.5cm} C. McCluskey


\vspace{.5cm}
\noindent
{\bf Please write your solutions in the answer booklet.}
Notes, books and scrap paper are not allowed.
CALCULATORS:  You may only use the McMaster Standard FX-991 Calculator.

\vspace{.5cm}
\noindent
The exam consists of 8 questions.  Please make certain that your exam
paper has all 8 questions.  You are responsible for ensuring that your
copy of this paper is complete.  Bring any discrepancy to the attention
of an invigilator.

\vspace{.5cm}
\noindent
You must sign the class list.
Please put out your Student Card for checking.




\vspace{2cm}


\noindent
{\bf 1.}  Consider a species that lives for two years and may reproduce
at the end of year one or year two.  Assume that the population is
undergoing density independent growth.

\noindent
{\bf a.} Construct a model to describe the population size where

- the mean number of offspring that 0-year olds have the following
year is .5 

- the mean number of offspring that 1-year olds have the following
year is 4

- the probability that a 0-year old survives to be a 1-year old
is 0.2.

\noindent
{\bf b.}  Suppose that the initial population consists of fifty 0-year
olds and ten 1-year olds.  How many 0-year olds and how many
1-year olds will there be one year later.

\noindent
{\bf c.}  Find the growth rate and the stable age distribution.

\vspace{3cm}

\noindent
{\bf 2.a.}  Construct a continuous time model for the population size
$N$ of a country under the following circumstances.  The country
allows immigrants to enter the country at a constant rate $B$.  Also,
the birth rate $bN$ and the death rate $dN$ are proportional to the
population size, where $d>b$.  (HINT: {\it It may be helpful to draw
a flow diagram, similar to the transfer diagram that is used for an
epidemic model.})

\noindent
{\bf b.}  Find the equilibria and determine the stability of each equilbrium.

\noindent
{\bf c.}  Suppose $b=.008$, $d=.013$, and $B=180,000$.  If the
initial population is $30,000,000$, find
$$
\lim_{t \to \infty} N(t).
$$


\newpage
\vspace{1cm}

\noindent
{\bf 3.}  Consider the following equations which model competition between
two species whose population sizes (measured in millions) are given by $x$
and $y$.
$$
\begin{aligned}
\dxdt &= 2x (1 - 3x - 4y)	\\
\dydt &= 4y (1 -  x -  y)
\end{aligned}
$$

\noindent
{\bf a.}  Sketch the phase plane, including the nullclines and the
equilibria (clearly labeled).  Show the general direction of
flow at different parts of the phase plane.

\noindent
{\bf b.}  Assuming $x(0), y(0) > 0$, find
$$
\lim_{t \to \infty} \bigl( x(t), y(t) \bigr).
$$


\vspace{3cm}

\noindent
{\bf 4.}  Consider the following equations which model a predator-prey
relationship between two species whose population sizes (measured in
thousands) are given by $X$ and $Y$.
$$
\begin{aligned}
\dXdt &= X (  A - BX - CY)		\\
\dYdt &= Y ( -D + EX).
\end{aligned}
$$
Assume that $A$, $B$, $C$, $D$ and $E$ are positive constants.

\vspace{.5cm}

\noindent
{\bf a.} Which variable represents the population size of the predator
species?  Justify your answer.

\noindent
{\bf b.} Find positive constants $A$, $B$, $C$, $D$ and $E$ such that
the following statements are true.

- The intrinsic rate of growth of the prey is $3$.

- The carrying capacity of the prey in the absence of the predator is $10$.

- The impact that the predator has on the per capita growth rate of the prey

\hspace{.2cm} is $4$ times as large as the impact of the prey.

- In the absence of the prey, the predator undergoes exponential decay
with

\hspace{.2cm} decay function $\eee^{-2t}$.  (i.e. the predator has
an intrinsic growth rate of $-2$.)

- The rate of change of the predator is:

\hspace{.4cm} positive if the population size of the prey is greater than $6$,

\hspace{.4cm} and negative if the population size of the prey is less than $6$.

\noindent
{\bf c.}  Find the equilibria.

\noindent
{\bf d.}  Use the Jacobian matrix to determine the stability of the
equilibrium for which each population size is positive.



\newpage
\vspace{1cm}

\noindent
{\bf 5.}  Consider an infectious disease for which each individual
in the population is either susceptible ($S$) or infective ($I$).
The total population size $N$ is {\bf not} assumed to be constant.
Each individual in the population has (on average) $c N$ contacts.
A fraction $\frac SN$ of these contacts are with susceptibles, and
a fraction $\frac IN$ are with infectious individuals.  For a contact
between an infectious individual and a susceptible individual, the
probability that the disease is transmitted is $\beta$.

The rate at which infectious individuals recover from the disease (becoming
susceptible again) is $\gamma I$.  Births occur at a constant rate $B$
({\it not $bN$.}) The death rate for each population subgroup is
proportional to the size of that group, but with a higher death rate
for the infective group.  Thus, the death rate for the susceptibles is
$d S$ and the death rate for the infectives is $(d+\nu) I$.

\noindent
{\bf a.}  Draw the transfer diagram.

\noindent
{\bf b.}  Give the differential equations which describe the rate of
change of the number of susceptibles and the number of infectives.

\noindent
{\bf c.}  Find the equilibria.

\noindent
{\bf d.}  What is the average amount of time spent in the infectious
class?  Make sure to take into account recovery ($\gamma$), natural
death ($d$), and disease-related death ($\nu$).

\noindent
{\bf e.}  Find $R_0$.

\noindent
{\bf f.}  Use the Jacobian matrix to determine the stability of
the {\bf disease-free equilibrium} for $R_0 < 1$.

\noindent
{\bf g.}  Use the Jacobian matrix to determine the stability of
the {\bf disease-free equilibrium} for $R_0 > 1$.


\vspace{3cm}

\noindent
{\bf 6.}  For each of the following payoff matrices, determine which
{\bf pure} strategies are evolutionarily stable.

\noindent
{\bf a.}
$$
\begin{matrix}
  & \begin{matrix} A & B & C \end{matrix} 	\\
\begin{matrix} A \\ B \\ C \end{matrix}		&
\begin{bmatrix}
1 & 4 & 1	\\
8 & 7 & 1 	\\
4 & 6 & 2
\end{bmatrix}
\end{matrix}
$$

\noindent
{\bf b.}
$$
\begin{matrix}
  & \begin{matrix} D & E \end{matrix} 	\\
\begin{matrix} D \\ E \end{matrix}		&
\begin{bmatrix}
4 & 3 	\\
2 & 3
\end{bmatrix}
\end{matrix}
$$



\newpage
\vspace{1cm}

\noindent
{\bf 7.}  Consider the following game which can be used to model
the behaviour of female lions hunting in pairs.  Suppose that the
lions have killed an animal and that the animal is interpreted
as having an energy value $v$.  Each lion can either share or be
greedy.  If both lions share, then each lion will eat half of the
animal.  If one is greedy and the other is willing to share, then
the greedy lion will eat $4/5$ of the animal and leave the
remaining $1/5$ for her partner.  If both lions are greedy,
then they will each end up with half of the animal, but will
have expended energy equal to $3v/4$.

\noindent
{\bf a.}  Give the payoff matrix.

\noindent
{\bf b.}  Find an evolutionarily stable strategy.


\vspace{3cm}

\noindent
{\bf 8.}  Consider the following version of the Iterated Prisoner's
Dilemma.  In each round, each player can choose to either Cooperate ($C$)
or Defect ($D$).  The payoff matrix for each round is:
$$
\begin{matrix}
  & \begin{matrix} C & D \end{matrix} 	\\
\begin{matrix} C \\ D \end{matrix}		&
\begin{bmatrix}
3 & 0 	\\
5 & 1
\end{bmatrix}
\end{matrix}
$$

\vspace{.3cm}

We will consider two strategies which will play $2n$ rounds of the
Prisoner's Dilemma.  In each round their payoffs will be given by
the matrix above.

\vspace{.3cm}

One strategy is {Friendly-Tit-for-Tat} ($F$).  In the first round
$F$ cooperates with the other player.  In each following round, $F$
does whatever the other player did in the previous round.

\vspace{.3cm}

The other strategy is {Mean-Tit-for-Tat} ($M$).  In the first round
$M$ defects.  In each following round, $M$ does whatever the other
player did in the previous round.

\vspace{.3cm}

\noindent
{\bf a.}  Give the $2 \times 2$ payoff matrix for the Iterated Prisoner's
Dilemma for strategies $F$ and $M$ in which they play $2n$ rounds of the
Prisoner's Dilemma.

\vspace{.3cm}

\noindent
{\bf b.}  Find an evolutionarily stable strategy for this new payoff
matrix.










\end{document}
