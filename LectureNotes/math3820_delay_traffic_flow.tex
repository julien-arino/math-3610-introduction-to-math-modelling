\chapter{A delayed model of traffic flow}
\label{chap:dde_traffic}


\section{A delayed model of traffic flow}
In the traffic flow model \eqref{eq:ode_traffic_flow} of Chapter~\ref{chap:traffic_flow}, reaction time is instantaneous.
In practice, this is known to be incorrect: reflexes and psychology play a role.
It takes at least a few instants to acknowledge a change of speed in the car in front.
If the change of speed is not threatening, then you may not want to react right away.
When you press the accelerator or the brake, there is a delay between the action and the reaction..

We consider the same setting as for system \eqref{eq:ode_traffic_flow}, except that now, for $t>0$,
\begin{equation}\label{eq:dde_traffic_flow}
u'_{n+1}(t)=\lambda(u_n(t-\tau)-u_{n+1}(t-\tau)),
\end{equation}
for $n=1,\ldots,N-1$. Here, $\tau\geq 0$ is called the \emph{time delay} (or \emph{time lag}), or for short, \emph{delay} (or \emph{lag}).
If $\tau=0$, we are back to the model \eqref{eq:ode_traffic_flow}.

\paragraph{Initial data}
For a delay equation such as \eqref{eq:dde_traffic_flow}, initial data must be specified on an interval of length $N\tau$, left of zero.
\vskip0.5cm
This is easy to see by looking at the terms: $u(t-\tau)$ involves, at time $t$, the state of $u$ at time $t-\tau$. So if $t<\tau$, we need to know what happened for $t\in[-\tau,0]$.
So, normally, we specify initial data as
\[
u_n(t)=\phi(t)\textrm{ for }t\in[-\tau,0],
\]
where $\phi$ is some function, that we assume to be continuous. We assume $u_1(t)$ is known.
Here, we assume, for $n=1,\ldots,N$,
\begin{equation}\label{sys:IC_DDE_traffic_flow}
u_n(t)=0,\qquad t\leq (n-1)\tau.
\end{equation}
The explanation for this form of the initial data is simple. Since it takes $\tau$ units of time for each driver to adjust to changes in the speed of the car in front of them, car number 1 has to have moved for $\tau$ units of time before car number 2 makes any adjustment. In turn, car number 2 must have moved for $\tau$ units of time before car number 3 makes any adjustment. Repeating this, we obtain the form \eqref{sys:IC_DDE_traffic_flow}.

\paragraph{Important remark}
Although \eqref{eq:dde_traffic_flow} looks very similar to \eqref{eq:ode_traffic_flow}, you must keep in mind that it is in fact much more complicated.
\begin{itemize}
\item
A solution to \eqref{eq:ode_traffic_flow} is a continuous function from $\IR$ to $\IR$ (or to $\IR^n$ if we consider the system).
\item
A solution to \eqref{eq:dde_traffic_flow} is a continuous function in the space of continuous functions.
\item
The space $\IR^n$ has dimension $n$. The space of continuous functions has dimension $\infty$.
\end{itemize}
We can use the Laplace transform to get some understanding of the nature of the solutions.




\section{Laplace transform of the DDE traffic flow model}
Let
\[
U_{k+1}(s)=\L\{u_{k+1}(t)\}=\int_0^\infty e^{-st}u_{k+1}(t)dt.
\]
Since we have assumed initial data of the form
\[
u_n(t)=0\qquad\textrm{for } t\leq(n-1)\tau,
\]
we have
\[
U_{k+1}(s)=\int_{k\tau}^\infty e^{-st}u_{k+1}(t)ds.
\]
Since $u_{n+1}(t)=0$ for $t\leq n\tau$,
\begin{align*}
\int_0^\infty e^{-st}u_{n+1}'(t)dt &= \left[u_{k+1}(t)e^{-st}\right]_{k\tau}^\infty +s\int_{k\tau}^\infty e^{-st}u_{k+1}(t)dt \\
&= sU_{k+1}(s)
\end{align*}
and
\begin{align*}
\int_0^\infty e^{-st}u_{k+1}(t-\tau)dt &= \int_{(k-1)\tau}^\infty e^{-st}u_{k+1}(t-\tau)dt \\
&= \int_{(k-2)\tau}^\infty e^{-s(t+\tau)}u_k(\tau)d\tau \\
&= e^{-s\tau}U_k(s),
\end{align*}
since $e^{-st}u_{k+1}(t)\to 0$ for the improper integral to exist.
Note that we could have obtained this directly using the properties of the Laplace transform.

Multiply 
\[
u_{n+1}'(t)=\lambda(u_{n}(t-\tau)-u_{n+1}(t-\tau))
\]
by $e^{-st}$,
\[
e^{-st}u_{n+1}'(t)=\lambda e^{-st}(u_{n}(t-\tau)-u_{n+1}(t-\tau))
\]
integrate over $(0,\infty)$ (using the expressions found above),
\[
sU_{n+1}(s)=\lambda(e^{-s\tau}U_{n}(s)-e^{-s\tau}U_{n+1}(s))
\]
which is equivalent to
\[
U_{n+1}(s)=\frac{\lambda U_n(s)}{\lambda+se^{s\tau}}
\]
Thus, when $U_1(s)$ is known, we can deduce the values for all $U_n$.
Suppose 
\[
u_1(t)=\alpha \sin(\omega t)
\]
From the table of Laplace transforms, it follows that 
\[
U_1(s)=\alpha\frac{\omega}{s^2+\omega^2}
\]
Therefore,
\[
U_2=\frac{\lambda U_1(s)}{\lambda+se^{st}}
=\alpha\frac{\lambda}{\lambda+se^{st}}\;\frac{\omega}{s^2+\omega^2}
\]
and we can continue.. 
\vskip0.5cm
However, even though we know the solution in $s$-space, it is difficult to get the behavior in $t$-space, by hand, and maple does not help us either.

