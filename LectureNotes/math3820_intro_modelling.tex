\chapter{Introduction to mathematical modelling}
\label{chap:intro_model}


Mathematical modelling is an idealization of real-world problems. It is used to
help understand mechanisms. Be careful: a model can never be a completely
accurate representation of reality.


\section{Steps of the modelling process}
\begin{enumerate}
\item identify the most important processes governing the problem (theoretical assumptions)
\item identify the state variables (quantities studied)
\item identify the basic principles that govern the state variables (physical
laws, interactions)
\item express mathematically these principles in terms of state variables (choice of formalism)
\item make sure units are consistent
\end{enumerate}
Once a model is obtained
\begin{enumerate}
\item identify and evaluate the values of parameters
\item identify the type of mathematical techniques required for the analysis of the model
\item conduct numerical simulations of the model
\item validate the model: it must represent accurately the real process
\item verify the model: it must reproduce know states of the real process
\end{enumerate}

How to represent a problem:
\begin{itemize}
\item static vs dynamic
\item stochastic vs deterministic
\item continous vs discrete
\item homogeneous vs detailed
\end{itemize}
Formalism:
ODE, PDE, DDE, SDE, integral equations, integro-differential equations, Markov Chains, game theory, graph theory, cellular automata, L-systems $\dots$.




\section{Example: biological problems}
\begin{itemize}
\item ecology (predator-prey system, populations in competition $\dots$)
\item etology 
\item epidemiology (propagation of infectious diseases)
\item physiology (neuron, cardiac cells, muscular cells)
\item immunology
\item cell biology
\item structural biology
\item molecular biology
\item genetics (spread of genes in a population)
\end{itemize}







%\frame{\frametitle{Classification of models}
%
%\begin{itemize}
%\item Stochastic versus Deterministic
%\item
%\end{itemize}
%}
%
%\frame{\frametitle{Stochastic versus Deterministic}
%\begin{itemize}
%\item In some cases, the deterministic method expresses the average state of the actual stochastic process.
%\item The larger the initial population, the better the agreement is between experimental values and values obtained from the deterministic method.
%\end{itemize}
%}
