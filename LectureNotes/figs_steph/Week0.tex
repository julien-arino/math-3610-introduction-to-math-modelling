\documentclass{beamer}

\usepackage{graphics}
\usepackage{graphicx}
\usepackage{amsmath,amssymb,amsthm}
%\usepackage{subeqnarray}
%\usepackage{easybmat}
%\usepackage{subfigure}



%\usepackage{HA-prosper}
%\usepackage[dvips,letterpaper]{geometry}


\def\R{\mathbb{R}}
\def\Rzero{\mathcal{R}_0}
\def\diag{\textrm{diag}}
\def\tr{\textrm{tr}}
\def\det{\textrm{det}}
\def\sgn{\textrm{sgn}}
\def\imply{$\Rightarrow$}

\begin{document}
\maketitle


\frame{\frametitle{Mathematical modeling}
an idealization of the real-world problems and never a completely accurate representation
\begin{itemize}
\item Identify the most important processes governing the problem (Theoretical assumptions)
\item Identify the state variables (quantities studied)
\item Identify the basic principles that govern the state variables (Physical
Laws, interactions, $\dots$)
\item Express mathematically these principles in terms of state variables (Choice of the formalism)
%\item Units for each state variable
\item Identify and evaluate the values of parameters
\item Make sure of the consistence of units
\end{itemize}

}

\frame{
\begin{block}{How to represent a problem}
\begin{itemize}
\item Static vs dynamic
\item Stochastic vs Deterministic
\item Continous vs Discrete
\item Homogeneous vs Detailed
\end{itemize}
\end{block}
\begin{block}{Formalism}
ODE, PDE, DDE, SDE, Integral equation, integro-differential equations, Markov Chains, Game theory, Graph theory, Cellular automata, L-systems $\dots$
\end{block}
}



\frame{\frametitle{Biological problems}
\begin{itemize}
\item Ecology (Predator-Prey system, Populations in competition $\dots$)
\item Etology 
\item Epidemiology (Propagation of infectious diseases)
\item Physiology (Neuron, cardiac cells, muscular cells)
\item Immunology
\item Cell biology
\item Structural biology
\item Molecular biology
\item Genetics (Spread of genes in a population)
\item $\dots$
\end{itemize}
}



\frame{\frametitle{Lotka-Volterra Predator-Prey Model}
\begin{block}{Assumptions}
{\footnotesize
\begin{enumerate}
\item Growth of prey population is exponential in absence of predators. Prey grow in an unlimited way when no predation.
\item Predators depend on the presence of their prey to survive. Predators decline exponentially in absence of prey.
\item The rate of predation depends on the likehood that a victim is encountered by a predator
\item The rate of growth of the predator population is proportional to food intake.
\end{enumerate}}
\end{block}
\begin{block}{Equations}
{\footnotesize
\begin{itemize}
\item $x(t)$ biomass or population densities of the prey
\item $y(t)$ biomass or population densities of the predators
\end{itemize}
$$\frac{dx}{dt}=ax-bxy$$
$$\frac{dy}{dt}=-cy+dxy$$}
\end{block}
}



\frame{\frametitle{Lotka-Volterra Predator-Prey Model}
\begin{block}{Equations}
\begin{itemize}
\item $x(t)$ biomass or population densities of the prey
\item $y(t)$ biomass or population densities of the predators
\end{itemize}
$$\frac{dx}{dt}=ax-bxy$$
$$\frac{dy}{dt}=-cy+dxy$$
\end{block}
\begin{block}{Parameters}
\begin{itemize}
\item $a$: net growth rate of the prey population when predators are absent $a>0$, ($1/time$).
\item $c$: net death rate of the predators in the absence of prey, $c>0$, ($1/time$).
\item $b/d$: efficiency of predation = efficiency of converting a unit of prey into a unit of predator mass.
\end{itemize}
\end{block}
}





%\frame{\frametitle{Classification of models}
%
%\begin{itemize}
%\item Stochastic versus Deterministic
%\item
%\end{itemize}
%}
%
%\frame{\frametitle{Stochastic versus Deterministic}
%\begin{itemize}
%\item In some cases, the deterministic method expresses the average state of the actual stochastic process.
%\item The larger the initial population, the better the agreement is between experimental values and values obtained from the deterministic method.
%\end{itemize}
%}


\end{document}