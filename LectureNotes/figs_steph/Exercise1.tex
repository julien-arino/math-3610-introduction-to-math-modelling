\documentclass{article}

\usepackage{graphics}
\usepackage{graphicx}
\usepackage{amsmath,amssymb,amsthm}
%\usepackage{subeqnarray}
%\usepackage{easybmat}
%\usepackage{subfigure}



%\usepackage{HA-prosper}
%\usepackage[dvips,letterpaper]{geometry}


\def\R{\mathbb{R}}
\def\Rzero{\mathcal{R}_0}
\def\diag{\textrm{diag}}
\def\tr{\textrm{tr}}
\def\det{\textrm{det}}
\def\sgn{\textrm{sgn}}
\def\imply{$\Rightarrow$}

\title{Exercises: Difference equations}

\begin{document}
\maketitle



\section{Propagation of annual plants}
Plants produce seeds at the end of their growth season (August), after which they die. A fraction of these seeds survive the winter, and some of these germinate at the beginning of the season (May), giving rise to the new generation of plants. The fraction that germinates depends on the age of the seeds. 

Additional assumption: Seeds older than two years are no longer viable


\begin{itemize}
\item $\gamma$ number of seeds produced per plant in August
\item $\sigma$ fraction of seeds that survive a given winter
\item $\alpha$ fraction of one-year-old seeds that germinate in May
\item $\beta$ fraction of two-year-old seeds that germinate in May
\end{itemize}

State variables
\begin{itemize}
\item $p_n$ number of plants in generation $n$
\item $s_n$ number of one-year-old seeds
\item $s_{n}^{old}$ number of two-year-old seeds
\end{itemize}

In May, $$p_n=\alpha s_n+\beta s_{n}^{old}$$

$$s_{n}=\sigma \gamma p_{n-1}$$

$$s_{n}^{old}=\sigma (1-\alpha)s_{n-1}$$

therefore, we can expressed the model as a system of 2 first-order difference equations
\begin{equation}
\begin{array}{ll}
p_n=&\alpha s_n+\beta \sigma (1-\alpha)s_{n-1}\\
s_{n}=&\sigma \gamma p_{n-1}
\end{array}
\end{equation}

or as one second-order equation:
$$p_n=\alpha \sigma \gamma p_{n-1}+\beta \sigma (1-\alpha)\sigma \gamma p_{n-2}$$
\end{document}