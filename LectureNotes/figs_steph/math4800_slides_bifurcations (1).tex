\documentclass{beamer}

\usepackage{graphics}
\usepackage{graphicx}
\usepackage{amsmath,amssymb,amsthm}
%\usepackage{srcltx}
%\usepackage{subeqnarray}
%\usepackage{easybmat}
%\usepackage{subfigure}



%\usepackage{HA-prosper}
%\usepackage[dvips,letterpaper]{geometry}

\def\M{\mathcal{M}}
\def\R{\mathcal{R}}
\def\D{\mathcal{D}}
\def\C{\mathcal{C}}
\def\I{\mathcal{I}}
\def\L{\mathcal{L}}
\def\IC{\mathbb{C}}
\def\IN{\mathbb{N}}
\def\IR{\mathbb{R}}
\def\IK{\mathbb{K}}
\def\II{\mathbb{I}}
\def\Rzero{\mathcal{R}_0}
\def\diag{\textrm{diag}}
\def\tr{\ensuremath{\mathsf{tr}}}
\def\det{\ensuremath{\mathsf{det}}}
\def\Span{\ensuremath{\mathsf{span}}}
\def\sign{\ensuremath{\mathsf{sign}}}
\def\sgn{\textrm{sgn}}
\def\imply{$\Rightarrow$}
\def\dbint{\int\!\!\!\int}
\def\dbintb{\mathop{\int\!\!\!\!\int}}
\def\tpint{\int\!\!\!\int\!\!\!\int}

\newtheorem{proposition}{Proposition}

%\setbeamertemplate{theorems}[numbered]
\setbeamertemplate{navigation symbols}{}
\setbeamertemplate{footline}
{%
\quad\insertsection\hfill p. \insertpagenumber\quad\mbox{}\vskip2pt
}

\title{Bifurcations}
\date{}

\begin{document}
\maketitle
%%%%%%%%%%%%%%
%%%%%%%%%%%%%%
\section{General context}
\frame[plain]{\tableofcontents[current]}

\frame{\frametitle{The general context of bifurcations}
Consider the discrete time system
\[
x_{t+1}=f(x_t)
\]
or the continuous time system
\[
x'=f(x).
\]
We start with a function $f:\IR^2\to\IR$, $C^r$ when a map is considered, $C^1$ when continuous time is considered.
\vskip1cm
In both cases, the function $f$ can depend on some parameters. We are interested in the differences of qualitative behavior, as one of these parameters, which we call $\mu$, varies.
}


\frame{
So we write
\begin{equation}\label{eq:DE}
x_{t+1}=f(x_t,\mu)=f_\mu(x_t)
\end{equation}
and
\begin{equation}\label{eq:ODE}
x'=f(x,\mu)=f_\mu(x)
\end{equation}
for $\mu\in\IR$.
}

\frame{\frametitle{Bifurcations}
\begin{definition}[Bifurcation]
Let $f_\mu$ be a parametrized family of functions. Then there is a \emph{bifurcation} at $\mu=\mu_0$ (or $\mu_0$ is a bifurcation point) if there exists $\varepsilon>0$ such that, if $\mu_0-\varepsilon<a<\mu_0$ and $\mu_0<b<\mu_0+\varepsilon$, then the dynamics of $f_a(x)$ are ``different'' from the dynamics of $f_b(x)$.
\end{definition}
\vskip0.5cm
An example of ``different'' would be that $f_a$ has a fixed point (that is, a 1-periodic point) and $f_b$ has a 2-periodic point.
\vskip0.5cm
Formally, $f_a$ and $f_b$ are \emph{topologically conjugate} to two different functions.
}


\frame{\frametitle{Topological conjugacy}
\begin{definition}
Let $f:D\to D$ and $g:E\to E$ be functions. Then $f$ \emph{topologically conjugate} to $g$ if there exists a homeomorphism $\tau:D\to E$, called a \emph{topological conjugacy}, such that $\tau\circ f=g\circ\tau$.
\end{definition}
}


\section{Some bifurcations in discrete-time equations}
\frame[plain]{\tableofcontents[current]}



\frame{\frametitle{Types of bifurcations (discrete time)}
Saddle-node (or tangent):
\[
x_{t+1}=\mu+x_t+x_t^2
\]
Transcritical:
\[
x_{t+1}=(\mu+1)x_t+x_t^2
\]
Pitchfork:
\[
x_{t+1}=(\mu+1)x_t-x_t^3
\]
Period doubling (or flip):
\[
x_{t+1}=\mu-x_t-x_t^2
\]
}


\frame{\frametitle{Discrete-time saddle-node}
\[
x_{t+1}=\mu+x_t+x_t^2
\]
\underline{Fixed points (FP)}
\begin{align*}
x=\mu+x+x^2 &\Leftrightarrow x^2=-\mu \\
&\Leftrightarrow x=\pm\sqrt{-\mu}
\end{align*}
So no real valued FP if $\mu>0$, 2 if $\mu<0$.
\vskip0.5cm
\underline{Stability of $\sqrt{-\mu}$}\quad $f'(x)=1+2x$, so, assuming $\mu<0$,
\[
f'(\sqrt{-\mu})=1+2\sqrt{-\mu}
\]
Thus
\[
|f'(\sqrt{-\mu})|<1\Leftrightarrow -1<1+2\sqrt{-\mu}<1\Leftrightarrow -1<\sqrt{-\mu}<0
\]
which is impossible. Therefore, $\sqrt{-\mu}$ is always repelling.
}

\frame{
\underline{Stability of $\sqrt{-\mu}$}\quad assuming $\mu<0$,
\[
f'(-\sqrt{-\mu})=1-2\sqrt{-\mu}
\]
Thus
\begin{align*}
|f'(-\sqrt{-\mu})|<1 &\Leftrightarrow -1<1-2\sqrt{-\mu}<1 \\ 
&\Leftrightarrow -1<-\sqrt{-\mu}<0 \\
&\Leftrightarrow 0<\sqrt{-\mu}<1\\
& \Leftrightarrow -1<\mu<0
\end{align*}
So, for $-1<\mu<0$, the FP $-\sqrt{-\mu}$ is attracting.
}

\frame{\frametitle{Summary: discrete-time saddle-node}
}


\frame{\frametitle{Discrete-time period doubling}
\[
x_{t+1}=\mu-x_t-x_t^2
\]
\underline{FP}:
\[
x=\mu-x-x^2 \Leftrightarrow x^2+2x-\mu=0
\]
Discriminant: $\Delta=4+4\mu=4(1+\mu)$. So we get
\[
x_{1,2}=\frac{-2\pm 2\sqrt{1+\mu}}{2}=-1\pm \sqrt{1+\mu}
\]
}

\section{Some bifurcations in continuous equations}
\frame[plain]{\tableofcontents[current]}



\frame{\frametitle{Types of bifurcations (continuous time)}
\begin{itemize}
\item
Saddle-node
\[
x'=\mu-x^2
\]
\item
Transcritical
\[
x'=\mu x-x^2
\]
\item
Pitchfork 
\begin{itemize}
\item supercritical
\[
x'=\mu x-x^3
\]
\item subcritical
\[
x'=\mu x+x^3
\]
\end{itemize}
\end{itemize}
}



\section{Saddle-node}
\frame[plain]{\tableofcontents[current]}


\frame{\frametitle{Saddle-node for maps}
\begin{theorem}
Assume $f\in C^r$ with $r\geq 2$, for both $x$ and $\mu$. Suppose that
\begin{enumerate}
\item $f(x_0,\mu_0)=x_0$,
\item $f'_{\mu_0}(x_0)=1$,
\item $f''_{\mu_0}(x_0)\neq 0$ and
\item $\dfrac{\partial f}{\partial \mu}(x_0,\mu_0)\neq 0$.
\end{enumerate}
Then $\exists I\ni x_0$ and $N\ni \mu_0$, and $m\in C^r(I,N)$, such that
\begin{enumerate}
\item $f_{m(x)}(x)=x$,
\item $m(x_0)=\mu_0$,
\item the graph of $m$ gives all the fixed points in $I\times N$.
\end{enumerate}
\end{theorem}
}

\frame{
\begin{theorem}[cont.]
Moreover, $m'(x_0)=0$ and
\[
m''(x_0)=\dfrac{-\dfrac{\partial^2f}{\partial x^2}(x_0,\mu_0)}{\dfrac{\partial f}{\partial\mu}(x_0,\mu_0)}\neq 0.
\]
These fixed points are attracting on one side of $x_0$ and repelling on the other.
\end{theorem}
}


\frame{\frametitle{Saddle-node for continuous equations}
Consider the system $x'=f(x,\mu)$, $x\in\IR$. Suppose that $f(x_0,\mu_0)=0$. Further, assume that the following nondegeneracy conditions hold:
\begin{enumerate}
\item $a_0=\frac 12 \frac{\partial^2 f}{\partial x^2}(x_0,\mu_0)\neq 0$,
\item $\frac{\partial f}{\partial\mu}(x_0,\mu_0)\neq 0$.
\end{enumerate}
Then, in a neighborhood of $(x_0,\mu_0)$, the equation $x'=f(x,\mu)$ is topologically equivalent to the normal form
\[
x'=\gamma+\sign(a_0)x^2
\]
}

\frame{\frametitle{Saddle-node for continuous systems}
\begin{theorem}
Consider the system $x'=f(x,\mu)$, $x\in\IR^n$. Suppose that $f(x,0)=x_0=0$. Further, assume that
\begin{enumerate}
\item The Jacobian matrix $A_0=Df(0,0)$ has a simple zero eigenvalue,
\item $a_0\neq 0$, where
\[
a_0=\frac 12\langle p,B(q,q)\rangle =\left. \frac 12 \frac{d^2}{d\tau^2}\langle p,f(\tau q,0)\rangle \right|_{\tau=0}
\]
\item $f_\mu(0,0)\neq 0$.
\end{enumerate}
$B$ is the bilinear function with components
\[
B_j(x,y)=\sum_{k,\ell=1}^n \left.
\dfrac{\partial^2f_j(\xi,0)}{\partial\xi_k\partial\xi_\ell}
\right|_{\xi=0}x_ky_\ell,\quad j=1,\ldots,n
\]
and $\langle p,q\rangle=p^Tq$ the standard inner product.
\end{theorem}
}

\frame{
\begin{theorem}[cont.]
Then, in a neighborhood of the origin, the system $x'=f(x,\mu)$ is topologically equivalent to the suspension of the normal form by the standard saddle,
\begin{align*}
y' &= \gamma+\sign(a_0)y^2 \\
y_S' &= -y_S \\
y_U' &= y_U
\end{align*}
with $y\in\IR$, $y_S\in\IR^{n_S}$ and $y_U\in\IR^{n_U}$, where $n_S+n_U+1=n$ and $n_S$ is number of eigenvalues of $A_0$ with negative real parts.
\end{theorem}
}


\section{Pitchfork}
\frame[plain]{\tableofcontents[current]}


\frame{\frametitle{Pitchfork bifurcation}
The ODE $x'=f(x,\mu)$, with the function $f(x,\mu)$ satisfying
\[
-f(x,\mu)=f(-x,\mu)
\]
(f is odd),
\begin{gather*}
\dfrac{\partial f}{\partial x}(0,\mu_0) = 0 , 
\dfrac{\partial^2 f}{\partial x^2}(0,\mu_0) = 0, 
\dfrac{\partial^3 f}{\partial x^3}(0,\mu_0) \neq 0, \\
\dfrac{\partial f}{\partial r}(0,\mu_0) = 0, 
\dfrac{\partial^2 f}{\partial r \partial x}(0,\mu_0) \neq 0.
\end{gather*}
has a pitchfork bifurcation at $(x,\mu)=(0,\mu_0)$. The form of the pitchfork is determined by the sign of the third derivative:
\[
\dfrac{\partial^3 f}{\partial x^3}(0,\mu_0)
\left\{
  \begin{matrix}
    < 0, & \textrm{supercritical} \\
    > 0, & \textrm{subcritical} 
  \end{matrix}
\right.
\]
}




\section{Period doubling}
\frame[plain]{\tableofcontents[current]}


\frame{
\begin{theorem}[Period doubling bifurcation]
Assume $f$ is $C^r$ in $x$ and $\mu$, with $r\geq 3$, and that
\begin{enumerate}
\item $x_0$ is a fixed point for $\mu=\mu_0$, i.e., $f(x_0,\mu_0)=x_0$,
\item $f'_{\mu_0}(x_0)=-1$ (so, since $\neq 1$, there is a curve of fixed points $x(\mu)$ for $\mu$ close to $\mu_0$),
\item the derivative of $f'_\mu(x(\mu))$ with respect to $\mu$ is nonzero,
\[
\alpha=\left.\left[\frac{\partial^2f}{\partial\mu\partial x}+\frac 12\left(\frac{\partial f}{\partial \mu}\right)\left(\frac{\partial ^2f}{\partial x^2}\right)\right]\right|_{(x_0,\mu_0)}\neq 0,
\]
\item the graph of $f^2_{\mu_0}$ has nonzero cubic terms in its tangency with the diagonal (the quadratic term is zero):
\[
\beta=\left(\frac 1{3!}\;\frac{\partial^3 f}{\partial x^3}(x_0,\mu_0)\right)+\left(\frac 1{2!}\frac{\partial^2f}{\partial x^2}(x_0,\mu_0)\right)^2\neq 0
\]
\end{enumerate}
\end{theorem}
}

\frame{
\begin{theorem}[Period doubling bifurcation (cont.)]
Then there is a period doubling bifurcation at $(x_0,\mu_0)$. More specifically,
\begin{enumerate}
\item there is a differentiable curve of fixed points, $x(\mu)$, passing through $(x_0,\mu_0)$, and the stability of the fixed point changes at $\mu_0$;
\item which side of $\mu_0$ is attracting depends on the sign of $\alpha$;
\item there is a differentiable curve $\gamma$ passing through $(x_0,\mu_0)$, such that $\gamma\setminus\{(x_0,\mu_0)\}$ is the union of hyperbolic period 2 orbits;
\item $\gamma$ is tangent to $\IR\times\{\mu_0\}$ at $(x_0,\mu_0)$, so $\gamma$ is the graph of a function $\mu=m(x)$, with $m'(x_0)=0$ and $m''(x_0)=-2\beta/\alpha\neq 0$;
\item the stability of the period 2 orbit depends on $\beta$: if $\beta>0$, it is attracting, if $\beta<0$, it is repelling.
\end{enumerate}
\end{theorem}
}


\section{Hopf}
\frame[plain]{\tableofcontents[current]}

\frame{\frametitle{Poincar\'e map}
Consider
\begin{equation}\label{eq:sys}
x'=f(x)
\end{equation} 
If $\Gamma$ is a periodic orbit of \eqref{eq:sys} through $x_0$, and $\Sigma$ is a hyperplane perpendicular to $\Gamma$ at $x_0$, then for any point $x\in\Sigma$ close enough to $x_0$, the solution through $x$ at $t=0$, $\phi_t(x)$, crosses $\Sigma$ again at a point $P(x)$ near $x_0$.
\vskip1cm
The mapping $x\mapsto P(x)$ is the \emph{Poincar\'e map}.
}

\frame{
\begin{theorem}
Let $E$ be an open subset of $\IR^n$ and $f\in C^1(E)$. Suppose that $\phi_t(x_0)$ is a periodic solution of \eqref{eq:sys} of period $T$, and that
\[
\Gamma=\{x\in\IR^n:\quad x=\phi_t(x_0),\quad 0\leq t\leq T\}
\]
is contained in $E$. Let $\Sigma$ be the hyperplane orthogonal to $\Gamma$ at $x_0$, i.e.,
\[
\Sigma=\{x\in\IR^n:\quad (x-x_0)\cdot f(x_0)=0\}.
\]
Then there exists $\delta>0$ and a unique function $\tau(x)$, defined and continuously differentiable for $x\in\mathcal{N}_\delta(x_0)$, such that $\tau(x_0)=T$ and
\[
\phi_{\tau(x)}(x)\in\Sigma
\]
for all $x\in\mathcal{N}_\delta(x_0)$. For $x\in\mathcal{N}_\delta(x_0)\cap\sigma$,
\[
P(x)=\phi_{\tau(x)}(x)
\]
is the \emph{Poincar\'e map} for $\Gamma$ at $x_0$.
\end{theorem}
}


\frame{\frametitle{Example}
Consider the system
\begin{align*}
x' &= -y+x(\mu-x^2-y^2) \\
y' &= x+y(\mu-x^2-y^2)
\end{align*}
Transform to polar coordinates:
\begin{align*}
r' &= r(\mu-r^2) \\
\theta' &= 1
\end{align*}
}

\frame{
\begin{align*}
r' &= r(\mu-r^2) \\
\theta' &= 1
\end{align*}
has solution
\[
r(t)=\frac{\sqrt{(1+e^{-2\mu t}C\mu)\mu}}{1+e^{-2\mu t}C\mu}
\]
\[
\theta(t)=t+\theta_0
\]
}


\frame{\frametitle{Hopf bifurcation}
\begin{theorem}[Hopf bifurcation theorem]
Consider the system
\begin{equation}\label{sys:hopf}
\frac{d}{dt}
\begin{pmatrix}
x\\ y
\end{pmatrix}
=
\begin{pmatrix}
a_{11}(\mu) & a_{12}(\mu) \\
a_{21}(\mu) & a_{22}(\mu)
\end{pmatrix}
\begin{pmatrix}
x\\ y
\end{pmatrix}
+
\begin{pmatrix}
f_1(x,y,\mu)\\
f_2(x,y,\mu)
\end{pmatrix},
\end{equation}
with $\mu\in\IR$ a parameter. Suppose $f_1,f_2\in C^3$, that the origin is an equilibrium of \eqref{sys:hopf}, and that the matrix
\[
J(\mu)=\begin{pmatrix}
a_{11}(\mu) & a_{12}(\mu) \\
a_{21}(\mu) & a_{22}(\mu)
\end{pmatrix}
\]
is valid in a neighborhood of the origin. Additionally, suppose that the eigenvalues of $J(\mu)$ are $\alpha(\mu)+i\beta(\mu)$, with $\alpha(0)=0$ and $\beta(r)\neq 0$, satisfying the transversality condition 
\[
\left.\frac{d\alpha}{d\mu}\right|_{\mu=0}\neq 0.
\]
\end{theorem}
}

\frame{
\begin{theorem}[Hopf bifurcation theorem, cont.]
Then, in any open set $\mathcal{U}\ni(0,0)$ in $\IR^2$ and for any $\mu_0>0$, there exists $\bar\mu$, $|\bar\mu|<\mu_0$, such that \eqref{sys:hopf} has a periodic solution for $\mu=\bar\mu$ in $\mathcal{U}$ with approximate period $T=2\pi/\beta(0)$.
\end{theorem}
}

\frame{\frametitle{Another formulation}
\begin{theorem}[Hopf bifurcation]
Let $x'=A(\mu)x+F(\mu,x)$ be a $C^k$ planar vector field, with $k\geq 0$, depending on the scalar parameter $\mu$ such that $F(\mu,0)=0$ and $D_xF(\mu,0)=0$ for all $\mu$ sufficiently close enough to the origin. Assume that the linear part $A(\mu)$ at the origin has the eigenvalue $\alpha(\mu)\pm i\beta(\mu)$, with $\alpha(0)=0$ and $\beta(0)\neq 0$. Furthermore, assume the eigenvalues cross the imaginary axis with nonzero speed, i.e.,
\[
\left.\frac{d}{d\mu}\alpha(\mu)\right|_{\mu=0}\neq 0.
\]
Then, in any neighborhood $\mathcal{U}\ni(0,0)$ in $\IR^2$ and any given $\mu_0>0$, there exists a $\bar\mu$ with $|\bar\mu|<\mu_0$ such that the differential equation $x'=A(\bar\mu)x+F(\bar\mu,x)$ has a nontrivial periodic orbit in $\mathcal{U}$.
\end{theorem}
}

\frame{\frametitle{Supercritical or subcritical Hopf?}
Transform the system into
\[
\frac{d}{dt}
\begin{pmatrix}
x\\ y
\end{pmatrix}
=
\begin{pmatrix}
\alpha(\mu) & \beta(\mu) \\
-\beta(\mu) & \alpha(\mu)
\end{pmatrix}
\begin{pmatrix}
x\\ y
\end{pmatrix}
+
\begin{pmatrix}
f_1(x,y,\mu)\\
g_1(x,y,\mu)
\end{pmatrix}
=\begin{pmatrix}
f(x,y,\mu)\\
g(x,y,\mu)
\end{pmatrix}
\]
The Jacobian at the origin is
\[
J(\mu)=\begin{pmatrix}
\alpha(\mu) & \beta(\mu) \\
-\beta(\mu) & \alpha(\mu)
\end{pmatrix}
\]
and thus eigenvalues are $\alpha(\mu)\pm i\beta(\mu)$, and $\alpha(0)=0$ and $\beta(0)>0$.
}

\frame{\frametitle{Supercritical or subcritical Hopf? (cont.)}
Define
\begin{align*}
C &= f_{xxx}+f_{xyy}+g_{xxy}+g_{yyy} \\
& \quad+ \frac{1}{\beta(0)}\left(-f_{xy}\left(f_{xx}+f_{yy}\right)+g_{xy}\left(g_{xx}+g_{yy}\right)+f_{xx}g_{xx}-f_{yy}g_{yy}\right),
\end{align*}
evaluated at $(0,0)$ and for $\mu=0$.
Then, if $d\alpha(0)/d\mu>0$,
\begin{enumerate}
\item If $C<0$, then for $\mu<0$, the origin is a stable spiral, and for $\mu>0$, there exists a stable periodic solution and the origin is unstable ({\bf supercritical Hopf}).
\item If $C>0$, then for $\mu<0$, there exists an unstable periodic solution and the origin is unstable, and for $\mu>0$, the origin is unstable ({\bf subcritical Hopf}).
\item If $C=0$, the test is inconclusive.
\end{enumerate}
}

\frame{\frametitle{Supercritical Hopf}
Here, $y_1,y_2$ are the variables, $\beta$ is the bifurcation parameter.
\begin{center}
\includegraphics[width=0.8\textwidth]{SuperHopf}
\end{center}
}

\frame{\frametitle{Subcritical Hopf}
Here, $y_1,y_2$ are the variables, $\beta$ is the bifurcation parameter.
\begin{center}
\includegraphics[width=0.8\textwidth]{SubHopf}
\end{center}
}


\frame{\frametitle{Example: predator-prey system}
Predator-prey system:
\begin{subequations}
\begin{align}
x' &= ax-bxy \\
y' &= cxy -dy
\end{align}
\end{subequations}
where $a,b,c,d>0$.
}


\frame{\frametitle{Example: a general chemostat}
Chemostat:
\begin{subequations}
\begin{align}
S' &= D(S^0-S)-u(S)x \\
x' &= (g(S)-D_1)x
\end{align}
\end{subequations}
with $u,g\in C^1$ such that
\begin{enumerate}
\item $u(0)=g(0)=0$,
\item $\exists M_u,M_g\in\IR$ such that for all $S\in\IR_+$, $u(S)\leq M_u$ and $g(S)\leq M_g$.
\end{enumerate}
}



\end{document}
