\documentclass[12pt]{article}


\usepackage{amsmath,amsfonts,amssymb}
\usepackage{url}


 \newenvironment{week}[2]{\begin{center}
 \begin{tabular}{p{0.95\textwidth}}
 \mbox{}\\ \hline
 \end{tabular}
 \end{center}
 \subsection*{Week #1\quad (#2)}}{}
% No page break
%\newenvironment{week}[2]{
%\subsection*{Week #1\quad (#2)}}{}
%% Page break
%\newenvironment{weeknp}[2]{
%\newpage
%\subsection*{Week #1\quad (#2)}}{}

\setlength{\textwidth}{18cm}
\addtolength{\oddsidemargin}{-2.2cm}
\setlength{\textheight}{25.2cm}
\addtolength{\topmargin}{-3.5cm}


\begin{document}
\begin{center}
\Large MATH 3820 -- Introduction to Mathematical Modelling\\
Winter 2007\\
Midterm review program
\end{center}
\vskip2cm
%%%%%%%%%
%%%%%%%%%
%%%%%%%%%
The review program involves most of what we have covered until now. Specifically, for this midterm, you should be able to work on the following problems. This does not mean that all of these topics will be on the midterm, but that you are responsible for knowing how to treat them if they are.

\section*{By mathematical topic}
\subsection*{Linear and nonlinear regression}
Be able to carry out simple regression problems. During the midterm, the problems will be simplified. For example, the various quantities that are needed for the regression will already be computed.

\subsection*{Scalar ordinary differential equations (ODEs)}
Know how to solve explicitly simple scalar ODEs (separable, Bernoulli, integrating factors, etc.). Also, know how to study a scalar equation qualitatively (as we did for the ODE logistic, for example).


\subsection*{Discrete time scalar equations}
Know how to check that a point is a fixed point, a periodic point. Know how to check stability for these points. What is a bifurcation, how do you study them?


\subsection*{Probability distributions}
Know how to compute the average time spent in a state, if the time spent in a state has an exponential distribution. How can you go from a general distribution to an ODE model?

\subsection*{Local analysis of nonlinear ODEs $x'=f(x)$}
Know how to find equilibria (find $p$ such that $f(p)=0$), and how to study their local asymptotic stability (show that the Jacobian matrix, evaluated at $p$, has all eigenvalues with negative real parts).


\subsection*{Phase plane analysis}
Know how to plot and study nullclines, and to establish the existence of equilibria using nullclines. Know how to establish the direction field, and to use this information, together with the nullclines, to understand the behavior of the system.

\subsection*{Linear cascades}
Solve one first-order linear equation, and use the result in another equation, etc.


\subsection*{Linear systems and transition matrices}
Know how to use the method to compute the form of a solution. For this midterm, the general form (diagonalized or Jordan) of the matrix will be provided if it is needed.


\section*{By modeling topic}

\subsection*{Single population growth}
Understand the various models, and in particular, the modelling assumptions. You should be able to modify the models to take into account new hypotheses, if need be. Be aware of the different outcomes generated by three different modelling methods (ODEs, delay differential equations, discrete time equations), when used to model the same problem (logistic population growth).


\subsection*{Residence time}
ODE compartmental models result from very strong assumptions on the distributions of probability used to describe, for example, the time that individuals spend in various compartments.


\subsection*{Epidemic models}
Understand how the models were formulated. If you are asked to do it, you should be able to write a somewhat similar model describing a slightly different situation. Understand the notion of reproduction number, and of threshold behavior.


\subsection*{Chemostat models}
Understand the modelling ideas. What are the possible behaviors of the system?

\subsection*{Traffic flow}
Understand the formulation of the basic model. How would you modify the model to take into account the fact that not all driver react at the same rate to changes in speed?


\section*{A few words of advice}
{\bf 1.} The midterm is open book. This will help you, as you will not have to memorize the results. It helps, however, if you memorize a few of them. Those that you use a lot when you practice should be easy, and if you use a result more than once when practicing, make sure to try to remember it.

\noindent{\bf 2.} Be sure to read your notes enough times before the test, to know exactly what information you have, and where it is located. Discovering the notes during the test is an extremely bad idea.

\noindent{\bf 3.} Try to do the computations in the notes, and the practice midterm. It is one thing to see me do them on the board, a totally different one to do them yourself.

\end{document}